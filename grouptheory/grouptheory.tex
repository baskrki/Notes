\documentclass[12pt]{article}
\usepackage{graphicx}
\usepackage[a4paper, margin=1in]{geometry}
\usepackage{amsmath,amssymb,amsthm}
\usepackage{xcolor}
\usepackage{fancyhdr}
% \usepackage{libertinus}
\usepackage[hidelinks]{hyperref}
\usepackage{enumitem} 
\usepackage{tikz}
\usepackage{tikz-cd}
\usetikzlibrary{positioning}

\setlength{\headheight}{15pt}

\pagestyle{fancy}
\fancyhf{}
\setlength{\parindent}{0pt}



% Header and Footer
\fancyhead[L]{Group Theory}
\fancyhead[R]{\thepage}
\fancyfoot[L]{}
\fancyfoot[R]{}
\newcommand{\normal}{\unlhd}
\renewcommand{\footrulewidth}{0.4pt}
\renewcommand{\contentsname}{\centering\Huge Content}

%new command

%Theorems styles without numbering
{
\theoremstyle{definition}
\newtheorem{definition}{Definition}[section]
\newtheorem{proposition}{Proposition}[section]

\theoremstyle{plain}
\newtheorem{theorem}{Theorem}[section]
\newtheorem{lemma}{Lemma}

\theoremstyle{definition}
\newtheorem*{remark}{Remark}
}


\begin{document}

\begin{titlepage}
    \centering
    \vspace*{4cm} % push title down a bit

    {\Huge Group Theory\\[0.5em]
    % \Huge Theory\\[1em]
    \huge Notes\\[1em]}
    
    {\LARGE Basu Dev Karki\par}
\end{titlepage}

{
    \thispagestyle{plain}
    \tableofcontents
}

\eject
\section{Subgroups}
First, we'll define some things which will come in handy later on.
\subsection{Definitions and Examples}

\begin{definition}
    A subgroup $H$ of $G$ is subset of $G$ such that every group axiom holds on $H$ with the same group operation of $G$.
    If $H$ is a subgroup of $G$ we shall write $H \le G$.
\end{definition}

\begin{proposition}
    A subset $H$ of $G$ is a subgroup of $G$ if and only if
    \begin{enumerate}
        \item $H \neq \emptyset$ and
        \item $\forall x,y \in H$, $xy^{-1} \in H$
    \end{enumerate}
\end{proposition}

\begin{definition}
    A \textit{kernel} of a function $\varphi : G \to H$ are groups, is the set 
    \[ \ker(\varphi) = \{ g \in G \mid \varphi(g)=e_H \} \]
    where $G$ and $H$ are groups.
\end{definition}

\begin{proposition}
    A function $\varphi : G \to H$ is injective if and only if 
    $\ker (\varphi) = \{0\} $.
\end{proposition}

\begin{definition}
    A group action is a function $\mu : G \times A \to A$ such that
    \begin{enumerate}
        \item $\mu(g_1, \mu(g_2,a))=\mu(g_1 \cdot g_2,a)$
        \item $\mu(e,a)=a$
    \end{enumerate}
    To make the notation simple enough, we abbreviate the notation of
    $\mu(g,a)$ to $g \cdot a$ or sometimes $ga$.    
\end{definition}

\begin{remark}
    Instead of saying 'Group Action of $G$ on $A$', we saying group $G$
    acts on the set $A$. 
\end{remark}

\begin{definition}
    Let group $G$ act on set $A$. 
    A \textit{stabilizer of} $a$, where $a \in A$, is a the set consisting of elements
    which fixes $a$. i.e
    \[ G_a = \{ g \in G \mid g \cdot a = a \} \]
\end{definition}

\begin{proposition}
    The $G_a$ is a subgroup of $G$ for all $a \in A$.
\end{proposition}

\begin{proposition}
    Let the group $G$ act on a set $A$.             
    The relation $\sim$ defined on $A$ by 
    \[ a \sim b \iff a = hb \quad \text{ for some } h \in G\]
    is a equivalence relation.
\end{proposition}

\begin{definition}
    For each $a \in A$ the equivalence class under $\sim$ is called
    \textit{orbit} of $a$ under action of $G$. Thus,
    \[ \mathcal{O}(a) = \{ x \in A \mid x \sim a \} = \{ ha \mid h \in G \}\ \]
\end{definition}

\begin{definition}
    Let $G$ be an abelian group. Define
    \[ t=\{ g \in G \mid |g| < \infty \} \]
    and call it the torsion subgroup of $G$.
\end{definition}

You can check that the set is a subgroup of $G$.

\eject

\subsection*{Problems and Solutions}

\paragraph{Problem :} Find a non-abelian group $G$ such that the set of all elements with finite order is not a subgroup of $G$.

\vspace{4mm}
\textit{Solution :} $G=GL_2({\mathbb{Q}})$.
Take $a=\begin{pmatrix}
    0 && 1 \\
    1 && 0 
\end{pmatrix}$ 
and 
$b = \begin{pmatrix}
    0 && 2 \\
    1/2 && 0
\end{pmatrix}$.

Here, $|a|=|b|=2$ but $|ab|=\infty$.


\paragraph{Problem :} Let $H$ and $K$ be subgroups of $G$. Prove that $H \cup K$ is a subgroup if and only if $H \subseteq K$ or 
$K \subseteq H$.

\vspace{4mm}
\textit{Solution :}
The $(\Leftarrow)$ is pretty simple. For $(\Rightarrow)$, suppose neither of $H \subseteq K$ or 
$K \subseteq H$ is true. Then, there exists a element $h,k$  s.t $h \in H$ and $h \not \in K$ and $k \in K$ and $k \not \in H$.
But since $H \cup K$ is a subgroup, $h\cdot k$ must be either in $H$ or $K$. If $h \cdot k \in H$ then $h^{-1} \cdot (h \cdot k) \in H$
and if $h \cdot k \in K$ then $(h \cdot k) \cdot k^{-1} \in K$

\paragraph{Problem :} Let $F$ be any field. Define 
\[ SL_n(F) = \{ A \in GL_n(F) \mid \det(A)=1 \} \]
(called the \textit{special linear group}). Prove that $SL_n(F) \le GL_n(F)$.

\vspace{4mm}    
\textit{Solution :} If we use some basic properties of determinants we should be able to prove that this is a subgroup of the general linear group.
We know that,
\[ \det(A B) = \det(A) \cdot \det(B)\]

From this basic fact, we should be able to verify every subgroup axiom.

\paragraph{Problem :} Prove that the intersection of arbitrary amount of non-empty collection of subgroups of $G$ is also a subgroup of $G$.


\vspace{4mm}
\textit{Solution :} Let us suppose 
\[ K = \bigcap G_i \]
where $G_i$ are the subgroups. 

Let us take $a \in K$. Since, $a \in K$ that implies that $a \in G_i$. Since, of them are subgroups $a^{-1} \in G_i$ thus $a^{-1} \in K$.
It's easy to see that $e_G \in K$.Associativity is also pretty easy to check. If $a \in G$ and $b \in G$ then $ab \in G$ as $a, b\in G_i$ which means $ab \in G_i$.

\paragraph{Problem :} Let $A$ be an abelian group and fix some $n \in \mathbb{Z}$. Prove that the following subsets are subgroup of $A$,
\begin{enumerate}
    \item $\{a^n \mid a \in A\}$
    \item $\{a \in A \mid a^n =1 \}$
\end{enumerate}

\vspace{4mm}
\textit{Solution :} The problem can be easily solved if we know a facts about abelian group. Let $a,b \in A$ then
\begin{enumerate}
    \item $(ab)^n= a^n b^n$
    \item $(a^{-1})^n=(a^n)^{-1}$ \quad (\textit{This is true in general for all group }$A$)
\end{enumerate}

\paragraph{Problem :} Let $H$ be a subgroup of additive group of rational numbers with the property that $1/x \in H$ for every non-zero element of $x \in H$.
Prove that $H=0$ or $H=\mathbb{Q}$.

\vspace{4mm}
\textit{Solution :} If $H$ has no non-zero element then $H=\{0\}$. If $H$ has has a non-zero element $x$ then $x=\frac{a}{b}$ for some $a,b \in \mathbb{Z}$.
Since, $\frac{a}{b} \in H$ then $a \in H$ because of the additive nature of $H$. Since, $a \in H$ we have $\frac{1}{a} \in H$, thus $1 \in H$ as 
$a \cdot \frac{1}{a} = 1$. Since, $1 \in H$ we must have $-1 \in H$ as well. Thus, every integer $a$ is in $H$. Since, $a \in H$ we have $\frac{1}{a} \in H$
thus, $\frac{b}{a} \in H$ for any $b \in \mathbb{Z}$. Thus, every rational number can be obtained by this method. Thus, $\mathbb{Q}=H$.

\paragraph{Problem :} Show that $\{x \in D_{2n} \mid x^2 = 1\}$ is not a subgroup of $D_{2n}$ ($n>2$).


\vspace{4mm}
\textit{Solution :} We know that $(r^k s)^2=1$ for all $0 \le k \le n$. Thus, $(r^{k}s)(r^{j}s)= r^k (s r^j) s = r^{k-j}$. Thus $r^k=r^j \implies k=j$. But
since $n > 2$ we can take different $k,j$. Thus, the set is not closed and thus it cannot be a subgroup of $D_2n$.

\begin{remark}
    Here, if you do not know about the Dihedral groups, then it might be little confusing but, essentially 
    \[ D_{2n} = \{1,r,r^2,\ldots,r^{n-1},s,sr,\ldots,sr^{n-1} \} \] 
    This Dihedral group is the set of symmetries of a regular $n$-gon. You can check that
    \[ sr^k = r^{-k}s  \]
\end{remark}
\eject
\subsection{Centralizers and Normalizers, Stabilizer and Kernels}

We'll look at some important families of subgroups. Let $A$ be any non-empty subset of $G$.
\begin{definition}
    Define $C_G(A)=\{g \in G \mid g a g^{-1} = a \hspace{2mm}\forall a \in A\}$. This subset is called \textit{centralizers} of $A$ in $G$.
    The set is the collection of elements in $G$ which commutes with every element of $A$.
\end{definition}


\begin{proposition}
    $C_G(A) \le G$
\end{proposition}

\begin{proof}
    One can check that $1 \in C_G(A)$. Suppose $x \in C_G(A)$, then $xax^{-1}=a \implies a = x^{-1}ax$. Thus, $x^{-1} \in C_G(A)$.
    Let $x,y \in C_G(A)$ then 
    \begin{align*}
        (xy) a (xy)^{-1} &= (xy) a (y^{-1}x^{-1}) \\
        &= x(yay^{-1})x^{-1} \\
        &= xax^{-1} \\
        &=a
    \end{align*}
    Thus, $xy \in C_G(A)$.
\end{proof}

\begin{definition}
    Define $Z(G)=\{g \in G \mid gx=xg \hspace{2mm} \forall x \in G\}$. This is the set of elements in $G$ such that it commutes with every 
    other element of $G$. This is called the \textit{center} of $G$. 
\end{definition}

\begin{remark}
    You may notice that $Z(G)=C_G(G)$. Thus, $Z(G) \le G$.
\end{remark}

\begin{definition}
    Define $gAg^{-1}=\{gag^{-1} \in G \mid a \in A\}$. Define the \textit{normalizers}
    of $A$ in $G$ to be the set 
    \[ N_G(A) = \{g \in G \mid g A g^{-1} = A\} \]
\end{definition}

\begin{proposition}
    $N_G(A) \le G$ and $C_G(A) \le N_G(A)$.
\end{proposition}

\begin{proof}
    Similar proof like \textbf{Proposition 1.4}. 
\end{proof}

\textbf{Examples}

\begin{enumerate}
    \item Let $G$ be abelian group. Then $Z(G)=G$, also $N_G(A)=C_G(A)=G$.
    \item Let $G=D_8$. And let $A=\{1,r,r^2,r^3\}$ be a subgroup of rotations in $D_8$. We show that $C_{D_8}(A)=A$.
          Since, powers of $r$ commute with each other, we have $A \le C_{D_8}(A)$. One can check $s \not \in C_{D_8}(A)$
          as $sr = r^{-1}s \neq rs$. Now, if $a \not \in A$ and $a \in D_8$ then $a$ must be of the form $sr^i$. If $a \in C_{D_8}(A)$
          with $a=sr^i$ then $s=(sr^{i})(r^{-i})$, thus a contradiction.
\end{enumerate}

\subsubsection*{Stabilizer and Kernels of a Group Action}

We have already seen what a stabilizer is, now lets look at what a kernel of a group action is.

\begin{definition}
    A \textit{kernel} of a group $G$ acting on a set $A$ is the set of elements in $g$ such that it fixes every element in $A$. That is,
    if $\phi : G \times A \to A$ is a group action then 
    \[ \ker(\phi) = \{ g \in G \mid g \cdot s = s \hspace{2mm} \forall s \in A \} \]
\end{definition}

\begin{proposition}
    Kernel of a group action is a subgroup of the group G. 
\end{proposition}

\begin{proof}
    If $g \in \ker (\phi)$ then $g \cdot s = s \implies g^{-1} \cdot (g \cdot s )=g^{-1} \cdot s \implies s = g^{-1} \cdot s$.
    Thus, $g^{-1} \in \ker(\phi)$. Similarly, you can verify other axioms.
\end{proof}

We'll see that the centralizers, normalizers and kernels are some special case of facts that stabilizer and kernels of actions are subgroups.
Let $S=P(G)$ be the collection of all the subsets of group $G$, and let $G$ act on $S$ by \textit{conjugation} i.e 
\[ \phi : G \times S \to S \quad \text{where} \quad g \cdot A=gAg^{-1} \]
where $gAg^{-1}$ is defined just like in \textbf{Definition 1.9.}

\vspace{4mm}
Under this action, the stabilizer of $A$ is same as normalizer of $A$ i.e $N_G(A)=G_A$.
This is basically of the definition, $N_G(A)=\{g \in G \mid gAg^{-1}=A\} = \{g \in G \mid g \cdot A = A \} = G_A$. Thus, $N_G(A) \le G$.

\vspace{4mm}
Next Let the group $N_G(A)$ act on $A \subseteq G$ by conjugation. One can check that the centralizer of $A$ is the same as 
kernel of this action. Thus, $C_G(A)=\ker(\phi) \le N_G(A)$ and from the above argument $C_G(A) \le N_G(A) \le G \implies C_G(A) \le G$. 
One can also check that $G$ acting on $G$ by conjugation has kernel same as the center of the group i.e $Z(G)$ thus, $Z(G) \le G$

\eject

\subsection*{Problems and Solutions}

\paragraph{Problem :} Prove that $C_G(Z(G))=G$ and $N_G(Z(G)=G$.

\vspace{4mm}
\textit{Solution :} We already know that $C_G(Z(G))$ and $N_G(Z(G))$ are the subgroups of $G$. Thus, if we prove every element of 
$C_G(Z(G))$ and $N_G(Z(G))$ is also an element of $G$ then we're done. Let $a \in Z(G)$, then $g a = a g$ for any $g \in G$. 
Thus, $g \in C_G(Z(G))$. Since, $gZ(G)g^{-1}=\{gag^{-1} \mid a \in Z(G) \} = \{ a \mid a \in Z(G) \} = Z(G)$. Thus, for any $g \in G$
we have $gZ(G)g^{-1}=Z(G)$ which means that $N_G(Z(G))$ collects all the $g \in G$. Thus, $N_G(Z(G))=G$.

\paragraph{Problem :} If $A$ and $B$ are the subsets of $G$ such that $A \subseteq B$ then $C_G(B) \le C_G(A)$.

\vspace{4mm}
\textit{Solution :} Every element of $C_G(B)$ is in $C_G(A)$ as $xb=ba$ for all $b \in B$ so $xa = ax$ for all $a \in A$ thus, $x \in C_G(A)$.
Thus, we are done.

\paragraph{Problem :} Let $H$ be a subgroup of $G$.
\begin{enumerate}
    \item Show that $H \le N_G(H)$. 
    \item Show that $H \le C_G(H) \iff H$ is abelian.  
\end{enumerate}

\vspace{4mm}
\textit{Solution :}
For the first part, $gHg^{-1}=\{ghg^{-1} \mid h \in H\}$. If we let $g \in H$ be an arbitrary element then $\{ghg^{-1} \mid h \in H\}=H$. This,
can be proved by proving $\varphi_g : H \to H$ is a bijection for $g \in H$. Since, $g$ is arbitrary $H \le N_G(H)$.

For the second part, if $H$ is abelian then $g a = a g $ for every $a,g \in H$ thus $H \le C_G(H)$. If $H \le C_G(H)$ then $g a = a g$ for all
$a \in H$ and since $H$ is a subgroup of $C_G(H)$ every element of $H$ is in $C_G(H)$ that means $g a = a g$ for every $a,g \in H$. Thus
$H$ is abelian.

\paragraph{Problem :} Let $n \in \mathbb{Z}$ and $n \ge 3$. Prove the following
\begin{enumerate}
    \item $Z(D_{2n}) = \{1\}$ if $n$ is odd
    \item $Z(D_{2n}) = \{1,r^k\}$ if $n=2k$ 
\end{enumerate}

\vspace{4mm}
\textit{Solution :} We know that only elements that commute with powers of $r$ are powers of $r$. Thus, $r^i$ be the element that commutes with 
every element of $D_{2n}$. Then, $r^i (sr^i)=(sr^i)r^i \implies r^i(r^{-i}s) = s r^{2i} \implies s = s r^{2i} \implies r^{2i}=1 
\implies n \mid i$ if $n$ is odd. But $i < n$ so $i=0$. If $n=2k$ then $n \mid 2i \implies 2i = nk$ but $2i < 2n \implies 2 > k \implies k=1$.
Thus $i=n/2$.  

\paragraph{Problem :} Let $G=S_n$ and fix an $i \in \{1,2,3,\ldots,n\}$ and let $G_i=\{\sigma \in G \mid \sigma(i)=i\}$. Prove that $G_i$ is a
subgroup of $G$ and find $|G_i|$.

\vspace{4mm}
\textit{Solution :} The subgroup part of this is pretty easy. To find, $|G_i|$ we fix the map $i \to i$ and let the other maps vary. The number
of ways to do this is $(n-1)!$ and this is the size of the group.

\eject

\paragraph{Problem :} For any subgroup $H$ of $G$ and for any non-empty subset of $A$ in $G$ define $N_H(A)=\{h \in H \mid hAh^{-1}=A\}$.
Show that $N_H(A)= N_G(A) \cap H$ and deduce that $N_H(A)$ is a subgroup of $H$.

\vspace{4mm}
\textit{Solution :} $N_H(A)$ collects every $h \in H$ for which $hAh^{-1}=A$. $N_{G}(A) \cap H$ also collects $h \in H$ for which $hAh^{-1}=A$
thus $N_G(A) \cap H = N_H(A)$. To deduce $N_H(A)$ is a subgroup of $H$, you can easily check the axioms. 

\paragraph{Problem :} Let $H$ be a subgroup of order 2 in $G$. Show that $N_G(H)=C_G(H)$. Deduce that if $N_G(H)=G$ then $H \le Z(G)$.

\vspace{4mm}
\textit{Solution :} Since, $H$ has order $2$ $H$ must be $\{e,h\}$ where $h \neq e $ and $h^2=e$. Now, if $gHg^{-1}=H$ then $\{ghg^{-1} \mid g \in G\}
=\{e,ghg^{-1}\}=\{e,h\} \implies gh=hg$. Thus, $N_G(H)$ collects $g \in G$ which commutes with $h$ which is exactly $C_G(H)$. 
For the second part, since $N_G(H)=G$ that means $h$ commutes with every $g \in G$. Thus, $\{e,h\} \subseteq Z(G)$ and $H \le Z(G)$.

\paragraph{Problem :} Prove that $Z(G) \le N_G(A)$ for any subset $A$ of $G$.

\vspace{4mm}
\textit{Solution :} Since $Z(G)$ collects every $g \in G$ such that it commutes with every other element of $G$, it must commute with 
every element of $A$. Thus, $gAg^{-1}=\{gag^{-1} \mid g \in Z(G)\} = \{a \mid g \in Z(G)\} = A$ which means every $g \in Z(G)$ is also an 
element of $N_G(A)$.  
\eject
\subsection{Cyclic and Cyclic Subgroups}

\begin{definition}
    A group $G$ is called \textit{cyclic} if if can be generated by a single element i.e there is some $x \in G$ such that 
    $H= \{x^n \mid n \in \mathbb{Z}\}$. We write is as $G=\langle x \rangle$ and say $G$ is generated by $x$.
\end{definition}


\begin{proposition}
    If $H= \langle x \rangle$ then $|H|=|x|$.
\end{proposition}

\begin{proof}
    Suppose $|x|=n < \infty$ then $1,x,\ldots,x^{n-1}$ are all distinct. Thus $|H|$ is at least $n$. Now, using the division algorithm
    we can show that these are all of them.
    
    Suppose now $|x| = \infty$ then that means there is no finite $n \in \mathbb{Z}$ s.t $x^n = 1$. If $x^b=x^c$ then $x^{b-c}=1$ 
    contradicting the fact that there is no $n$ s.t $x^n=1$. Thus, all of the powers of $x$ are different and thus $|H|=\infty$.
\end{proof}

\begin{proposition}
    Let $G$ be a group and let $x \in G$. If $x^m=1$ for some $m \in \mathbb{Z}$ then $|x|$ divides $m$.
\end{proposition}

\begin{proposition}
    Any two cyclic group of same order are isomorphic. More specifically, 
    \begin{enumerate}
        \item If $n \in \mathbb{Z}^{+}$ and $\langle x \rangle$ and $\langle y \rangle$ are both cyclic groups of order $n$, then the map
                \[ \varphi: \langle x \rangle \to \langle y \rangle \]
                \[ x^k \to y^k \]
            is well defined and is an isomorphism.
        \item If $\langle x \rangle$ is an infinite cyclic group then
                \[\varphi: \mathbb{Z} \to \langle x \rangle\]
                \[k \to x^k\]
            is well defined and is an isomorphism.
    \end{enumerate}
\end{proposition}

\begin{proposition}
    Let $G$ be a group, let $x \in G$ and let $a \in \mathbb{Z}\setminus\{0\}$ then
    \begin{enumerate}
        \item If $|x|=\infty$ then $|x^a| = \infty$
        \item If $|x|=n < \infty$ then $|x^a|=\frac{n}{(n,a)}$
    \end{enumerate} 
\end{proposition}

\begin{proof}
    $1.$ is pretty simple. Suppose $|x^a|=k$ then $x^{ak}=1$ now $n \mid ak$.
    Write $(n,a)=d$ and $n=du$ and $a=dv$. Then, $du \mid dvk \implies u \mid k$. But
    \[ x^a = x^{dv} \implies (x^{du})^v = (x^{n})^v = 1  \]
    \[ \implies (x^{a})^u=1 \]
    \[ \implies k \mid u \]. 
    Thus, $k=u \implies n=dk \implies k = \frac{n}{d} = \frac{n}{(n,a)}$.
\end{proof}


\begin{proposition}
    Let $H=\langle x \rangle$. 
    \begin{enumerate}
        \item Assume $|x|= \infty$. Then $H = \langle x^a \rangle \iff a = \pm 1$.
        \item Assume $|x|=n < \infty$. Then $H = \langle x^a \rangle \iff (a,n)=1$.
            The number of generators of $H$ is $\varphi(n)$.  
    \end{enumerate}
\end{proposition}

\begin{proof}
    For $1.$ we know $x \in \langle x^a \rangle$ as $\langle x^a \rangle = \langle x \rangle$ thus 
    \[ x=x^{ak} \implies ak = 1 \implies a = \pm 1 \]
    
    For $2.$ we know $H=\langle x^a \rangle \implies |H|=|x^a| \iff |x|=|x^a|$ ,
    \[ \iff \frac{n}{(n,a)}=n \]
    \[ \iff (n,a)=1 \]
    Since, the number of positive integers less than $n$ and co-prime to $n$ are exactly $\varphi(n)$, thus the number of generators are
    exactly equal to $\varphi(n)$.
\end{proof}

\begin{theorem}
Let $H = \langle x \rangle$ be a cyclic group. 
\begin{enumerate}
    \item Every subgroup of $H$ is cyclic. More precisely, if $K \leq H$, then either 
    $K = \{1\}$ or $K = \langle x^d \rangle$, where $d$ is the smallest positive integer such that $x^d \in K$.
    
    \item If $|H| = \infty$, then for any distinct nonnegative integers $a$ and $b$, 
    $\langle x^a \rangle \neq \langle x^b \rangle$. 
    
    \item If $|H| = n < \infty$, then for each positive integer $a$ dividing $n$ 
    there is a unique subgroup of $H$ of order $a$. This subgroup is the cyclic group 
    $\langle x^d \rangle$, where $d = \tfrac{n}{a}$. \\
    Furthermore, for every integer $m$, 
    $\langle x^m \rangle = \langle x^{(n,m)} \rangle$, so that the subgroups of $H$ correspond 
    bijectively with the positive divisors of $n$.
\end{enumerate}
\end{theorem}

\begin{proof}
    $1.$ and $2.$ are pretty easy. For $3.$ the cyclic group $\langle x^{n/a} \rangle$ has order $a$. To prove uniqueness,
    suppose $K$ is any subgroup of $H$ with order $a$, then $\langle x^b \rangle = K$ where $b$ is the smallest positive 
    integer $b$ s.t $x^b \in K$(this is from $1.$). Thus,
    \[ |\langle x^{n/a} \rangle|=|\langle x^b \rangle| \implies \frac{n}{d}=a=\frac{n}{(n,b)}\]
    \[ \implies d=(n,b) \implies d \mid b \] 
    Hence, $\langle x^b \rangle \le \langle x^d \rangle $ and since they both have same order $\langle x^b \rangle = \langle x^d \rangle $.
    
    For the assertion on $3.$, one can prove that $\langle x^{m} \rangle  \le \langle x^{(n,m)}\rangle $ as $(n,m) \mid n$, and 
    since they have same order $\langle x^{m}\rangle = \langle x^{(n,m)} \rangle $. This means that the number of subgroups has a bijection
    with the divisors of $n$.
\end{proof}

\eject

\subsection*{Problems and Solutions}

\paragraph{Problem :} Find all subgroup of $\mathbf{Z}_{45} = \langle x \rangle$, giving a generator of each. Describe the containment 
between these subgroups.

\vspace{4mm}
\textit{Solution :} There are exactly $6$ different subgroups of $\mathbf{Z}_{45}$. Since there is a one to one correspondence between
divisors of $45$ and subgroups of $\mathbf{Z}_45$ we can list all of them,
\[ \{1\}, \langle r \rangle , \langle r^3 \rangle , \langle r^5 \rangle, \langle r^9 \rangle, \langle r^15 \rangle \]

\paragraph{Problem :} If $x$ is an element of a finite group $G$ and $|x|=|G|$. Prove that $G= \langle x \rangle$.

\vspace{4mm}
\textit{Solution :} Let $|x|=n$. We know that $\{1,x,\ldots,x^{n-1}\}$ is a subgroup of $G$. Since, it is a subgroup and has the same 
order as $G$ thus $G=\{1,x,\ldots,x^{n-1}\}=\langle x \rangle$.

\paragraph{Problem :} Let $\mathbf{Z}_{48}=\langle x \rangle$ and use isomorphism $\mathbb{Z}/48\mathbb{Z} \cong \mathbf{Z}_{48}$ with 
$[1] \mapsto x$ to find all the subgroups of $Z_{48}$.

\vspace{4mm}
\textit{Solution :} If $\langle [x] \rangle$ is a cyclic subgroup of $\mathbb{Z}/48\mathbb{Z}$ then $\langle \varphi([x]) \rangle$ is a subgroup
of $\mathbf{Z}_{48}$ where $\varphi$ is the isomorphic map.

\paragraph{Problem :} Let $\mathbf{Z}_{48}=\langle x \rangle$. For which integer $a$ does the map $\varphi_a$ defined by 
$\varphi_a : [1] \mapsto x^a$ extends to an isomorphism from $\mathbb{Z} / 48 \mathbb{Z}$ to $\mathbf{Z}_{48}$.

\vspace{4mm}
\textit{Solution :}  We already know it is an homomorphism as
\[ \varphi([u]+[v])=(x^{a})^{u+v}= (x^{a})^u (x^{a})^v = \varphi([u]) \varphi([v]) \]
But to be an isomorphism $x^{na}$ needs to cover $\mathbf{Z}_{48}$ for all $n \in \mathbb{Z}$. Thus,
\[ \mathbf{Z}_{48} = \langle x \rangle = \langle x^a \rangle \]
\[ \implies (48,a)=1 \]
So, for all the $a$ which are co-prime to $48$ the map, $\varphi_a$ is an isomorphism. 

\paragraph{Problem :} Let $\mathbf{Z}_{36}=\langle x \rangle$. For which integer $a$ does the map $\psi_a : [1] \mapsto x^a$ extend to an
well defined homomorphism from $\mathbb{Z}/48\mathbb{Z}$ onto $\mathbf{Z}_{36}$. Can $\psi_a$ ever be surjective?

\vspace{4mm}
\textit{Solution :}  One can check that the map is a homomorphism.  Now, we need to show that 
\[ [u]=[v] \implies \psi_a([u])=\psi_a([v]) \]
If $[u]=[v]$ then $u-v = 48 m$ 
\[ 1=\psi_a([0])=\psi_a([u-v])=x^{a(u-v)} = x^{48am} \]
\[ \implies 36 \mid 48 am \]
\[ \implies 3 \mid am \]
Since $3 \mid am$ must hold for all integer $m$, if $3 \nmid a$ then $3 \mid m$ for all integer $m$ which is clearly absurd thus $3 \mid a$.
Thus, $x^{48 \cdot 3k \cdot m} = 1 $ as $36 \mid 144km$. Thus,
\[ x^{a(u-v)}=1  \implies x^{au}=x^{av} \implies \psi_a([u])=\psi_a([v])\]

\paragraph{Problem :} Find a presentation for $\mathbf{Z}_n$ with one generator.

\vspace{4mm}
\textit{Solution :} $\mathbf{Z}_{n}=\langle r \mid r^n = 1 \rangle$.

\paragraph{Problem :} Show that if $H$ is any group with $h^n=1$ then there exists a unique homomorphism from 
$\mathbf{Z}_n=\langle x \rangle$ to $H$ such that $x \mapsto h$.

\vspace{4mm}
\textit{Solution :} Define $\psi : \mathbf{Z}_n \to H$ by $\psi(x^k)=h^k$. This is a homomorphism and is unique because the output is 
completely determined by $h$. 

\paragraph{Problem :} Show that if $H$ is any group and $h$ is an element of $H$, then there is a unique homomorphism from $\mathbb{Z}$ 
to $H$ such that $1 \to h$.

\vspace{4mm}
\textit{Solution :} Define $\psi : \mathbb{Z} \to H$ by $\psi(k)=h^k$. This is a homomorphism and is unique as the output is completely 
determined by $h$.

\paragraph{Problem :} Let $p$ be a prime and $n$ be a positive integer. Show that if $x$ is an element of the group $G$ such that $x^{p^n}=1$
then $|x|=p^m$ for some $m \le n$.

\vspace{4mm}
\textit{Solution :} We know that if $x^n=1$ then $|x|$ must divide $n$. Thus, $|x|$ must divide $p^n$ but the only divisors of $p^n$ are 
powers of $p$. Thus, $|x|=p^m$ for some $m \le n$.


\paragraph{Problem :} Show that $(\mathbb{Z}/2^n \mathbb{Z})^{\times}$ is not cyclic.

\vspace{4mm}
\textit{Solution :} Consider $\{1,-1\}$ and ${1,1+2^{n-1}}$. They both are subgroups of order $2$. But a cyclic group has exactly $1$ 
subgroup of order $d$, where $d$ is the divisor of order of the cyclic group $G$. But we found two distinct subgroups of the group with same
order.

\paragraph{Problem :} Let $G$ be a finite group and let $x \in G$.
  \begin{enumerate}
    \item Prove that if $g \in N_G(\langle x \rangle)$ then $gxg^{-1} = x^a$ for some $a \in \mathbb{Z}$.
    \item Prove conversely that if $gxg^{-1} = x^a$ for some $a \in \mathbb{Z}$ then $g \in N_G(\langle x \rangle)$. 
    [Show first that $g x^k g^{-1} = (gxg^{-1})^k = x^{ak}$ for any integer $k$, so that 
    $g \langle x \rangle g^{-1} \leq \langle x \rangle$. If $x$ has order $n$, show the elements 
    $gx^i g^{-1}$, $i=0,1,\dots,n-1$, are distinct, so that 
    $|g \langle x \rangle g^{-1}| = |\langle x \rangle| = n$ and conclude that 
    $g \langle x \rangle g^{-1} = \langle x \rangle$.]
  \end{enumerate}

\paragraph{Problem :} Let $G$ be a cyclic group of order $n$ and let $k$ be an integer relatively prime to $n$. 
  Prove that the map $x \mapsto x^k$ is surjective. Use Lagrange's Theorem 
  (Exercise 19, Section 1.7) to prove the same is true for any finite group of order $n$. 
  (For such $k$ each element has a $k$th root in $G$. It follows from Cauchy’s Theorem in 
  Section 3.2 that if $k$ is not relatively prime to the order of $G$ then the map $x \mapsto x^k$ 
  is not surjective.)

\paragraph{Problem :} Let $\mathbf{Z}_n$ be a cyclic group of order $n$ and for each integer $a$ let
  \[
    \sigma_a : \mathbf{Z}_n \to \mathbf{Z}_n \quad \text{by} \quad \sigma_a(x) = x^a \quad \text{for all } x \in Z_n.
  \]
  \begin{enumerate}
    \item Prove that $\sigma_a$ is an automorphism of $Z_n$ if and only if $a$ and $n$ are relatively prime.
    \item Prove that $\sigma_a = \sigma_b$ if and only if $a \equiv b \pmod{n}$.
    \item Prove that every automorphism of $Z_n$ is equal to $\sigma_a$ for some integer $a$.
    \item Prove that $\sigma_a \circ \sigma_b = \sigma_{ab}$. Deduce that the map 
    $a \mapsto \sigma_a$ is an isomorphism of $(\mathbb{Z}/n\mathbb{Z})^\times$ 
    onto the automorphism group of $Z_n$ (so $\operatorname{Aut}(Z_n)$ is an abelian group 
    of order $\varphi(n)$).
  \end{enumerate}

\eject
\subsection{Subgroups Generated by Subsets of a Group}

\begin{proposition}
    If $\mathcal{A}$ is any non empty collection of subsets of $G$ then the intersection of all members of $\mathcal{A}$ is also a 
    subgroup of $G$. 
\end{proposition}

\begin{proof}
    Trivial.
\end{proof}

\begin{definition}
    If $A$ is any subset of group $G$ define
    \[ \langle A \rangle = \bigcap_{\substack{A \subseteq H \\ H \le G}} H \]
    This is called subgroup generated by $A$.
\end{definition}

\begin{definition}
    Let $A=\{a_1,\ldots,a_n\}$ then define
    \[ \bar{A} = \{a_1^{\epsilon_1} a_2^{\epsilon_2} \cdots a_n^{\epsilon_n} \mid n \in \mathbb{Z}_{\ge 0}, a_i \in A, \epsilon_{i}=\pm1 \} \]
    where $\bar{A}=\{1\}$ if $A= \emptyset$.
\end{definition}

\begin{remark}
    Here, $a_i$'s need not to be distinct.
\end{remark}

\begin{proposition}
    $\bar{A}=\langle A \rangle$
\end{proposition}

\begin{proof}
    First we prove that $\bar{A}$ is a subgroup. Note that $\bar{A}\neq \emptyset$. If $a, b  \in \bar{A}$ then write 
    $a=a_1^{\epsilon_1} \cdots a_n^{\epsilon_n}$ and $b=b_1^{\delta_1} \cdots b_m^{\delta_m}$ then one can check that $ab^{-1} \in \bar{A}$.
    Thus, $\bar{A}$ is a subgroup of $G$.
    
    Now, since $A \subseteq \bar{A}$ as $a = a^1$ for every $a \in A$, we can say that $\langle A \rangle \subseteq \bar{A}$. It is because
    $\langle A \rangle$ is the intersection of all the subgroups containing $A$. Now, since $\langle A \rangle$ contains $A$ and is a group,
    it must contain every element of form $a_1^{\epsilon_1} \cdots a_n^{\epsilon_n}$ thus $\bar{A} \subseteq \langle A \rangle$. This
    completes the proposition.
\end{proof}

\eject

\subsection*{Problems and Solutions}

\paragraph{1.}  Prove that if $H$ is a subgroup then $\langle H \rangle = H$.

\vspace{4mm}
\textit{Solution :} From the definition,
\[ \langle H \rangle = \bigcap_{\substack{H \subseteq K \\ K \le G}} K \]
Since, $H \subseteq H$ and $H \le G$ thus $\langle H \rangle \subseteq H$. But also $H \subseteq \langle H \rangle$.

\paragraph{2.} Prove if $A$ is a subset of $B$ then $\langle A \rangle \le \langle B \rangle$. Give an example of $A \subseteq B$
with $A \neq B$ but $\langle A \rangle = \langle B \rangle$.

\vspace{4mm}
\textit{Solution :} From the definition we have,
\[ \langle B \rangle = \bigcap_{\substack{B \subseteq K \\ K \le G}} K \]
Since, $A \subseteq B$ we have $\langle A \rangle \le \langle B \rangle$. For the example, take $G=D_{16}$ and 
$A = \{r\}$ and $B=\{r,r^3\}$.

\paragraph{3.} Prove if $H$ is an abelian subgroup of $G$ then $\langle H, Z(G) \rangle$ is abelian. Give and explicit example 
of a abelian subgroup $H$ such that $\langle H, C_G(H) \rangle$ is not abelian.

\vspace{4mm}
\textit{Solution :} We know that, $\bar{A}=\langle A \rangle$ thus 
\[ \langle H,Z(G) \rangle = \{a_1^{\epsilon_1} \cdots a_n^{\epsilon_n}
\mid e_i = \pm 1 , n \in \mathbb{Z}_{\ge 0}, a_i \in H \cup Z(G) \}
\]
So, if you take two elements from $\langle H, Z(G) \rangle$, they will commute thus $\langle H, Z(G) \rangle$ is an abelian group.
For the example, choose $H=\{1,r^2\}$ and $G=D_8$.

\paragraph{3. } Prove that $H$ is a subgroup then $H$ is generated by $H-\{1\}$.

\vspace{4mm}
\textit{Solution :} Since, we know that $\bar{A}=\langle A \rangle$ thus
\[ \langle H-\{1\} \rangle = \{ a_1^{\epsilon_1} \cdots a_n^{\epsilon_n} \mid a_i \in H-\{1\}, n \in \mathbb{Z}_{\ge 0} ,
 \epsilon_i = \pm 1 \} \]

Thus, for $a \in H$ and $a \neq 1$, $a \in \langle H-\{1\} \rangle$ and also $1=a^{1}a^{-1} \in \langle H - \{1\} \rangle$.
Also, $a \in \langle H - \{1\} \rangle$ is just some combination of elements in $H$ thus $a \in H$. Thus, we have
\[ \langle H - \{1\} \rangle = H \]

\paragraph{4. } Prove that the multiplicative group of positive rational numbers is generated by the set 
\{ $\frac{1}{p} \mid$ $p$ is a prime \}.

\vspace{4mm}
\textit{Solution :} Since,
\[
\left \langle \left \{ \frac{1}{p} \mid p \text{ is a prime} \right \} \right \rangle =  
\left \{ \frac{p_1^{a_1} \cdots p_{n}^{a_n}}{q_1^{b_1} \cdots q_m^{a_m}} \mid p_i, q_i \in \mathbf{Primes} \right \} = \mathbb{Q}_{>0}
\]

\paragraph{5.} A group $H$ is called \textit{finitely generated} if there is a finite set $A$ such that $H = \langle A \rangle$.
\begin{enumerate}
    \item[(a)] Prove that every finite group is finitely generated.
    \item[(b)] Prove that $\mathbb{Z}$ is finitely generated.
    \item[(c)] Prove that every finitely generated subgroup of the additive group $\mathbb{Q}$ is cyclic. [If $H$ is a finitely generated subgroup of $\mathbb{Q}$, show that $H \le \langle \frac{1}{k} \rangle$, where $k$ is the product of all the denominators which appear in a set of generators for $H$.]
    \item[(d)] Prove that $\mathbb{Q}$ is not finitely generated.
\end{enumerate}

\paragraph{6.} Exhibit a proper subgroup of $\mathbb{Q}$ which is not cyclic.

\paragraph{7.} A subgroup $M$ of a group $G$ is called a \textit{maximal subgroup} if $M \neq G$ and the only subgroups of $G$ which contain $M$ are $M$ and $G$.
\begin{enumerate}
    \item[(a)] Prove that if $H$ is a proper subgroup of the finite group $G$ then there is a maximal subgroup of $G$ containing $H$.
    \item[(b)] Show that the subgroup of all rotations in a dihedral group is a maximal subgroup.
    \item[(c)] Show that if $G = \langle x \rangle$ is a cyclic group of order $n \ge 1$ then a subgroup $H$ is maximal if and only if $H = \langle x^p \rangle$ for some prime $p$ dividing $n$.
\end{enumerate}

\paragraph{8.} This is an exercise involving Zorn's Lemma (see Appendix I) to prove that every nontrivial finitely generated group possesses maximal subgroups. Let $G$ be a finitely generated group, say $G = \{g_1, g_2, \dots, g_n\}$, and let $\mathcal{S}$ be the set of all proper subgroups of $G$. Then $\mathcal{S}$ is partially ordered by inclusion. Let $\mathcal{C}$ be a chain in $\mathcal{S}$.
\begin{enumerate}
    \item[(a)] Prove that the union, $H$, of all the subgroups in $\mathcal{C}$ is a subgroup of $G$.
    \item[(b)] Prove that $H$ is a proper subgroup. [If not, each $g_i$ must lie in $H$ and so must lie in some element of the chain $\mathcal{C}$. Use the definition of a chain to arrive at a contradiction.]
    \item[(c)] Use Zorn's Lemma to show that $\mathcal{S}$ has a maximal element (which is, by definition, a maximal subgroup).
\end{enumerate}

\paragraph{9.} Let $p$ be a prime and let 
\[
Z = \{ z \in \mathbb{C} \mid z^{p^m} = 1 \text{ for some } n \in \mathbb{Z}^+ \}
\] 
(so $Z$ is the multiplicative group of all $p$-power roots of unity in $\mathbb{C}$). For each $k \in \mathbb{Z}^+$ let 
\[
H_k = \{ z \in Z \mid z^{p^k} = 1 \} 
\] 
(the group of $p^k$th roots of unity). Prove the following:
\begin{enumerate}
    \item[(a)] $H_k \le H_m$ if and only if $k \le m$.
    \item[(b)] $H_k$ is cyclic for all $k$ (assume that for any $n \in \mathbb{Z}^+$, $\{ e^{2 \pi i t / n} \mid t = 0,1,\dots, n-1\}$ is the set of all $n$th roots of $1$ in $\mathbb{C}$).
    \item[(c)] Every proper subgroup of $Z$ equals $H_k$ for some $k \in \mathbb{Z}^+$ (in particular, every proper subgroup of $Z$ is finite and cyclic).
    \item[(d)] $Z$ is not finitely generated.
\end{enumerate}

\paragraph{10.} A nontrivial abelian group $A$ (written multiplicatively) is called \textit{divisible} if for each element $a \in A$ and each nonzero integer $k$ there is an element $x \in A$ such that $x^k = a$, i.e., each element has a $k$th root in $A$ (in additive notation, each element is the $k$th multiple of some element of $A$).
\begin{enumerate}
    \item[(a)] Prove that the additive group of rational numbers, $\mathbb{Q}$, is divisible.
    \item[(b)] Prove that no finite abelian group is divisible.
\end{enumerate}

\paragraph{11.} Prove that if $A$ and $B$ are nontrivial abelian groups, then $A \times B$ is divisible if and only if both $A$ and $B$ are divisible groups.



\eject
\section{Quotients Groups and Homomorphism}
\subsection{Definition and Examples}

\begin{definition}
    Let $\varphi : G \to H$ be a homomorphism. A \textit{fiber} over $a$, where $a \in \operatorname{im}(\varphi)$, is the set of elements 
    in $G$ that gets mapped to $a$ under $\varphi$ i.e 
    \[ X_a = \{ g \in G \mid \varphi(g)=a \} \]
    It is also denoted by $\varphi^{-1}(a)$.
\end{definition}


\begin{definition}
    We define the product of \textit{fibers} as following
    \[ X_a \cdot X_b = \{g_1 g_2 \mid g_1 \in X_a , g_2 \in X_b\} \]
\end{definition}

\begin{remark}
    By definition of the product of fibers, we can see that
    \[ X_a \cdot X_b = X_{ab} \]
\end{remark}

\begin{proposition}
    The set of \textit{fibers} over the elements of $\operatorname{im}(\varphi)$ forms a group. 
\end{proposition}

\begin{proof}
    The identity element of the set is going to be $X_{1_H}$. The inverse of $X_a$ is going to be $X_{a^{-1}}$. And one can check that
    the associativity the closure property holds.
\end{proof}

\begin{definition}
    If $\varphi$ is a homomorphism $\varphi : G \to H$, the \textit{kernel} of $\varphi$ is the set 
    \[ \{g \in G \mid \varphi(g)=1_H\} \]
    and will be denoted by $\ker \varphi$.
\end{definition}

\begin{remark}
    The kernel of $\varphi$ is the same as the fiber over $1_h$ i.e 
    \[ \ker \varphi = X_{1_H} \] 
\end{remark}

\begin{definition}
    Let $\varphi : G \to H$ be a homomorphism with kernel $K$. The \textit{quotient group} or the \textit{factor group}, $G/K$
    (read as $G$ \textit{mod} $K$), is the group whose elements are the \textit{fibers} of $\varphi$. 
\end{definition}


\begin{proposition}
    Let $\varphi : G \to H$ be a homomorphism of groups with kernel $K$. Let $X \in G/K$ be the fiber above $a$. i.e 
    $X = \varphi^{-1}(a)$
    \begin{enumerate}
        \item For any $u \in X$, $X = \{u k \mid k \in K\} = uK$
        \item For any $u \in X$, $X=\{ku \mid k \in K\} = Ku$
    \end{enumerate}
\end{proposition}

\begin{proof}
    We'll prove $2.$ and leave $1.$ for the future me. Suppose $k \in K$ then 
    \begin{align*}
        \varphi(ku) &= \varphi(k) \varphi(u) \\
        &= 1 \cdot \varphi(u) \\
        &= a
    \end{align*}
    Thus, $ku \in X \implies Ku \subseteq X$. Now, to show $X \subseteq Ku$, take any $g \in X$ then define $k := gu^{-1}$ thus
    \[ \varphi(k)=\varphi(gu^{-1})=a a^{-1}=1 \]
    \[ \implies k \in K \]
    Thus, $g=ku \in Ku \implies X \subseteq Ku$. 
\end{proof}


\begin{definition}
    For a $N \le G$ and any $g \in G$ let 
    \[ gN = \{ gn \mid n \in N\} \quad \text{and} \quad Ng = \{ ng \mid n \in G\} \]
    be the \textit{left coset} and \textit{right coset} of $N$ in $G$. Any element of a coset is called a \textit{representative} of the 
    coset.
\end{definition}




\end{document}