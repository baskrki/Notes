\subsection{The Isomorphism Theorems}

\begin{theorem}(First Isomorphism Theorem)
    If $\varphi : G \to H$ is a group homomorphism, then $\ker \varphi \normal G$ and $G/\ker \varphi \cong 
    \varphi(G)$. 
\end{theorem}

\begin{proof}
    we have already shown the first part of the theorem, which is $\ker \varphi \normal G$. For the second part, define a map
    \begin{align*}
        \Phi : G/\ker \varphi \to \varphi(G) \\ 
        g \ker \varphi \mapsto \varphi(g) 
    \end{align*}
    This map is well-defined as 
    \begin{align*}
        g \ker \varphi &= h \ker \varphi \\
        \implies gh^{-1} &\in \ker \varphi \\
        \implies \varphi(g) &=\varphi(h) \\
        \implies \Phi(g \ker \varphi) &= \Phi(h \ker \varphi)
    \end{align*}

    This map is also a homomorphism as
    \begin{align*}
        \Phi((g \ker \varphi) \cdot (h \ker \varphi)) &= \Phi((gh)\ker \varphi)  \\
        &= \varphi(gh) \\
        &= \varphi(g) \varphi(h) \\
        &= \Phi(g \ker \varphi) \Phi(h \ker \varphi)
    \end{align*}
    This map is trivially surjective and for the injective part, suppose 
    \[ \Phi(g \ker \varphi) = \Phi(h \ker \varphi) \]
    \[ \implies \varphi(g) = \varphi(h) \]
    \[ \implies \varphi(gh^{-1})=e \]
    \[ \implies gh^{-1} \in \ker \varphi \]
    Thus, $g \in (\ker \varphi)h  = h (\ker \varphi)$, which implies $g \ker \varphi = h \ker \varphi$.
\end{proof}

\begin{corollary}
    Let $\varphi : G \to H$ be a group homomorphism. Then,
    \begin{enumerate}
        \item $\varphi$ is injective $\iff$ $\ker \varphi = \{e\}$.
        \item $|G : \ker \varphi| = |\varphi(G)|$.
    \end{enumerate}
\end{corollary}

\begin{theorem}(Second Isomorphism Theorem)
    Let $G$ be a group and let $A$ and $B$ be subgroups of $G$ and assume $A \le N_G(B)$. Then $AB$ is a subgroup, $B \normal AB$,
    $A \cap B \normal A$ and $AB/B \cong  A/A \cap B$.
\end{theorem}

\begin{proof}
    Since $A \le N_G(B)$, every element of $A$ normalizes $B$; that is,
\[
aBa^{-1} = B \qquad \text{for all } a \in A.
\]

Also, $B \le N_G(B)$ trivially, because for any $b \in B$,
\[
bBb^{-1} = B.
\]

Now consider an arbitrary element of $AB$. Every element of $AB$ can be written as $ab$ for some $a \in A$ and $b \in B$.
 We compute the conjugate of $B$ by such an element:
\[
(ab)B(ab)^{-1}
    = a \bigl( b B b^{-1} \bigr) a^{-1}
    = a B a^{-1}
    = B,
\]
since $bBb^{-1} = B$ and $aBa^{-1} = B$.

Thus every element of $AB$ normalizes $B$, so
\[
AB \le N_G(B).
\]

Since $kBk^{-1}=B$ for any $k \in AB$ , it follows that
\[
B \normal AB.
\]

Since, $B$ is normal in $AB$ the quotient group $AB/B$ is well defined. Define the map $\varphi : A \to AB/B$ by 
\[\varphi(a)=aB\]
One can check that this map is a homomorphism. Also, it is clear that this map is surjective from the definition. The identity in $AB/B$ is 
$1B$ so 
\[\ker \varphi = \{a \in A \mid aB=1B\}= \{ a \in A \mid a \in B\} = A \cap B\]
By the first isomorphism theorem, $A \cap B \normal A$ and $A / A \cap B \cong AB/B$.
\end{proof}
