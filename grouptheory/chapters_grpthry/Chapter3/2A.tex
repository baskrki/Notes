\subsection{Definition and Examples}

\begin{definition}
    Let $\varphi : G \to H$ be a homomorphism. A \textit{fiber} over $a$, where $a \in \operatorname{im}(\varphi)$, is the set of elements 
    in $G$ that gets mapped to $a$ under $\varphi$ i.e 
    \[ X_a = \{ g \in G \mid \varphi(g)=a \} \]
    It is also denoted by $\varphi^{-1}(a)$.
\end{definition}


\begin{definition}
    We define the product of \textit{fibers} as following
    \[ X_a \cdot X_b = \{g_1 g_2 \mid g_1 \in X_a , g_2 \in X_b\} \]
\end{definition}

\begin{remark}
    By definition of the product of fibers, we can see that
    \[ X_a \cdot X_b = X_{ab} \]
\end{remark}

\begin{proposition}
    The set of \textit{fibers} over the elements of $\operatorname{im}(\varphi)$ forms a group. 
\end{proposition}

\begin{proof}
    The identity element of the set is going to be $X_{1_H}$. The inverse of $X_a$ is going to be $X_{a^{-1}}$. And one can check that
    the associativity the closure property holds.
\end{proof}

\begin{definition}
    If $\varphi$ is a homomorphism $\varphi : G \to H$, the \textit{kernel} of $\varphi$ is the set 
    \[ \{g \in G \mid \varphi(g)=1_H\} \]
    and will be denoted by $\ker \varphi$.
\end{definition}

\begin{remark}
    The kernel of $\varphi$ is the same as the fiber over $1_h$ i.e 
    \[ \ker \varphi = X_{1_H} \] 
\end{remark}

\begin{definition}
    Let $\varphi : G \to H$ be a homomorphism with kernel $K$. The \textit{quotient group} or the \textit{factor group}, $G/K$
    (read as $G$ \textit{mod} $K$), is the group whose elements are the \textit{fibers} of $\varphi$. 
\end{definition}


\begin{proposition}
    Let $\varphi : G \to H$ be a homomorphism of groups with kernel $K$. Let $X \in G/K$ be the fiber above $a$. i.e 
    $X = \varphi^{-1}(a)$
    \begin{enumerate}
        \item For any $u \in X$, $X = \{u k \mid k \in K\} = uK$
        \item For any $u \in X$, $X=\{ku \mid k \in K\} = Ku$
    \end{enumerate}
\end{proposition}

\begin{proof}
    We'll prove $2.$ and leave $1.$ for the future me. Suppose $k \in K$ then 
    \begin{align*}
        \varphi(ku) &= \varphi(k) \varphi(u) \\
        &= 1 \cdot \varphi(u) \\
        &= a
    \end{align*}
    Thus, $ku \in X \implies Ku \subseteq X$. Now, to show $X \subseteq Ku$, take any $g \in X$ then define $k := gu^{-1}$ thus
    \[ \varphi(k)=\varphi(gu^{-1})=a a^{-1}=1 \]
    \[ \implies k \in K \]
    Thus, $g=ku \in Ku \implies X \subseteq Ku$. 
\end{proof}

\begin{remark}
    Any coset(check below for the definition) is also an \textit{fiber} for some element. It is because
    \[ uK = \varphi^{-1}(\varphi(u)) = uK\] 
\end{remark}


\begin{definition}
    For a $N \le G$ and any $g \in G$ let 
    \[ gN = \{ gn \mid n \in N\} \quad \text{and} \quad Ng = \{ ng \mid n \in G\} \]
    be the \textit{left coset} and \textit{right coset} of $N$ in $G$. Any element of a coset is called a \textit{representative} of the 
    coset.
\end{definition}

\begin{remark}
    To verify a map is a well defined map, one can't just use the condition imposed on the map. For example,
    the proof of the theorem below, I have said the operation is indeed well-defined but didn't prove it. You can't go about doing the 
    following.

    Let $uK=u'K$ and $vK=v'K$ thus
    \begin{align*}
        (uv)K &= (uK)(vK) \\
        &= (u'K)(v'K)\\
        &= (u'v')K
    \end{align*}

    Here, you're assuming that $a=b \implies f(a)=f(b)$ which is true if $f$ were to be a function. But to be a function, it needs to be 
    well defined. Therefore you're assuming it's well defined to begin with.
\end{remark}

\begin{theorem}
    Let $G$ be a group and let $K$ be the kernel of some homomorphism from $G$ to another group. Then the set whose elements are left
    cosets of $K$ in $G$ with operation defined by
    \[ (uK) \circ (vK) = (uv)K \]
    forms a group, $G/K$.
\end{theorem}

\begin{proof}
One can check that the set whose elements are left cosets of $K$ in $G$ with operation defined above, does indeed form a group. 
Note that the operation is also well defined.
Now, if $X$ and $Y$ are fibers then $Z=XY$ is also a fiber. Now, we can write each fiber as 
\[ X=uK, \quad Y=vK, \quad XY = jK \]
But we set $j=uv$ as $uv \in XY$. Thus, every element of $G/K$ is in set $\{uK \mid u \in G\}$. But we also that 
every coset is also a fiber thus, every element of $\{uK \mid u \in G\}$ is in $G/K$.
\end{proof}
