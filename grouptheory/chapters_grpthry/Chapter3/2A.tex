\subsection{Definition and Examples}

\begin{definition}
    Let $\varphi : G \to H$ be a homomorphism. A \textit{fiber} over $a$, where $a \in \operatorname{im}(\varphi)$, is the set of elements 
    in $G$ that gets mapped to $a$ under $\varphi$ i.e 
    \[ X_a = \{ g \in G \mid \varphi(g)=a \} \]
    It is also denoted by $\varphi^{-1}(a)$.
\end{definition}


\begin{definition}
    We define the product of \textit{fibers} as following
    \[ X_a \cdot X_b = \{g_1 g_2 \mid g_1 \in X_a , g_2 \in X_b\} \]
\end{definition}

\begin{remark}
    By definition of the product of fibers, we can see that
    \[ X_a \cdot X_b = X_{ab} \]
\end{remark}

\begin{proposition}
    The set of \textit{fibers} over the elements of $\operatorname{im}(\varphi)$ forms a group. 
\end{proposition}

\begin{proof}
    The identity element of the set is going to be $X_{1_H}$. The inverse of $X_a$ is going to be $X_{a^{-1}}$. And one can check that
    the associativity the closure property holds.
\end{proof}

\begin{definition}
    If $\varphi$ is a homomorphism $\varphi : G \to H$, the \textit{kernel} of $\varphi$ is the set 
    \[ \{g \in G \mid \varphi(g)=1_H\} \]
    and will be denoted by $\ker \varphi$.
\end{definition}

\begin{remark}
    The kernel of $\varphi$ is the same as the fiber over $1_h$ i.e 
    \[ \ker \varphi = X_{1_H} \] 
\end{remark}

\begin{definition}
    Let $\varphi : G \to H$ be a homomorphism with kernel $K$. The \textit{quotient group} or the \textit{factor group}, $G/K$
    (read as $G$ \textit{mod} $K$), is the group whose elements are the \textit{fibers} of $\varphi$. 
\end{definition}


\begin{proposition}
    Let $\varphi : G \to H$ be a homomorphism of groups with kernel $K$. Let $X \in G/K$ be the fiber above $a$. i.e 
    $X = \varphi^{-1}(a)$
    \begin{enumerate}
        \item For any $u \in X$, $X = \{u k \mid k \in K\} = uK$
        \item For any $u \in X$, $X=\{ku \mid k \in K\} = Ku$
    \end{enumerate}
\end{proposition}

\begin{proof}
    We'll prove $2.$ and leave $1.$ for the future me. Suppose $k \in K$ then 
    \begin{align*}
        \varphi(ku) &= \varphi(k) \varphi(u) \\
        &= 1 \cdot \varphi(u) \\
        &= a
    \end{align*}
    Thus, $ku \in X \implies Ku \subseteq X$. Now, to show $X \subseteq Ku$, take any $g \in X$ then define $k := gu^{-1}$ thus
    \[ \varphi(k)=\varphi(gu^{-1})=a a^{-1}=1 \]
    \[ \implies k \in K \]
    Thus, $g=ku \in Ku \implies X \subseteq Ku$. 
\end{proof}

\begin{remark}
    Any coset(check below for the definition) is also an \textit{fiber} for some element. It is because
    \[ uK = \varphi^{-1}(\varphi(u)) = uK\] 
\end{remark}


\begin{definition}
    For a $N \le G$ and any $g \in G$ let 
    \[ gN = \{ gn \mid n \in N\} \quad \text{and} \quad Ng = \{ ng \mid n \in G\} \]
    be the \textit{left coset} and \textit{right coset} of $N$ in $G$. Any element of a coset is called a \textit{representative} of the 
    coset.
\end{definition}

\begin{remark}
    To verify a map is a well defined map, one can't just use the condition imposed on the map. For example,
    the proof of the theorem below, I have said the operation is indeed well-defined but didn't prove it. You can't go about doing the 
    following.

    Let $uK=u'K$ and $vK=v'K$ thus
    \begin{align*}
        (uv)K &= (uK)(vK) \\
        &= (u'K)(v'K)\\
        &= (u'v')K
    \end{align*}

    Here, you're assuming that $a=b \implies f(a)=f(b)$ which is true if $f$ were to be a function. But to be a function, it needs to be 
    well defined. Therefore you're assuming it's well defined to begin with.
\end{remark}

\begin{theorem}
    Let $G$ be a group and let $K$ be the kernel of some homomorphism from $G$ to another group. Then the set whose elements are left
    cosets of $K$ in $G$ with operation defined by
    \[ (uK) \circ (vK) = (uv)K \]
    forms a group, $G/K$.
\end{theorem}

\begin{proof}
One can check that the set whose elements are left cosets of $K$ in $G$ with operation defined above, does indeed form a group. 
Note that the operation is also well defined.
Now, if $X$ and $Y$ are fibers then $Z=XY$ is also a fiber. Now, we can write each fiber as 
\[ X=uK, \quad Y=vK, \quad XY = jK \]
But we set $j=uv$ as $uv \in XY$. Thus, every element of $G/K$ is in set $\{uK \mid u \in G\}$. But we also that 
every coset is also a fiber thus, every element of $\{uK \mid u \in G\}$ is in $G/K$.
\end{proof}

\begin{proposition}
    Let $N$ be any subgroup of the group $G$. The set of left cosets of $N$ in $G$ form a partition of $G$. Furthermore, for all
    $u,v \in G$, $uN=vN \iff v^{-1}u \in N$ and in particular, $uN=vN$ if and only if $u$ and $v$ are representative of the same coset. 
\end{proposition}

\begin{proof}
    Since, $g \in gN$ as $1 \in N$, we can say that
    \[ g = \bigcup_{g \in G} gN \]
    Now, if $x \in uN \cap vN$ then 
    \[ x = un = vm \]
    Thus, $u=vmn^{-1} \implies ut = vmn^{-1}t \in vN$. Thus, $uN \subseteq vN$ as $ut$ covers every element of $uN$. 
    
    Now, one can reverse the roles and prove $vN \subseteq uN$, which altogether implies $uN=vN$. Thus, if $uN \cap vN \neq \emptyset$ then
    $uN=vN$.

    For the other part of the proposition, $uN = vN \iff u = vn \iff v^{-1}u = n \in N$.

    If $uN = vN = K$ then $u,v \in K$. Thus they are the representative of the same coset. Also, if
    $u \in tN$ and $v \in tN$ then $uN=tN=vN$.
\end{proof}

\begin{proposition}
    Let $G$ be a group and let $N$ be a subgroup of $G$.
    \begin{enumerate}
        \item The operation on the set of left cosets of $N$ on $G$ defined by 
        \[ (uN) \cdot (vN) = (uv) N \]
        is well defined if and only if $gng^{-1} \in N$ for all $g \in G$ and for all $n \in N$.
        \item If the above operation is well-defined then the it makes the set of left coset into a group.
    \end{enumerate} 
\end{proposition}

\begin{proof}
    Assume the operation is well-defined i.e $u,u_1 \in uN$, $v,v_1 \in vN \implies uv N = u_1v_1 N$.
    Let $g$ be an arbitrary element of $G$ and let $n$ be an arbitrary element of $N$. Then, set $u=1,u_1=n$ and $v_1=v=g^{-1}$ thus
    \[ 1 g^{-1}N =  n g^{-1} N \implies g^{-1} N = ng^{-1} N   \]
    Thus, $ng^{-1} \in g^{-1}N \implies gng^{-1} = k \in N$.
    
    Now, suppose $gng^{-1} \in N$ for all $g \in G$ and $n \in N$. Let $u,u_1 \in uN$ and $v,v_1 \in vN$. We need to show
    \[ (uv)N=(u_1 v_1)N \]
    Since, $u_1 \in uN$ and $v_1 \in vN$ we can write them as $u_1 = un_1$ and $v_1 = vm$ for some $n,m \in N$.
    Now, if we can prove $u_1 v_1 \in (uv)N$ then we'd be done.
    \begin{align*}
        u_1 v_1 &= (un)(vm) \\
        &= u(vv^{-1})nvm \\
        &= (uv)(v^{-1}nv)m = (un)(n_1m) \\
    \end{align*}
    where $n_1 = v^{-1}nv = v^{-1}n(v^{-1})^{-1} \in N$ as per the assumption. Thus, $u_1 v_1 \in uv N \implies (u_1v_1)N = (uv)N$.

    For the second part, just check the group axioms.
\end{proof}

\begin{definition}
    The element $gng^{-1}$ is called the \textit{conjugate} of $n$ by $g$. The set $gNg^{-1} = \{gng^{-1} \mid n \in N\}$ is called 
    \textit{conjugate} of $N$ by $g$. The element $g$ is said to \textit{normalize} $N$ if $gNg^{-1}=N$. A subgroup $N$ is called 
    normal if every element of $G$ \textit{normalizes} $N$ i.e $gNg^{-1}=N$ for all $g \in G$. If $N$ is a normal subgroup of $G$ then 
    we write it as $N \normal G$.
\end{definition}

\begin{proposition}
    Let $N$ be the subgroup of $G$. Then the following are equivalent 
    \begin{enumerate}
        \item $N \normal G$
        \item $N_G(N)=G$
        \item $gN=Ng$ for all $g \in G$
        \item $gNg^{-1} \subseteq N$
    \end{enumerate}
\end{proposition}

\begin{proof}
    Most of them easily follow from the definition and previous propositions.
\end{proof}

\begin{definition}
    We define $G/N=\{gN \mid g \in G\}$ for $N \le G$.
\end{definition}

\begin{proposition}
    Let $N \le G$. Then $N$ is normal if and only if $N$ is a kernel of some homomorphism.
\end{proposition}

\begin{proof}
    Suppose $N$ is a kernel of some homomorphism $\varphi$. Then $gN = Ng$ for all $g \in G$ and by previous proposition we can say 
    $N \normal G$. Now, suppose $N \normal G$ then we define a map $\psi : G \to G/N$ such that $g \mapsto gN$.
    Then,
    \begin{align*}
        \psi(g_1 g_2) &= (g_1 g_2)N \\
        &= g_1 N g_2 N \\
        &= \psi(g_1) \psi(g_2)
    \end{align*}
    Thus, $\psi$ is indeed a homomorphism. Now,
    \begin{align*}
        \ker \psi &= \{ g \mid \psi(g) = 1 N\} \\
        &= \{ g \mid gN = N \} \\
        &= \{ g \mid g \in N\} \\
        &= N
    \end{align*}
    Thus, $N$ is the kernel of $\psi$.
\end{proof}

\begin{definition}
    Let $N \normal G$. The homomorphism $\psi : G \to G/N$ defined by $\psi(g)=gN$ is called the \textit{natural projection} of $G$ onto
    $G/N$. If $\bar{H} \le G/N$ is a subgroup of $G/N$, the \textit{complete preimage} of $\bar{H}$ in $G$ is the preimage of $\bar{H}$
    under the natural projection.
\end{definition}

\eject

\subsection*{Problems and Solutions}

\paragraph{Problem :} Let $\varphi : G \to H$ be an homomorphism and Let $E$ be a subgroup of $H$. Prove that $\varphi^{-1}(E) \le G$, where
$\varphi^{-1}(E) = \{x \mid \varphi(x) \in E\}$. If $E \normal H$, prove that $\varphi^{-1}(E) \normal G$. Deduce that $\ker \varphi 
\normal G$. 

\vspace{4mm}
\textit{Solution :} Since, $\varphi^{-1}(E) = \{x \mid \varphi(x) \in E\}$. This subset of $G$ is clearly not empty and
if $x,y \in \varphi^{-1}(E)$ then $\varphi(xy^{-1})=\varphi(x) \varphi(y)^{-1} \in E$ as $E$ is a subgroup and 
$\varphi(x), \varphi(y) \in E$. Thus, $xy^{-1} \in \varphi^{-1}(E)$ for all $x,y \in \varphi^{-1}(E)$ which implies that 
$\varphi^{-1}(E) \le G$.

If $E \normal H$ then take any arbitrary element $g$ of $G$ and take any arbitrary element of $n$ of $\varphi^{-1}(E)$.
Thus, $\varphi(gng^{-1}) = \varphi(g)\varphi(n) \varphi(g)^{-1} \in E$ as $\varphi(n) \in E$ and $\varphi(g) \in H$ and $E$ is normal.
Thus, $gng^{-1} \in \varphi^{-1}(E)$ which implies $g\varphi^{-1}(E)g \subseteq \varphi^{-1}(E) \implies \varphi^{-1}(E) \normal G$.
For kernel part, take $E = \{1\} \normal H$

\paragraph{Problem :} Let $\varphi : G \to H$ be a homomorphism of groups with kernel $K$ and let $a,b \in \varphi(G)$. Let $X \in G/K$ be
the the fiber above $a$ and let $Y$ be the fiber above $b$, i.e, $X = \varphi^{-1}(a)$ and $Y = \varphi^{-1}(b)$. Fix an element of $X$.
Prove that if $XY=Z$ in the quotient group $G/K$ and $w$ is any member of $Z$, the there is some $v \in Y$ such that $uv=w$.

\vspace{4mm}
\textit{Solution :} First part follows immediately from \textbf{Definition 1.2.} and for the second part look at $\varphi(u^{-1}w)$ where 
$w$ is an arbitrary member of $Z$. Thus,
\begin{align*}
    \varphi(u^{-1}w) &= a^{-1} (ab) \\
    &= b 
\end{align*}
Thus, $u^{-1}w \in Y \implies w = uv$ for some $v \in Y$.

\paragraph{Problem :} Let $A$ be an abelian group and let $B$ be an subgroup of $A$. Prove that $A/B$ is abelian.
Give an example of a non-abelian of a non-abelian group $G$ containing a proper normal subgroup $N$ such that $G/N$ is abelian. 

\vspace{4mm}
\textit{Solution :} Any subgroup of a abelian group is a normal subgroup. Notice that, 
\[ (uB) \circ (vB) = (uv)B = (vu)B = (uB) \circ (vB) \]

For the example part, take $G= S_3$ and $N = \{e,(123),(132)\}$.

\paragraph{Problem :} Prove that in quotient group $G/N$, $(gN)^{\alpha} = g^{\alpha} N$ for all $\alpha \in \mathbb{Z}$.

\vspace{4mm}
\textit{Solution :} If we let $(gN)^{\alpha} := \underbrace{(gN) \cdot (gN) \cdots (gN)}_{\alpha}$ for $\alpha \ge 0$ then 
\begin{align*}
    (gN) \cdot (gN) \cdots (gN) &= \{(gN) \cdot (gN) \} \cdots (gN) \\
    &= (g^2N) \cdot (gN) \cdots (gN)  \quad \text{(\textbf{Proposition 2.4.})} \\
    &= g^\alpha N
\end{align*}

\paragraph{Problem :} Prove that the order of the element $gN$ in $G/N$ is $n$, where $n$ is the smallest positive integer such that 
$g^n \in N$(and $gN$ has infinite order if no such $n$ exists).Give an example to show that the order of $gN$ in $G/N$ may be strictly 
smaller than the order of $g$ in $G$.

\vspace{4mm}
\textit{Solution :} Let $n$ be the smallest positive integer such that $g^n \in N$. Then, $g^nN=N$.
Now, if there exists a $s < n$ s.t $g^s \in N$, then it would contradict our assumption. Thus order of $gN$ is $n$.  
Now, Suppose $g^n \not \in N$ for any $n > 0$ then $g^nN \neq N$ for any $n > 0$. Thus the order is infinite if no such $n$ exists.

An example of $g^sN = N$ and $s < |g|$ is $N=\{1,r,r^2\} \normal D_6$, $g=r$.

\paragraph{Problem :} Define $\varphi : \mathbb{R}^{\times} \to \{\pm 1\}$ by letting $\varphi(x)$ be $x$ divided by the absolute value of $x$.
Describe the fibers of $\varphi$ and prove that $\varphi$ is a homomorphism.

\vspace{4mm}
\textit{Solution :} The fiber over $-1$ is $X_{-1} = \{ x \mid \varphi(x) = -1\}$. Since $\varphi(x)= \frac{x}{|x|}=-1$, the only numbers
that get mapped to $-1$ are $x < 0$. Similarly, the only number that get mapped to $1$ are $x > 0$. Thus the fiber over $1$ and $-1$ are the 
positive and negative reals.

Now, $\varphi(x \cdot y) = \frac{xy}{|xy|} =\frac{x}{|x|} \frac{y}{|y|} = \varphi(x) \varphi(y)$.

\paragraph{Problem :} Define $\pi : \mathbb{R}^{2} \to \mathbb{R}$ by $\pi(x,y) = x+y$. Prove that $\pi$ is a surjective homomorphism and
describe the kernel and fibers of $\pi$ geometrically.

\vspace{4mm}
\textit{Solution :} To prove its a homomorphism, take 
\[ \pi((x,y)+(a,b))=\pi(x+a,y+b)=x+y+a+b = \pi(x,y) + \pi(a,b) \]
To prove its surjectivity , notice that for a real number $x$ there are always two real numbers that add up to $x$.

The $\ker \pi$ is the set of solution to the equation $x+y=0$ and the fiber over $a$ of $\varphi$ is the set of solution to the equation
$x+y=a$.

\paragraph{Problem :} Let $\varphi : \mathbb{R}^{\times} \to \mathbb{R}^{\times}$ be a map sending $x$ to $|x|$. Prove it is a homomorphism and 
find the image of $\varphi$. Describe the kernel and the fibers of $\varphi$.

\vspace{4mm}
\textit{Solution :} To show that it is a homomorphism, 
\[ \varphi(x \cdot y) = |x \cdot y| = |x| \cdot |y| = \varphi(x) \varphi(y) \]

The image of $\varphi$ would be the positive reals. The $\ker \varphi = \{\pm 1\}$ and $X_a = \{\pm a\}$.

\paragraph{Problem :} Define $\varphi : \mathbb{C}^{\times} \to \mathbb{R}^{\times}$ by $\varphi(a+ib)=a^2 +b^2$. Prove that the map 
is a homomorphism and find its image. Describe the kernel and the fibers of $\varphi$ geometrically.

\vspace{4mm}
\textit{Solution :} To prove its homomorphism,
\begin{align*}
    \varphi((a+bi)\cdot (c+di)) &= \varphi(ac-bd+(ad+bc)i) \\
    &= (ac-bd)^2 + (ad+bc)^2 \\
    &= (a^2+b^2)(c^2+d^2)
\end{align*}

The image of $\varphi$ is $\mathbb{R}_{>0}$ and the kernel is a set of solution to $x^2+y^2=1$ which is a circle with radius $1$.

The fiber of $X_a$ is also the set of solution to the equation of circle with radius $\sqrt{a}$.