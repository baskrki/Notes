\subsection{Composition Series and H\"older Program}

\begin{proposition}
    If $G$ is a finite abelian group and $p$ is a prime dividing $|G|$, then $G$ contains and element of order $p$.
\end{proposition}

\begin{proof}
    We'll use induction to prove the result.

    \vspace{4mm}
    \textbf{Base case:} $|G|=p$ 
    If $|G| = p$ then $x$ has order $p$ by Lagrange's Theorem and we are done. 
    
    \vspace{4mm}
    \textbf{Inductive Hypothesis:} Suppose for every abelian group $H$ such that $p \mid |H|$ and $|H| < |G|$, there exists a element with 
    order $p$. 
    
    \vspace{4mm}
    Suppose $x \in G$ with $x \neq 1$ and $p$ divides $|x|$ and write $|x| = pn $. By \textbf{Proposition 1.11}($2$), 
    $ |x^n| = p $, and again we have an element of 
    order $ p $. We may therefore assume $ p $ does not divide $ |x| $.
    Let $ N = \langle x \rangle $. Since $ G $ is abelian, $ N \trianglelefteq G $. By Lagrange's Theorem, 
    $ |G/N| = \frac{|G|}{|N|} $ and since $ N \neq 1 $, $ |G/N| < |G| $. Since $ p $ does not divide $ |N| $, we must have 
    $ p \mid |G/N| $. We can now apply the induction assumption to the smaller group $ G/N $ to conclude it contains an element, 
    $ \bar{y} = yN $, of order $ p $. Thus, we get $ y \notin N $ ($ \bar{y} \neq \bar{1} $) but $ y^p \in N $. Now, suppose $|y^p|=|y|$
    then $(|y|,p)=1$ but $|y^p|=|y|$ we have $\langle y^p \rangle=\langle y \rangle$, thus $y=y^{pz} \in N$ 
    a contradiction.
    
    Thus, $|y^p| \neq |y|$ and from \textbf{Proposition 1.11.}($2$) we have $|y^p| = \frac{|y|}{(|y|,m)}$ which implies $(|y|,p) > 1$. 
    Thus, $p \mid |y|$.
     We are now in the situation described in the preceding paragraph, so that argument again produces an element of order $ p $. 
     The induction is complete.
\end{proof}

\begin{definition}
    A (finite or infinite) group $G$ is called \textit{simple} if $|G|>1$ and the only normal subgroups of $G$ are $1$ and $G$.
\end{definition}

\begin{definition}
    In a group $G$ a sequence of subgroups 
    \[ 1 = N_0 \le N_1 \le N_2 \le \cdots \le N_{k-1} \le N_k = G \]
    is called \textit{composition series} if $N_i \normal N_{i+1}$ and $N_{i+1}/N_{i}$ is a simple group for $0 \le i \le k-1$.
    If the above series is a composition series then the quotient groups $N_{i+1}/N_{i}$ are called \textit{composition factor}.  
\end{definition}

\begin{remark}
    Normality is not \textbf{transitive}.
\end{remark}

