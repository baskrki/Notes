\subsection{More on Cosets and Lagrange's Theorem}

\begin{theorem}[Lagrange's Theorem]
    If $G$ is a finite subgroup and $H$ be its subgroup then order of $H$ divides order of $G$ and the number of left cosets of $H$ in $G$
    is exactly $\frac{|G|}{|H|}$.  
\end{theorem}

\begin{proof}
    Let $|H|=n$ and let $k$ be the number of left cosets of $H$ in $G$. By definition of a left coset the map
    \[ \Psi : H \to gH \quad \text{defined by} \quad h \to gh \] 
    is a surjection from $H$ to left coset  $gH$. Also, if $\Psi(h_1)=\Psi(h_2)$ then $gh_1 = gh_2 \implies h_1=h_2$. Thus,
    the map $\Psi$ is a bijection and 
    \[ |H|=|gH| \]
    Since, $G$ is partitioned into $k$ disjoint cosets of $H$ which have the same order as $H$, we have 
    \[ |G|=nk \implies \frac{|G|}{|H|}=k \] 
\end{proof}

\begin{remark}
    We could add a similar proof for $gH$ and conclude number of left cosets = number of right cosets for any finite group $G$.
\end{remark}

\begin{definition}
    If $G$ is a group (possibly infinite) and $H \le G$, then the number of left cosets of $H$ in $G$ is called the \textit{index} of $H$
    in $G$ and is denoted by $|G:H|$.
\end{definition}

\begin{proposition}
    If $G$ is a finite group and $x \in G$, then order of $x$ divides the order of $G$. In particular $x^{|G|}=1$ for all $x \in G$.
\end{proposition}

\begin{proof}
    Apply lagrange's theorem on $H=\langle x \rangle$.
\end{proof}

\begin{proposition}
    If $G$ is a group of prime order $p$, then $G$ is cyclic, hence $G \cong \mathbf{Z}_{p}$.
\end{proposition}

\begin{proof}
    Take any $x \in G$ such that $x \neq 1$ then $|\langle x \rangle|$ divides $p$ which implies  $\langle x \rangle = G$. 
\end{proof}

\begin{proposition}
    Let $G$ be a group and $H$ be a subgroup of $G$ with $|G:H|=2$. Then $H \normal G$. 
\end{proposition}

\begin{proof}
Let \( g \in G \) be arbitrary. If \( g \in H \), then clearly
\[
gH = H = Hg.
\]
If \( g \notin H \), then since there are exactly two left cosets of \( H \) in \( G \)
and \( g \notin H \), these must be \( H \) and \( gH \).
Because left cosets are disjoint, we have
\[
gH = G \setminus H.
\]
Similarly, the right cosets of \( H \) in \( G \) are also disjoint, 
so the two right cosets must be \( H \) and \( Hg \),
and therefore
\[
Hg = G \setminus H.
\]
Thus,
\[
Hg = G \setminus H = gH.
\]
Hence \( gH = Hg \) for all \( g \in G \),
and therefore \( H \trianglelefteq G \).
\end{proof}


\begin{theorem}[Cauchy's Theorem]
    If $G$ is finite group and $p$ is a prime dividing $|G|$ then $G$ has an element of order $p$. 
\end{theorem}

\begin{proof}
    Next Chapter 
\end{proof}

\begin{theorem}[Sylow's Theorem]
    If $G$ is a finite group and $|G|=p^{\alpha}m$, where $p \nmid m$ then $G$ has a subgroup of order $p^{\alpha}$.
\end{theorem}

\begin{proof}
    Next Chapter
\end{proof}

\begin{definition}
    Let $G$ be a group and $H,K \le G$ and define
    \[ HK = \{hk \mid h \in H, k \in K\} \]
\end{definition}