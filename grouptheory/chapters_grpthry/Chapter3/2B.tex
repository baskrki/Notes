\subsection{More on Cosets and Lagrange's Theorem}

\begin{theorem}[Lagrange's Theorem]
    If $G$ is a finite subgroup and $H$ be its subgroup then order of $H$ divides order of $G$ and the number of left cosets of $H$ in $G$
    is exactly $\frac{|G|}{|H|}$.  
\end{theorem}

\begin{proof}
    Let $|H|=n$ and let $k$ be the number of left cosets of $H$ in $G$. By definition of a left coset the map
    \[ \Psi : H \to gH \quad \text{defined by} \quad h \to gh \] 
    is a surjection from $H$ to left coset  $gH$. Also, if $\Psi(h_1)=\Psi(h_2)$ then $gh_1 = gh_2 \implies h_1=h_2$. Thus,
    the map $\Psi$ is a bijection and 
    \[ |H|=|gH| \]
    Since, $G$ is partitioned into $k$ disjoint cosets of $H$ which have the same order as $H$, we have 
    \[ |G|=nk \implies \frac{|G|}{|H|}=k \] 
\end{proof}

\begin{remark}
    We could add a similar proof for $gH$ and conclude number of left cosets = number of right cosets for any finite group $G$.
\end{remark}

\begin{definition}
    If $G$ is a group (possibly infinite) and $H \le G$, then the number of left cosets of $H$ in $G$ is called the \textit{index} of $H$
    in $G$ and is denoted by $|G:H|$.
\end{definition}

\begin{proposition}
    If $G$ is a finite group and $x \in G$, then order of $x$ divides the order of $G$. In particular $x^{|G|}=1$ for all $x \in G$.
\end{proposition}

\begin{proof}
    Apply lagrange's theorem on $H=\langle x \rangle$.
\end{proof}

\begin{proposition}
    If $G$ is a group of prime order $p$, then $G$ is cyclic, hence $G \cong \mathbf{Z}_{p}$.
\end{proposition}

\begin{proof}
    Take any $x \in G$ such that $x \neq 1$ then $|\langle x \rangle|$ divides $p$ which implies  $\langle x \rangle = G$. 
\end{proof}

\begin{proposition}
    Let $G$ be a group and $H$ be a subgroup of $G$ with $|G:H|=2$. Then $H \normal G$. 
\end{proposition}

\begin{proof}
Let \( g \in G \) be arbitrary. If \( g \in H \), then clearly
\[
gH = H = Hg.
\]
If \( g \notin H \), then since there are exactly two left cosets of \( H \) in \( G \)
and \( g \notin H \), these must be \( H \) and \( gH \).
Because left cosets are disjoint, we have
\[
gH = G \setminus H.
\]
Similarly, the right cosets of \( H \) in \( G \) are also disjoint, 
so the two right cosets must be \( H \) and \( Hg \),
and therefore
\[
Hg = G \setminus H.
\]
Thus,
\[
Hg = G \setminus H = gH.
\]
Hence \( gH = Hg \) for all \( g \in G \),
and therefore \( H \trianglelefteq G \).
\end{proof}


\begin{theorem}[Cauchy's Theorem]
    If $G$ is finite group and $p$ is a prime dividing $|G|$ then $G$ has an element of order $p$. 
\end{theorem}

\begin{proof}
    Next Chapter 
\end{proof}

\begin{theorem}[Sylow's Theorem]
    If $G$ is a finite group and $|G|=p^{\alpha}m$, where $p \nmid m$ then $G$ has a subgroup of order $p^{\alpha}$.
\end{theorem}

\begin{proof}
    Next Chapter
\end{proof}

\begin{definition}
    Let $G$ be a group and $H,K \le G$ and define
    \[ HK = \{hk \mid h \in H, k \in K\} \]
\end{definition}

\begin{proposition}
    If $H$ and $K$ are finite subgroups of a group then
    \[ |HK| = \frac{|HK|}{|H \cap K|} \]
\end{proposition}

\begin{proof}
    Notice that 
    \[ HK = \bigcup_{h \in H} hK \]
    Since, each coset has size $|K|$ it is enough to find the number of left cosets of form $hK$ where $h \in H$. But $h_1 K = h_2 K$ where
    $h_1,h_2 \in H$ if and only if $h_2^{-1}h_1 \in K$. Thus,
    \[ h_1 K = h_2 K \iff h_2^{-1}h_1 \in (H \cap K) \iff h_1 (H \cap K) = h_2 (H \cap K) \]
    Thus the number of distinct cosets of the form $hK$ where $h \in H$ is the number of distinct left cosets of $H \cap K$ in $H$.
    Thus, by lagrange's theorem we've the number of distinct left cosets of $H \cap K$ in $H$ equal to $\frac{|H|}{|H\cap K|}$.
    Since, there are $\frac{|H|}{|H\cap K|}$ distinct cosets and each coset has a size of $|K|$ we get our desired formula.
\end{proof}

\begin{proposition}
    If $H$ and $K$ are subgroups of a group, $HK$ is a subgroup if and only if $HK=KH$.
\end{proposition}

\begin{proof}
Assume first that \( HK = KH \) and let \( a, b \in HK \). We prove \( ab^{-1} \in HK \), so \( HK \) is a subgroup by the subgroup criterion.
Let
\[
a = h_1 k_1 \quad \text{and} \quad b = h_2 k_2,
\]
for some \( h_1, h_2 \in H \) and \( k_1, k_2 \in K \). Then \( b^{-1} = k_2^{-1} h_2^{-1} \), so 
\[
ab^{-1} = h_1 k_1 k_2^{-1} h_2^{-1}.
\]
Let \( k_3 = k_1 k_2^{-1} \in K \) and \( h_3 = h_2^{-1} \). Thus \( ab^{-1} = h_1 k_3 h_3 \). Since \( HK = KH \),
\[
k_3 h_3 = h_4 k_4 \quad \text{for some } h_4 \in H, \ k_4 \in K.
\]
Therefore,
\[
ab^{-1} = h_1 (h_4 k_4) = (h_1 h_4) k_4,
\]
and since \( h_1 h_4 \in H \), \( k_4 \in K \), we obtain \( ab^{-1} \in HK \), as desired.

Conversely, assume that \( HK \) is a subgroup of \( G \). Since \( K \leq HK \) and \( H \leq HK \), by the closure property of subgroups,
 \( KH \subseteq HK \). To show the reverse containment, let \( hk \in HK \). Since \( HK \) is a subgroup, \( hk = a^{-1} \) for some
  \( a \in HK \). If \( a = h_1 k_1 \), then
\[
hk = (h_1 k_1)^{-1} = k_1^{-1} h_1^{-1} \in KH,
\]
so \( HK \subseteq KH \). Hence \( HK = KH \), completing the proof.
    
\end{proof}

\begin{proposition}
    If \( H \) and \( K \) are subgroups of \( G \) and \( H \leq N_G(K) \), then \( HK \) is a subgroup of \( G \). 
    In particular, if \( K \trianglelefteq G \) then \( HK \leq G \) for any \( H \leq G \).
\end{proposition}

\begin{proof}
    We prove \( HK = KH \). Let \( h \in H \), \( k \in K \). By assumption, \( hkh^{-1} \in K \), hence
    \[
    hk = (hkh^{-1})h \in KH.
    \]
    This proves \( HK \subseteq KH \). Similarly, \( kh = h(h^{-1}kh) \in HK \), proving the reverse containment. 
    
    For the second statement: if \( K \trianglelefteq G \), then \( N_G(K) = G \), so in particular \( H \leq N_G(K) \)
    for any subgroup \( H \leq G \). The result then follows from the first part.
\end{proof}

\begin{definition}
    If $A$ is any subset of $N_G(K)$(or $C_G(K)$), we shall say $A$ \textit{normalizes} $K$(\textit{centralizes} $K$, respectively). 
\end{definition}