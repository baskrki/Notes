\subsection{Subgroups Generated by Subsets of a Group}

\begin{proposition}
    If $\mathcal{A}$ is any non empty collection of subsets of $G$ then the intersection of all members of $\mathcal{A}$ is also a 
    subgroup of $G$. 
\end{proposition}

\begin{proof}
    Trivial.
\end{proof}

\begin{definition}
    If $A$ is any subset of group $G$ define
    \[ \langle A \rangle = \bigcap_{\substack{A \subseteq H \\ H \le G}} H \]
    This is called subgroup generated by $A$.
\end{definition}

\begin{definition}
    Let $A=\{a_1,\ldots,a_n\}$ then define
    \[ \bar{A} = \{a_1^{\epsilon_1} a_2^{\epsilon_2} \cdots a_n^{\epsilon_n} \mid n \in \mathbb{Z}_{\ge 0}, a_i \in A, \epsilon_{i}=\pm1 \} \]
    where $\bar{A}=\{1\}$ if $A= \emptyset$.
\end{definition}

\begin{remark}
    Here, $a_i$'s need not to be distinct.
\end{remark}

\begin{proposition}
    $\bar{A}=\langle A \rangle$
\end{proposition}

\begin{proof}
    First we prove that $\bar{A}$ is a subgroup. Note that $\bar{A}\neq \emptyset$. If $a, b  \in \bar{A}$ then write 
    $a=a_1^{\epsilon_1} \cdots a_n^{\epsilon_n}$ and $b=b_1^{\delta_1} \cdots b_m^{\delta_m}$ then one can check that $ab^{-1} \in \bar{A}$.
    Thus, $\bar{A}$ is a subgroup of $G$.
    
    Now, since $A \subseteq \bar{A}$ as $a = a^1$ for every $a \in A$, we can say that $\langle A \rangle \subseteq \bar{A}$. It is because
    $\langle A \rangle$ is the intersection of all the subgroups containing $A$. Now, since $\langle A \rangle$ contains $A$ and is a group,
    it must contain every element of form $a_1^{\epsilon_1} \cdots a_n^{\epsilon_n}$ thus $\bar{A} \subseteq \langle A \rangle$. This
    completes the proposition.
\end{proof}

\eject

\subsection*{Problems and Solutions}

\paragraph{1.}  Prove that if $H$ is a subgroup then $\langle H \rangle = H$.

\vspace{4mm}
\textit{Solution :} From the definition,
\[ \langle H \rangle = \bigcap_{\substack{H \subseteq K \\ K \le G}} K \]
Since, $H \subseteq H$ and $H \le G$ thus $\langle H \rangle \subseteq H$. But also $H \subseteq \langle H \rangle$.

\paragraph{2.} Prove if $A$ is a subset of $B$ then $\langle A \rangle \le \langle B \rangle$. Give an example of $A \subseteq B$
with $A \neq B$ but $\langle A \rangle = \langle B \rangle$.

\vspace{4mm}
\textit{Solution :} From the definition we have,
\[ \langle B \rangle = \bigcap_{\substack{B \subseteq K \\ K \le G}} K \]
Since, $A \subseteq B$ we have $\langle A \rangle \le \langle B \rangle$. For the example, take $G=D_{16}$ and 
$A = \{r\}$ and $B=\{r,r^3\}$.

\paragraph{3.} Prove if $H$ is an abelian subgroup of $G$ then $\langle H, Z(G) \rangle$ is abelian. Give and explicit example 
of a abelian subgroup $H$ such that $\langle H, C_G(H) \rangle$ is not abelian.

\vspace{4mm}
\textit{Solution :} We know that, $\bar{A}=\langle A \rangle$ thus 
\[ \langle H,Z(G) \rangle = \{a_1^{\epsilon_1} \cdots a_n^{\epsilon_n}
\mid e_i = \pm 1 , n \in \mathbb{Z}_{\ge 0}, a_i \in H \cup Z(G) \}
\]
So, if you take two elements from $\langle H, Z(G) \rangle$, they will commute thus $\langle H, Z(G) \rangle$ is an abelian group.
For the example, choose $H=\{1,r^2\}$ and $G=D_8$.

\paragraph{3. } Prove that $H$ is a subgroup then $H$ is generated by $H-\{1\}$.

\vspace{4mm}
\textit{Solution :} Since, we know that $\bar{A}=\langle A \rangle$ thus
\[ \langle H-\{1\} \rangle = \{ a_1^{\epsilon_1} \cdots a_n^{\epsilon_n} \mid a_i \in H-\{1\}, n \in \mathbb{Z}_{\ge 0} ,
 \epsilon_i = \pm 1 \} \]

Thus, for $a \in H$ and $a \neq 1$, $a \in \langle H-\{1\} \rangle$ and also $1=a^{1}a^{-1} \in \langle H - \{1\} \rangle$.
Also, $a \in \langle H - \{1\} \rangle$ is just some combination of elements in $H$ thus $a \in H$. Thus, we have
\[ \langle H - \{1\} \rangle = H \]

\paragraph{4. } Prove that the multiplicative group of positive rational numbers is generated by the set 
\{ $\frac{1}{p} \mid$ $p$ is a prime \}.

\vspace{4mm}
\textit{Solution :} Since,
\[
\left \langle \left \{ \frac{1}{p} \mid p \text{ is a prime} \right \} \right \rangle =  
\left \{ \frac{p_1^{a_1} \cdots p_{n}^{a_n}}{q_1^{b_1} \cdots q_m^{a_m}} \mid p_i, q_i \in \mathbf{Primes} \right \} = \mathbb{Q}_{>0}
\]

\paragraph{5.} A group $H$ is called \textit{finitely generated} if there is a finite set $A$ such that $H = \langle A \rangle$.
\begin{enumerate}
    \item[(a)] Prove that every finite group is finitely generated.
    \item[(b)] Prove that $\mathbb{Z}$ is finitely generated.
    \item[(c)] Prove that every finitely generated subgroup of the additive group $\mathbb{Q}$ is cyclic. [If $H$ is a finitely generated subgroup of $\mathbb{Q}$, show that $H \le \langle \frac{1}{k} \rangle$, where $k$ is the product of all the denominators which appear in a set of generators for $H$.]
    \item[(d)] Prove that $\mathbb{Q}$ is not finitely generated.
\end{enumerate}

\paragraph{6.} Exhibit a proper subgroup of $\mathbb{Q}$ which is not cyclic.

\paragraph{7.} A subgroup $M$ of a group $G$ is called a \textit{maximal subgroup} if $M \neq G$ and the only subgroups of $G$ which contain $M$ are $M$ and $G$.
\begin{enumerate}
    \item[(a)] Prove that if $H$ is a proper subgroup of the finite group $G$ then there is a maximal subgroup of $G$ containing $H$.
    \item[(b)] Show that the subgroup of all rotations in a dihedral group is a maximal subgroup.
    \item[(c)] Show that if $G = \langle x \rangle$ is a cyclic group of order $n \ge 1$ then a subgroup $H$ is maximal if and only if $H = \langle x^p \rangle$ for some prime $p$ dividing $n$.
\end{enumerate}

\paragraph{8.} This is an exercise involving Zorn's Lemma (see Appendix I) to prove that every nontrivial finitely generated group possesses maximal subgroups. Let $G$ be a finitely generated group, say $G = \{g_1, g_2, \dots, g_n\}$, and let $\mathcal{S}$ be the set of all proper subgroups of $G$. Then $\mathcal{S}$ is partially ordered by inclusion. Let $\mathcal{C}$ be a chain in $\mathcal{S}$.
\begin{enumerate}
    \item[(a)] Prove that the union, $H$, of all the subgroups in $\mathcal{C}$ is a subgroup of $G$.
    \item[(b)] Prove that $H$ is a proper subgroup. [If not, each $g_i$ must lie in $H$ and so must lie in some element of the chain $\mathcal{C}$. Use the definition of a chain to arrive at a contradiction.]
    \item[(c)] Use Zorn's Lemma to show that $\mathcal{S}$ has a maximal element (which is, by definition, a maximal subgroup).
\end{enumerate}

\paragraph{9.} Let $p$ be a prime and let 
\[
Z = \{ z \in \mathbb{C} \mid z^{p^m} = 1 \text{ for some } n \in \mathbb{Z}^+ \}
\] 
(so $Z$ is the multiplicative group of all $p$-power roots of unity in $\mathbb{C}$). For each $k \in \mathbb{Z}^+$ let 
\[
H_k = \{ z \in Z \mid z^{p^k} = 1 \} 
\] 
(the group of $p^k$th roots of unity). Prove the following:
\begin{enumerate}
    \item[(a)] $H_k \le H_m$ if and only if $k \le m$.
    \item[(b)] $H_k$ is cyclic for all $k$ (assume that for any $n \in \mathbb{Z}^+$, $\{ e^{2 \pi i t / n} \mid t = 0,1,\dots, n-1\}$ is the set of all $n$th roots of $1$ in $\mathbb{C}$).
    \item[(c)] Every proper subgroup of $Z$ equals $H_k$ for some $k \in \mathbb{Z}^+$ (in particular, every proper subgroup of $Z$ is finite and cyclic).
    \item[(d)] $Z$ is not finitely generated.
\end{enumerate}

\paragraph{10.} A nontrivial abelian group $A$ (written multiplicatively) is called \textit{divisible} if for each element $a \in A$ and each nonzero integer $k$ there is an element $x \in A$ such that $x^k = a$, i.e., each element has a $k$th root in $A$ (in additive notation, each element is the $k$th multiple of some element of $A$).
\begin{enumerate}
    \item[(a)] Prove that the additive group of rational numbers, $\mathbb{Q}$, is divisible.
    \item[(b)] Prove that no finite abelian group is divisible.
\end{enumerate}

\paragraph{11.} Prove that if $A$ and $B$ are nontrivial abelian groups, then $A \times B$ is divisible if and only if both $A$ and $B$ are divisible groups.

