\subsection{Cyclic and Cyclic Subgroups}

\begin{definition}
    A group $G$ is called \textit{cyclic} if if can be generated by a single element i.e there is some $x \in G$ such that 
    $H= \{x^n \mid n \in \mathbb{Z}\}$. We write is as $G=\langle x \rangle$ and say $G$ is generated by $x$.
\end{definition}


\begin{proposition}
    If $H= \langle x \rangle$ then $|H|=|x|$.
\end{proposition}

\begin{proof}
    Suppose $|x|=n < \infty$ then $1,x,\ldots,x^{n-1}$ are all distinct. Thus $|H|$ is at least $n$. Now, using the division algorithm
    we can show that these are all of them.
    
    Suppose now $|x| = \infty$ then that means there is no finite $n \in \mathbb{Z}$ s.t $x^n = 1$. If $x^b=x^c$ then $x^{b-c}=1$ 
    contradicting the fact that there is no $n$ s.t $x^n=1$. Thus, all of the powers of $x$ are different and thus $|H|=\infty$.
\end{proof}

\begin{proposition}
    Let $G$ be a group and let $x \in G$. If $x^m=1$ for some $m \in \mathbb{Z}$ then $|x|$ divides $m$.
\end{proposition}

\begin{proposition}
    Any two cyclic group of same order are isomorphic. More specifically, 
    \begin{enumerate}
        \item If $n \in \mathbb{Z}^{+}$ and $\langle x \rangle$ and $\langle y \rangle$ are both cyclic groups of order $n$, then the map
                \[ \varphi: \langle x \rangle \to \langle y \rangle \]
                \[ x^k \to y^k \]
            is well defined and is an isomorphism.
        \item If $\langle x \rangle$ is an infinite cyclic group then
                \[\varphi: \mathbb{Z} \to \langle x \rangle\]
                \[k \to x^k\]
            is well defined and is an isomorphism.
    \end{enumerate}
\end{proposition}

\begin{proposition}
    Let $G$ be a group, let $x \in G$ and let $a \in \mathbb{Z}\setminus\{0\}$ then
    \begin{enumerate}
        \item If $|x|=\infty$ then $|x^a| = \infty$
        \item If $|x|=n < \infty$ then $|x^a|=\frac{n}{(n,a)}$
    \end{enumerate} 
\end{proposition}

\begin{proof}
    $1.$ is pretty simple. Suppose $|x^a|=k$ then $x^{ak}=1$ now $n \mid ak$.
    Write $(n,a)=d$ and $n=du$ and $a=dv$. Then, $du \mid dvk \implies u \mid k$. But
    \[ x^a = x^{dv} \implies (x^{du})^v = (x^{n})^v = 1  \]
    \[ \implies (x^{a})^u=1 \]
    \[ \implies k \mid u \]. 
    Thus, $k=u \implies n=dk \implies k = \frac{n}{d} = \frac{n}{(n,a)}$.
\end{proof}


\begin{proposition}
    Let $H=\langle x \rangle$. 
    \begin{enumerate}
        \item Assume $|x|= \infty$. Then $H = \langle x^a \rangle \iff a = \pm 1$.
        \item Assume $|x|=n < \infty$. Then $H = \langle x^a \rangle \iff (a,n)=1$.
            The number of generators of $H$ is $\varphi(n)$.  
    \end{enumerate}
\end{proposition}

\begin{proof}
    For $1.$ we know $x \in \langle x^a \rangle$ as $\langle x^a \rangle = \langle x \rangle$ thus 
    \[ x=x^{ak} \implies ak = 1 \implies a = \pm 1 \]
    
    For $2.$ we know $H=\langle x^a \rangle \implies |H|=|x^a| \iff |x|=|x^a|$ ,
    \[ \iff \frac{n}{(n,a)}=n \]
    \[ \iff (n,a)=1 \]
    Since, the number of positive integers less than $n$ and co-prime to $n$ are exactly $\varphi(n)$, thus the number of generators are
    exactly equal to $\varphi(n)$.
\end{proof}

\begin{theorem}
Let $H = \langle x \rangle$ be a cyclic group. 
\begin{enumerate}
    \item Every subgroup of $H$ is cyclic. More precisely, if $K \leq H$, then either 
    $K = \{1\}$ or $K = \langle x^d \rangle$, where $d$ is the smallest positive integer such that $x^d \in K$.
    
    \item If $|H| = \infty$, then for any distinct nonnegative integers $a$ and $b$, 
    $\langle x^a \rangle \neq \langle x^b \rangle$. 
    
    \item If $|H| = n < \infty$, then for each positive integer $a$ dividing $n$ 
    there is a unique subgroup of $H$ of order $a$. This subgroup is the cyclic group 
    $\langle x^d \rangle$, where $d = \tfrac{n}{a}$. \\
    Furthermore, for every integer $m$, 
    $\langle x^m \rangle = \langle x^{(n,m)} \rangle$, so that the subgroups of $H$ correspond 
    bijectively with the positive divisors of $n$.
\end{enumerate}
\end{theorem}

\begin{proof}
    $1.$ and $2.$ are pretty easy. For $3.$ the cyclic group $\langle x^{n/a} \rangle$ has order $a$. To prove uniqueness,
    suppose $K$ is any subgroup of $H$ with order $a$, then $\langle x^b \rangle = K$ where $b$ is the smallest positive 
    integer $b$ s.t $x^b \in K$(this is from $1.$). Thus,
    \[ |\langle x^{n/a} \rangle|=|\langle x^b \rangle| \implies \frac{n}{d}=a=\frac{n}{(n,b)}\]
    \[ \implies d=(n,b) \implies d \mid b \] 
    Hence, $\langle x^b \rangle \le \langle x^d \rangle $ and since they both have same order $\langle x^b \rangle = \langle x^d \rangle $.
    
    For the assertion on $3.$, one can prove that $\langle x^{m} \rangle  \le \langle x^{(n,m)}\rangle $ as $(n,m) \mid n$, and 
    since they have same order $\langle x^{m}\rangle = \langle x^{(n,m)} \rangle $. This means that the number of subgroups has a bijection
    with the divisors of $n$.
\end{proof}

\eject

\subsection*{Problems and Solutions}

\paragraph{Problem :} Find all subgroup of $\mathbf{Z}_{45} = \langle x \rangle$, giving a generator of each. Describe the containment 
between these subgroups.

\vspace{4mm}
\textit{Solution :} There are exactly $6$ different subgroups of $\mathbf{Z}_{45}$. Since there is a one to one correspondence between
divisors of $45$ and subgroups of $\mathbf{Z}_45$ we can list all of them,
\[ \{1\}, \langle r \rangle , \langle r^3 \rangle , \langle r^5 \rangle, \langle r^9 \rangle, \langle r^15 \rangle \]

\paragraph{Problem :} If $x$ is an element of a finite group $G$ and $|x|=|G|$. Prove that $G= \langle x \rangle$.

\vspace{4mm}
\textit{Solution :} Let $|x|=n$. We know that $\{1,x,\ldots,x^{n-1}\}$ is a subgroup of $G$. Since, it is a subgroup and has the same 
order as $G$ thus $G=\{1,x,\ldots,x^{n-1}\}=\langle x \rangle$.

\paragraph{Problem :} Let $\mathbf{Z}_{48}=\langle x \rangle$ and use isomorphism $\mathbb{Z}/48\mathbb{Z} \cong \mathbf{Z}_{48}$ with 
$[1] \mapsto x$ to find all the subgroups of $Z_{48}$.

\vspace{4mm}
\textit{Solution :} If $\langle [x] \rangle$ is a cyclic subgroup of $\mathbb{Z}/48\mathbb{Z}$ then $\langle \varphi([x]) \rangle$ is a subgroup
of $\mathbf{Z}_{48}$ where $\varphi$ is the isomorphic map.

\paragraph{Problem :} Let $\mathbf{Z}_{48}=\langle x \rangle$. For which integer $a$ does the map $\varphi_a$ defined by 
$\varphi_a : [1] \mapsto x^a$ extends to an isomorphism from $\mathbb{Z} / 48 \mathbb{Z}$ to $\mathbf{Z}_{48}$.

\vspace{4mm}
\textit{Solution :}  We already know it is an homomorphism as
\[ \varphi([u]+[v])=(x^{a})^{u+v}= (x^{a})^u (x^{a})^v = \varphi([u]) \varphi([v]) \]
But to be an isomorphism $x^{na}$ needs to cover $\mathbf{Z}_{48}$ for all $n \in \mathbb{Z}$. Thus,
\[ \mathbf{Z}_{48} = \langle x \rangle = \langle x^a \rangle \]
\[ \implies (48,a)=1 \]
So, for all the $a$ which are co-prime to $48$ the map, $\varphi_a$ is an isomorphism. 

\paragraph{Problem :} Let $\mathbf{Z}_{36}=\langle x \rangle$. For which integer $a$ does the map $\psi_a : [1] \mapsto x^a$ extend to an
well defined homomorphism from $\mathbb{Z}/48\mathbb{Z}$ onto $\mathbf{Z}_{36}$. Can $\psi_a$ ever be surjective?

\vspace{4mm}
\textit{Solution :}  One can check that the map is a homomorphism.  Now, we need to show that 
\[ [u]=[v] \implies \psi_a([u])=\psi_a([v]) \]
If $[u]=[v]$ then $u-v = 48 m$ 
\[ 1=\psi_a([0])=\psi_a([u-v])=x^{a(u-v)} = x^{48am} \]
\[ \implies 36 \mid 48 am \]
\[ \implies 3 \mid am \]
Since $3 \mid am$ must hold for all integer $m$, if $3 \nmid a$ then $3 \mid m$ for all integer $m$ which is clearly absurd thus $3 \mid a$.
Thus, $x^{48 \cdot 3k \cdot m} = 1 $ as $36 \mid 144km$. Thus,
\[ x^{a(u-v)}=1  \implies x^{au}=x^{av} \implies \psi_a([u])=\psi_a([v])\]

\paragraph{Problem :} Find a presentation for $\mathbf{Z}_n$ with one generator.

\vspace{4mm}
\textit{Solution :} $\mathbf{Z}_{n}=\langle r \mid r^n = 1 \rangle$.

\paragraph{Problem :} Show that if $H$ is any group with $h^n=1$ then there exists a unique homomorphism from 
$\mathbf{Z}_n=\langle x \rangle$ to $H$ such that $x \mapsto h$.

\vspace{4mm}
\textit{Solution :} Define $\psi : \mathbf{Z}_n \to H$ by $\psi(x^k)=h^k$. This is a homomorphism and is unique because the output is 
completely determined by $h$. 

\paragraph{Problem :} Show that if $H$ is any group and $h$ is an element of $H$, then there is a unique homomorphism from $\mathbb{Z}$ 
to $H$ such that $1 \to h$.

\vspace{4mm}
\textit{Solution :} Define $\psi : \mathbb{Z} \to H$ by $\psi(k)=h^k$. This is a homomorphism and is unique as the output is completely 
determined by $h$.

\paragraph{Problem :} Let $p$ be a prime and $n$ be a positive integer. Show that if $x$ is an element of the group $G$ such that $x^{p^n}=1$
then $|x|=p^m$ for some $m \le n$.

\vspace{4mm}
\textit{Solution :} We know that if $x^n=1$ then $|x|$ must divide $n$. Thus, $|x|$ must divide $p^n$ but the only divisors of $p^n$ are 
powers of $p$. Thus, $|x|=p^m$ for some $m \le n$.


\paragraph{Problem :} Show that $(\mathbb{Z}/2^n \mathbb{Z})^{\times}$ is not cyclic.

\vspace{4mm}
\textit{Solution :} Consider $\{1,-1\}$ and ${1,1+2^{n-1}}$. They both are subgroups of order $2$. But a cyclic group has exactly $1$ 
subgroup of order $d$, where $d$ is the divisor of order of the cyclic group $G$. But we found two distinct subgroups of the group with same
order.

\paragraph{Problem :} Let $G$ be a finite group and let $x \in G$.
  \begin{enumerate}
    \item Prove that if $g \in N_G(\langle x \rangle)$ then $gxg^{-1} = x^a$ for some $a \in \mathbb{Z}$.
    \item Prove conversely that if $gxg^{-1} = x^a$ for some $a \in \mathbb{Z}$ then $g \in N_G(\langle x \rangle)$. 
    [Show first that $g x^k g^{-1} = (gxg^{-1})^k = x^{ak}$ for any integer $k$, so that 
    $g \langle x \rangle g^{-1} \leq \langle x \rangle$. If $x$ has order $n$, show the elements 
    $gx^i g^{-1}$, $i=0,1,\dots,n-1$, are distinct, so that 
    $|g \langle x \rangle g^{-1}| = |\langle x \rangle| = n$ and conclude that 
    $g \langle x \rangle g^{-1} = \langle x \rangle$.]
  \end{enumerate}

\paragraph{Problem :} Let $G$ be a cyclic group of order $n$ and let $k$ be an integer relatively prime to $n$. 
  Prove that the map $x \mapsto x^k$ is surjective. Use Lagrange's Theorem 
  (Exercise 19, Section 1.7) to prove the same is true for any finite group of order $n$. 
  (For such $k$ each element has a $k$th root in $G$. It follows from Cauchy’s Theorem in 
  Section 3.2 that if $k$ is not relatively prime to the order of $G$ then the map $x \mapsto x^k$ 
  is not surjective.)

\paragraph{Problem :} Let $\mathbf{Z}_n$ be a cyclic group of order $n$ and for each integer $a$ let
  \[
    \sigma_a : \mathbf{Z}_n \to \mathbf{Z}_n \quad \text{by} \quad \sigma_a(x) = x^a \quad \text{for all } x \in Z_n.
  \]
  \begin{enumerate}
    \item Prove that $\sigma_a$ is an automorphism of $Z_n$ if and only if $a$ and $n$ are relatively prime.
    \item Prove that $\sigma_a = \sigma_b$ if and only if $a \equiv b \pmod{n}$.
    \item Prove that every automorphism of $Z_n$ is equal to $\sigma_a$ for some integer $a$.
    \item Prove that $\sigma_a \circ \sigma_b = \sigma_{ab}$. Deduce that the map 
    $a \mapsto \sigma_a$ is an isomorphism of $(\mathbb{Z}/n\mathbb{Z})^\times$ 
    onto the automorphism group of $Z_n$ (so $\operatorname{Aut}(Z_n)$ is an abelian group 
    of order $\varphi(n)$).
  \end{enumerate}
