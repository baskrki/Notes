\subsection{Product and Quotients of Vector Spaces}

\subsubsection{Products of Vector Spaces}

\begin{definition}
    Suppose $V_1,\ldots,V_m$ are vector spaces over $\mathbf{F}$.
    \begin{itemize}
        \item The product $V_1 \times \cdots \times V_m$ is defined by
            \[ V_1 \times \cdots \times V_m = \{(v_1,\ldots,v_m) : v_1 \in V, \ldots, v_m \in V\}  \]
        \item Addition on $V_1 \times \cdots \times V_m$ is defined by
            \[ (u_1,\ldots,u_m) + (v_1,\ldots,v_m) = (u_1 + v_1 , \ldots, u_m + v_m) \]
        \item Scalar Multiplication on $V_1 \times \cdots \times V_m$ is defined by
            \[ \lambda (v_1, \ldots, v_m) = (\lambda v_1, \ldots, \lambda v_m) \]
    \end{itemize}
\end{definition}

\begin{proposition}
    Suppose $V_1,\ldots,V_m$ are vector spaces over $\mathbf{F}$. Then $V_1 \times \cdots \times V_m$ is a vector space over $\mathbf{F}$.
\end{proposition}

\begin{proof}
    Just check the vector axioms.
\end{proof}

\begin{proposition}
    Suppose $V_1, \ldots, V_m$ are finite-dimensional vector spaces. Then $V_1 \times \cdots \times V_m$ is finite-dimensional and
    \[ \dim (V_1 \times \cdots V_m) = \dim V_1 + \cdots + \dim V_m \]
\end{proposition}

\begin{proof}
    Choose a basis of $V_k$ and consider every element of $V_1 \times \cdots \times V_k$ that equals a element from the basis of the vector
    $V_k$ in the $k$-th sloth and $0$ in others. The list of vector spans $V_1 \times \cdots \times V_m$ and is linearly independent. 
    Thus, it is the basis of $V_1 \times \cdots \times V_m$. The length of the basis is $\dim V_1 + \cdots + \dim V_m$.
\end{proof}

\begin{proposition}
    Suppose that $V_1, \ldots, V_m$ are subspaces of $V$. Define a linear map $\Gamma : V_1 \times \cdots \times V_m \to V_1 + \cdots + V_m$
    by
    \[ \Gamma(v_1, \cdots, v_m) = v_1 + \cdots + v_m \]
    Then $V_1 + \cdots + V_m$ is a direct sum if and only if  $\Gamma$ is injective.
\end{proposition}

\begin{proof}
    If $V_1 + \cdots + V_m$ is a direct sum then the only way we can write $0$ is by choosing $0$ from each $V_i$. Thus,
    \[ \Gamma(v_1, \ldots, v_m) = 0 \iff v_1 = v_2 = \cdots = v_m = 0 \]
    Thus, $\operatorname{null} \Gamma = \{0\}$ which implies $\Gamma$ is injective.
    
    Now, suppose $\Gamma$ is injective then $\operatorname{null} \Gamma = \{0\}$, which means that the only element that gets mapped to $0$
    is $(0,\ldots,0)$. Thus, the only way to write $0$ is by choosing $0$ from each $V_i$. Thus, $V_1 + \cdots + V_m$ is a direct sum. 
\end{proof}

\begin{proposition}
    Suppose $V$ is finite-dimensional and $V_1, \ldots, V_m$ are subspaces of $V$. Then $V_1 + \cdots + V_m$ is direct sum if and only if
    \[ \dim (V_1 + \cdots V_m) = \dim V_1 + \cdots \dim V_m \]
\end{proposition}

\begin{proof}
    The map $\Gamma$ is surjective. And $V_1 + \cdots + V_m$ \text{ is a direct sum} 
    \begin{align*}
        & \iff \Gamma \text{ is injective} \\
        & \iff \operatorname{null} \Gamma = \{0\} \\
        & \iff \dim (V_1 \times \cdots \times V_m) = \dim(V_1 + \cdots + V_m)
    \end{align*}
    Combining the result of \textbf{Proposition 1.26.} we get our desired result.
\end{proof}


\subsubsection{Quotients Spaces}

\begin{definition}
    Suppose $v \in V$ and $U \subseteq V$. Then $v+U$ is a subset of $V$ defined by
    \[ v+U = \{v+u \mid u \in U\} \]
\end{definition}

\begin{definition}
    For $v \in V$ and $U$ a subset of $V$, the set $v+U$ is said to be a \textit{translate} of $U$.
\end{definition}

\begin{definition}
    Suppose $U$ is a subspace of $V$. Then the \textit{quotient space} $V/U$ is the set of all translate of $U$,
    \[ V/U = \{ v+ U \mid v \in V \} \]
\end{definition}

\begin{proposition}
    Suppose $U$ is a subspace of $V$ and $v,w \in V$ then
    \[ v-w \in U \iff v+U = w + U \iff (v+U) \cap (u+V) \neq \emptyset \] 
\end{proposition}

\begin{proof}
    Suppose $v-w \in U$ then $v=w+u'$ for some $u \in U$ thus, $v+u = w+(u'+u) \in w + U$ which implies $v+U \subseteq w+U$.
    Thus, similarly $w+U \subseteq v+U \implies v+U = w+U \implies (v+U) \cap (w+U) \neq \emptyset$.

    Now, suppose $(v+U) \cap (w+U) \neq \emptyset$ then $v+u_1 = w+u_2$ for some $u_1,u_2 \in U$ which implies $v-w \in U$, which implies 
    $v+U=w+U$. And $v+U = w+U \implies v-w \in U$. Thus, we proved every direction of the proof.
\end{proof}

\begin{definition}
    Suppose $U$ is a subspace of $V$. Then \textit{addition} and \textit{scalar multiplication} are defined on $V/U$ by
    \[ (v+U)+(w+U) = (v+w)+U\]
    \[ \lambda(v+U) = (\lambda v) + U \]
    for all $v, w \in V$ and all $\lambda \in F$.
\end{definition}

\begin{proposition}
    Suppose $U$ is a subspace of $V$. Then $V/U$, with the operations of addition and scalar multiplication as defined above, is a vector
    space.
\end{proposition}

\begin{proof}
    Just use previous definitions and check the vector axioms. 
\end{proof}

\begin{proposition}
    Suppose $U$ is a subspace of $V$. The \textit{quotient map} $\pi : V \to V/U$ is a linear map defined by 
    \[ \pi(v) = v+U \]
    for each $v \in V$.
\end{proposition}

\begin{proof}
    Note that $\pi(a+b)=(a+b)+U=(a+U)+(b+U)=\pi(a)+\pi(b)$ and $\pi(\lambda a) = (\lambda a) + U = \lambda(a+U) = \lambda \pi(a)$. 
\end{proof}

\begin{proposition}
    Suppose $V$ is finite-dimensional vector space and $U$ is a subspace of $V$ then
    \[ \dim V/U  = \dim V - \dim U\]
\end{proposition}

\begin{proof}
    We use the map quotient map $\pi$ introduced before. We know that $a+U = 0+U \iff a \in U$ this $\operatorname{null} \pi = U$ and 
    $\operatorname{range} \pi = V/U$ as the map is surjective. Thus, using the fundamental theorem of linear map we get our desired result. 
\end{proof}

\begin{definition}
    Suppose $T \in \LL(V,W)$. Define $\widetilde{T} : V/(\operatorname{null} T) \to W$ by
    \[ \widetilde{T}(v+\operatorname{null} T) = Tv \]
\end{definition}

This map indeed is well-defined as $u+\operatorname{null} T = v + \operatorname{null} T \implies v-u \in \operatorname{null} T$ and thus
$T(v-u) = 0 \implies Tv=Tu$. Also, the map is a linear map.


\begin{proposition}
    Suppose $T \in \LL(V,W)$ then
    \begin{enumerate}
        \item[(a)] $\widetilde{T} \circ \pi = T$ where $\pi$ is the quotient map with $U=\operatorname{null} T$
        \item[(b)] $\widetilde{T}$ is injective
        \item[(c)] $\operatorname{range} T = \operatorname{range} \widetilde{T}$
        \item[(d)] $V/(\operatorname{null} T)$ and $\operatorname{range} T $ are isomorphic vector spaces.   
    \end{enumerate}
\end{proposition}

\begin{proof} We prove each point individually 
    \begin{enumerate}
        \item[(a)] If $v \in V$ then $\widetilde{T} \circ \pi (v) = \widetilde{T}(v+\operatorname{null} T) = Tv$ as desired.
        \item[(b)] If $\widetilde{T}(v+\operatorname{null} T )=0$ then $Tv=0$ thus $v \in \operatorname{null} T $. Thus,
            $v+\operatorname{null} T = 0 + \operatorname{null} T \implies \operatorname{null} \widetilde{T} = \{0+\operatorname{null} T\}$.
        \item[(c)] By definition of $\widetilde{T}$.
        \item[(d)] From (b) and (c).  
    \end{enumerate}
\end{proof}

\subsubsection{Exercise}

\paragraph{Problem :} Suppose $T$ is a function from $V$ to $W$. The graph of $T$ is the subset of $V \times W$ defined by
\[ \text{graph of } T = \{(v,Tv) \in V \times W \mid v \in V\} \]
Prove that $T$ is a linear map if and only if graph of $T$ is a subspace of $V \times W$.

\vspace{4mm}
\textit{Solution :} For $(\Rightarrow)$, you just need to check the vector axioms and just the definition of a linear map.
For the $(\Leftarrow)$, consider 
\[ (v,Tv) + (w,Tw) = (v+w,Tv+Tw) \]
Since, graph of $T$ is a subspace and all of its element is of the form $(v,Tv)$ for any $v$, it must be that $(v+w,T(v+w))=(v+w,Tv+Tw)$.
Similarly, $\lambda(v,Tv) = (\lambda v , \lambda Tv) = (\lambda v, T (\lambda v))$. And we're done.

\paragraph{Problem :} Suppose $V_1, \ldots, V_m$ are vector spaces such that $V_1 \times \cdots \times V_m$ is finite-dimensional. Prove that
$V_k$ is finite-dimensional for each $k=1,\cdots,m$.

\vspace{4mm}
\textit{Solution :} Let $\dim (V_1 \times \cdots \times V_m)=k$ and suppose $e_1,\ldots,e_k$ are the basis of $V_1 \times \cdots \times V_m$.
Let $e_i=(e_{1i},e_{2i},\ldots,e_{mi})$ for $1 \le i \le k$. Notice that to cover elements of $V_j$, only the $j$-th component of $e_i$'s 
play a role. Thus,
\[ \operatorname{span}\{e_{j1},e_{j2},\ldots,e_{jm}\} = V_k \]
Since, every span can be reduced to a basis we get that each $V_k$ are finite-dimensional.