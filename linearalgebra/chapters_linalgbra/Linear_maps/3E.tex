\subsection{Product and Quotients of Vector Spaces}

\subsubsection{Products of Vector Spaces}

\begin{definition}
    Suppose $V_1,\ldots,V_m$ are vector spaces over $\mathbf{F}$.
    \begin{itemize}
        \item The product $V_1 \times \cdots \times V_m$ is defined by
            \[ V_1 \times \cdots \times V_m = \{(v_1,\ldots,v_m) : v_1 \in V, \ldots, v_m \in V\}  \]
        \item Addition on $V_1 \times \cdots \times V_m$ is defined by
            \[ (u_1,\ldots,u_m) + (v_1,\ldots,v_m) = (u_1 + v_1 , \ldots, u_m + v_m) \]
        \item Scalar Multiplication on $V_1 \times \cdots \times V_m$ is defined by
            \[ \lambda (v_1, \ldots, v_m) = (\lambda v_1, \ldots, \lambda v_m) \]
    \end{itemize}
\end{definition}

\begin{proposition}
    Suppose $V_1,\ldots,V_m$ are vector spaces over $\mathbf{F}$. Then $V_1 \times \cdots \times V_m$ is a vector space over $\mathbf{F}$.
\end{proposition}

\begin{proof}
    Just check the vector axioms.
\end{proof}

\begin{proposition}
    Suppose $V_1, \ldots, V_m$ are finite-dimensional vector spaces. Then $V_1 \times \cdots \times V_m$ is finite-dimensional and
    \[ \dim (V_1 \times \cdots V_m) = \dim V_1 + \cdots + \dim V_m \]
\end{proposition}

\begin{proof}
    Choose a basis of $V_k$ and consider every element of $V_1 \times \cdots \times V_k$ that equals a element from the basis of the vector
    $V_k$ in the $k$-th sloth and $0$ in others. The list of vector spans $V_1 \times \cdots \times V_m$ and is linearly independent. 
    Thus, it is the basis of $V_1 \times \cdots \times V_m$. The length of the basis is $\dim V_1 + \cdots + \dim V_m$.
\end{proof}

\begin{proposition}
    Suppose that $V_1, \ldots, V_m$ are subspaces of $V$. Define a linear map $\Gamma : V_1 \times \cdots \times V_m \to V_1 + \cdots + V_m$
    by
    \[ \Gamma(v_1, \cdots, v_m) = v_1 + \cdots + v_m \]
    Then $V_1 + \cdots + V_m$ is a direct sum if and only if  $\Gamma$ is injective.
\end{proposition}

\begin{proof}
    If $V_1 + \cdots + V_m$ is a direct sum then the only way we can write $0$ is by choosing $0$ from each $V_i$. Thus,
    \[ \Gamma(v_1, \ldots, v_m) = 0 \iff v_1 = v_2 = \cdots = v_m = 0 \]
    Thus, $\operatorname{null} \Gamma = \{0\}$ which implies $\Gamma$ is injective.
    
    Now, suppose $\Gamma$ is injective then $\operatorname{null} \Gamma = \{0\}$, which means that the only element that gets mapped to $0$
    is $(0,\ldots,0)$. Thus, the only way to write $0$ is by choosing $0$ from each $V_i$. Thus, $V_1 + \cdots + V_m$ is a direct sum. 
\end{proof}

\begin{proposition}
    Suppose $V$ is finite-dimensional and $V_1, \ldots, V_m$ are subspaces of $V$. Then $V_1 + \cdots + V_m$ is direct sum if and only if
    \[ \dim (V_1 + \cdots V_m) = \dim V_1 + \cdots \dim V_m \]
\end{proposition}

\begin{proof}
    The map $\Gamma$ is surjective. And $V_1 + \cdots + V_m$ \text{ is a direct sum} 
    \begin{align*}
        & \iff \Gamma \text{ is injective} \\
        & \iff \operatorname{null} \Gamma = \{0\} \\
        & \iff \dim (V_1 \times \cdots \times V_m) = \dim(V_1 + \cdots + V_m)
    \end{align*}
    Combining the result of \textbf{Proposition 1.26.} we get our desired result.
\end{proof}