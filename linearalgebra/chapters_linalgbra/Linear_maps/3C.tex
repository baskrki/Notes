\subsection{Matrices}

\subsubsection{Representing a Linear Map by a Matrix}

\begin{definition}
    Suppose $m$ and $n$ are two non-negative integers. A $m \times n$ matrix is $A$ is a rectangular array of elements in $\mathbf{F}$ 
    with $m$ rows and $n$ columns
    \[ A = \begin{pmatrix}
        A_{1,1} && \cdots && A_{1,n} \\
        \vdots && && \vdots \\
        A_{m,1} && \cdots && A_{m,n} 
    \end{pmatrix} \]
    The $A_{i,j}$ represents the entry in $i$-th row and $j$-th column.
\end{definition}

\begin{definition}
    Suppose $T \in \LL(V,W)$ and $v_1, \ldots, v_n$ is the basis of $V$ and $w_1, \ldots, w_m$ is a basis of $W$. The matrix of 
    $T$ with respect to these bases is the $m \times n$ matrix $\mathcal{M}(T)$ whose entries $A_{i,j}$ are defined by
    \[ Tv_k = A_{1,k} w_1 + \cdots A_{m,k} w_m \] 
\end{definition}

\textbf{Examples :}

\vspace{4mm}
Suppose $T \in \LL(\mathbf{F}^2, \mathbf{F}^3)$ is defined by 
\[ T(x,y) = (x+3y, 2x+5y, 7x+9y) \]
Then,
\[ \mathcal{M}(T) = \begin{pmatrix}
    1 && 3 \\
    2 && 5 \\
    7 && 9 \\
\end{pmatrix} \] 

As $T(1,0) = 1(1,0,0) + 2(0,1,0) + 7(0,0,1)$ and $T(0,1) = 3(1,0,0)+5(0,1,0)+9(0,0,1)$.


\subsubsection{Addition and Scalar Multiplication of Matrices}

\begin{definition}
    The sum of two matrices of same size is obtained by adding corresponding entries in the matrices i.e 
    \[ 
    \begin{pmatrix}
        A_{1,1} && \cdots && A_{1,n} \\
        \vdots && && \vdots \\
        A_{m,1} && \cdots && A_{m,n} 
    \end{pmatrix} + \begin{pmatrix}
        B_{1,1} && \cdots && B_{1,n} \\
        \vdots && && \vdots \\
        B_{m,1} && \cdots && B_{m,n} 
    \end{pmatrix} \]
    \[= \begin{pmatrix}
        A_{1,1} + B_{1,1}&& \cdots && A_{1,n}+B_{1,n} \\
        \vdots && && \vdots \\
        A_{m,1} + B_{1,1} && \cdots && A_{m,n} + B_{m,n}
    \end{pmatrix}
     \]
\end{definition}

\begin{proposition}
    Suppose $S,T \in \LL(V,W)$. Then $\mathcal{M}(S+T) = \mathcal{M}(S) + \mathcal{M}(T)$.
\end{proposition}

\begin{proof}
    Follows from the definition.
\end{proof}

\begin{definition}
    The product of a scalar and a matrix is obtained by multiplying each entry by the scalar i.e 
    \[ \lambda \begin{pmatrix}
        A_{1,1} && \cdots && A_{1,n} \\
        \vdots && && \vdots \\
        A_{m,1} && \cdots && A_{m,n} 
    \end{pmatrix} = \begin{pmatrix}
        \lambda A_{1,1} && \cdots && \lambda A_{1,n} \\
        \vdots && && \vdots \\
        \lambda A_{m,1} && \cdots && \lambda A_{m,n} 
    \end{pmatrix} \]
\end{definition}

\begin{proposition}
    Suppose $T \in \LL(V,W)$ and $\lambda \in \mathbf{F}$ then $\lambda \mathcal{M}(T) = \mathcal{M}(\lambda T)$.
\end{proposition}

\begin{proof}
    Again, just use the definitions.
\end{proof}

\begin{theorem}
    Suppose $\mathbf{F}^{m,n}$ be the set of all the matrices with entries in $\mathbf{F}$. Then, with addition and scalar multiplication
    defined above $\mathbf{F}^{m,n}$ is a vector space and $\dim \mathbf{F}^{m,n} = mn$. 
\end{theorem}

\begin{proof}
    Proving it is a vector space is pretty easy. To verify $\dim \mathbf{F}^{m,n}=mn$ define
    \[ X_{i,j} = \begin{pmatrix}
        0 && \cdots && \cdots && \cdots  \\
        \vdots && \ddots && \vdots && \vdots \\
        \vdots && \vdots && 1 && \vdots \\
        \vdots && \vdots && \vdots && \ddots \\
    \end{pmatrix} \]    
    where every entry is $0$ expect the $A_{i,j}$ entry which is equal to $1$. Now, its easy to see that every $Z \in \mathbf{F}^{m,n}$
    can be written as some linear combination of $X_{i,j}$'s. Thus, $F^{m,n}= \operatorname{span}\{X_{i,j}\}$ where $i,j$ vary with 
    $i=1,\ldots,m$ and $j=1,\ldots,n$. Also, every matrix with only $0$ as its entry can only be written as linear combination of $X_{i,j}$
    with all of its scalars equal to $0$. Since, there are $mn$ entries the dimension of $\mathbf{F}^{m,n}$ is equal to $mn$. 
\end{proof}

\subsubsection{Matrix Multiplication}

\begin{definition}
    Suppose $A$ is a $m \times n$ matrix and $B$ is $n \times p$ matrix. Then $AB$ is defined as to be an $m \times p$ matrix whose 
    entry in row $j$ and column $k$ is given by
    \[ (AB)_{j,k} = \sum_{r=1}^{n} A_{j,r} B_{k,r} \]
\end{definition}

\begin{remark}
    The motivation for us to define the product like this comes from questioning, does $\mathcal{M}(ST)=\mathcal{M}(S) \mathcal{M}(T)?$
    Suppose $v_1, \cdots, v_n$ is a basis of $V$ and $w_1, \cdots, w_m$ is the basis of $W$. Suppose $u_1, \cdots, u_k$ is the basis of
    $U$ then consider the map $T : U \to V$ and $S : V \to W$. Suppose 
    $\mathcal{M}(S)=A$ and $\mathcal{T}=B$. Then
    \begin{align*}
        (ST)u_k &= S\left( \sum_{r=1}^{n} B_{r,k} v_r \right) \\
        &= \sum_{r=1}^{n} B_{r,k} Sv_r \\
        &= \sum_{r=1}^{n} B_{r,k} \sum_{j=1}^{m} A_{j,r} w_j \\
        &= \sum_{j=1}^{m} \left( \sum_{r=1}^{n} A_{j,r} B_{r,k}  \right) w_j
    \end{align*}
    That is how we define $M(ST)$ and that is why \textbf{Definition 1.13.} makes sense.
\end{remark}

\begin{proposition}
    Suppose $T \in \LL(U,V)$ and $S \in \LL(V,W)$, then 
    \[ \MM(ST)=\MM(S) \MM(T) \]
\end{proposition}

\begin{proof}
    It follows from our remark and how we defined the product of the matrix.
\end{proof}

\begin{definition}
    Suppose $A$ is a $m \times n$ matrix then
    \begin{enumerate}
        \item If $1 \le j \le m$ then $A_{j,\cdot}$ denotes the $1 \times n$ matrix consisting of row $j$ of $A$.
        \item If $1 \le j \le n$ then $A_{\cdot, j}$ denotes $m \times 1$ matrix consisting of column $j$ of $A$.
    \end{enumerate} 
\end{definition}

\textbf{Example :}

Suppose $A = \begin{pmatrix}
    1 && 2 && 3 \\
    4 && 5 && 6
\end{pmatrix}$ then 
\[
 A_{1, \cdot} = \begin{pmatrix}
    1 && 2 && 3
\end{pmatrix}
\] 

\[
A_{\cdot, 3} = \begin{pmatrix}
     3 \\ 6  
\end{pmatrix}
 \]


\begin{theorem}
    Suppose $A$ is a $m \times n$ matrix and $B$ is a $n \times p$ matrix. Then
    \[ (AB)_{j,k} = A_{j, \cdot} B_{\cdot,k} \]
    where $1 \le j \le m$ and $1 \le k \le p$.
\end{theorem}

\begin{proof}
    The definition of matrix multiplication states that
    \begin{align*}
    (AB)_{j,k} &= \sum_{r=1}^{n} A_{j,r} B_{r,k} \\
    &= A_{j,1} B_{1,k} + \cdots + A_{j,n} B_{n,k}
    \end{align*} 
    Now, if you take $A_{j,\cdot}$ and $B_{\cdot,k}$ and multiply it out you'll get the same thing. 
\end{proof}

\begin{theorem}
    Suppose $A$ is a $m \times n$ matrix and $B$ is a $n \times p$ matrix. Then
    \[ (AB)_{\cdot, k} = A B_{\cdot, k} \]
    for $1 \le k \le p$.
\end{theorem}

\begin{proof}
    Both of the matrix have size $m \times 1$. The $j$-th row of $(AB)_{\cdot,k}$ has the element $(AB)_{j,k}$ 
    and the $j$-th row of $A B_{\cdot,k}$ has element $A_{j,1} B_{1,1} + A_{j,2} B_{2,1} + \cdots A_{j,n} B_{n,1}$.
    Thus, from our previous theorem they're equal.   
\end{proof}

\begin{remark}
    The row version of this is 
    \[ (AB)_{k, \cdot} = A_{k,\cdot} B \]
\end{remark}

\begin{theorem}
    Suppose $A$ is a $m \times n$ matrix and $b = \begin{pmatrix}
        b_1 \\ \vdots \\ b_n
    \end{pmatrix}$ is a $n \times 1$ matrix. Then, 
    \[ Ab = b_1 A_{\cdot,1} + \cdots + b_n A_{\cdot, n} \] 
\end{theorem}

\begin{proof}
    They both have same size and the entries of of $Ab$ is the same as of the right side.
\end{proof}

\begin{theorem}
    Suppose $C$ is an $m \times c$ matrix and $R$ is a $c \times n$ matrix
    \begin{enumerate}
        \item The column $k$ of $CR$ is the linear combination of the columns of $C$, with coefficients of this linear combination coming
              from column $k$ of $R$.
        \item Then row $j$ of $CR$ is a linear combination of the rows of $R$, with the coefficients of this linear combination coming from
              row $j$ of $C$.
    \end{enumerate}
\end{theorem}

\begin{proof}
    Use \textbf{Theorem 1.7.} and \textbf{Theorem 1.8.} for $1.$ and we'll prove $2.$ in the exercise section.
\end{proof}

\subsubsection{Column-Row Factorization and Rank of a Matrix}

\begin{definition}
    Suppose $A$ is a $m \times n$ matrix with entries in $\mathbf{F}$.
    \begin{enumerate}
        \item The \textbf{column rank} of $A$ is the dimension of the span of columns of $A$ in $\mathbf{F}^{1,m}$. 
        \item The \textbf{row rank} of $A$ is the dimension of the span of rows of $A$ in $\mathbf{F}^{n,1}$.
    \end{enumerate}
\end{definition}

\textbf{Example :}
Suppose 
\[ A = \begin{pmatrix}
    1 && 2 && 3 \\
    4 && 5 && 6 
\end{pmatrix} \]
then the column rank is the dimension of 
\[ \operatorname{span}\left( \begin{pmatrix}
    1 \\ 4
\end{pmatrix} , \begin{pmatrix}
    2 \\ 5
\end{pmatrix}, \begin{pmatrix}
    3 \\ 6
\end{pmatrix} \right) \]

and the row rank is the dimension of
\[ \operatorname{span} \left(  \begin{pmatrix}
    1 && 2 && 3
\end{pmatrix} , \begin{pmatrix}
    4 && 5 && 6
\end{pmatrix}\right) \]

\begin{definition}
    The \textit{transpose} of a $m \times n$ matrix $A$, denoted by $A^t$, is the $n \times m$ matrix whose entries are given by
    \[ (A^t)_{i,j} = A_{j,i} \] 
\end{definition}

\begin{theorem}
    Suppose $A$ is an $m \times n$ matrix with entries in $\mathbf{F}$ and column rank $c \ge 1$. Then there exists a $m \times c$ matrix
    $C$ and $c \times n$ matrix $R$, both with entries in $F$, such that $A=CR$. 
\end{theorem}

\begin{proof}
    The list $A_{\cdot, 1}, \ldots, A_{\cdot, n}$ of columns of $A$ can be reduced to a basis of the span of the columns of $A$. This basis
    has length $c$ by definition of column rank. The $c$ columns can be put together to form $m \times c$.
    
    Now, each column $k$ of $A$ is a linear combination of columns of $C$. Make the coefficients of this linear combination column $k$ of 
    $R$. This matrix $R$ has size $c \times n$. Thus, $A = CR$ follows form \textbf{Theorem 1.9.}(a).
\end{proof}

\begin{theorem}
    Suppose $A \in \mathbf{F}^{m,n}$ then the column rank of $A$ equals row rank of $A$.
\end{theorem}

\begin{proof}
    Let $c$ be the column rank of $A$. Then $A=CR$ by the previous theorem where $C$ and $R$ are the matrix whose size are $m \times c$ and 
    $c \times n$ respectively. Now, from textbf{Theorem 1.9.} (b) each row of $A$ is a linear combination of rows of $R$. Since, $R$ has 
    $c$ columns this implies that 
    \[  \operatorname{row rank} A \le c = \operatorname{column rank} A \] 

    Now applying the same thing to $A^t$ we get
    \begin{align*}
        \operatorname{column rank } A &= \operatorname{row rank } A^t \\
        &\le \operatorname{column rank} A^t \\ 
        &= \operatorname{row rank} A 
    \end{align*}
    Thus, we're done.
\end{proof}

\begin{remark}
    From now on, we'll limit our use our terminology of ``row rank" and ``column rank" to just ``rank". 
\end{remark}

\eject
\subsubsection{Exercise}

\paragraph{Problem :} Suppose $T \in \LL(V,W)$. Show that with respect to each choice of basis of $V$ and $W$, the matrix of $T$ has 
at least $\dim \operatorname{range} T$ nonzero entries.  

\vspace{4mm}
\textit{Solution :} Let $v_1, \ldots, v_n$ be the the basis of $V$ and $w_1, \ldots, w_m$ be the basis of $W$ then suppose $k$ of those
vectors are $0$ under $T$ and let those vectors be $v_1, \ldots, v_k$. Thus, 
\[ \operatorname{range} T = \operatorname{span} \{Tv_{k+1}, \ldots, Tv_{n}\} \]
Thus, $\dim \operatorname{range} T \le n-k$. But since $T(v_j) \neq 0$ for each $k+1 \le j \le n$, there must be one entry thats not $0$
for each $T(v_j)$. Since, the number of $T(v_j) \neq 0$ are exactly $n-k$ and $n-k \ge \dim \operatorname{range} T$ this means there is 
at least $\dim \operatorname{range} T$ nonzero entries in matrix of $T$.

\paragraph{Problem :} Suppose $V$ and $W$ are finite-dimensional and $T \in \LL(V,W)$. Prove that $\dim \operatorname{range} T = 1$ if and 
only if there exist a basis of $V$ and a basis of $W$ such that with respect to these bases, all the entries of $\MM(T)$ is $1$.

\vspace{4mm}
\textit{Solution :} Let us first prove $(\Leftarrow)$. Suppose $A_{i,j}=1$ for all $i,j$. That means, $T(v_i) = \sum w_j$ where 
$v_1,\ldots,v_n$ is the basis of $V$ and $w_1,\ldots,w_m$ is the basis of $W$. Thus, $Tv_1=Tv_2=\cdots=Tv_n=k$ and
$\operatorname{range} T = \operatorname{span}\{Tv_1,\cdots,Tv_n\} = \operatorname{span}\{k\} \implies \dim \operatorname{range} T = 1$.

Now, for the $(\Rightarrow)$ we use the Fundamental theorem of linear maps,
\[ \dim V = \dim \operatorname{null} T + \dim \operatorname{range} T  \]
\[ \implies \dim V = \dim \operatorname{null} T + 1 \]

Now, suppose $v_2, \ldots, v_n$ be that basis of $\operatorname{null} T$. Then extend this basis to $V$, suppose $v_2,\ldots,v_n,v$ then
\[ T(v_2)=T(v_3)= \cdots = T(v_n) = 0 \]
\[ \implies T(v) = T(v_2+v)= \cdots = T(v_n+v) \]
One can check that $v_2+v,\ldots,v$ is a basis of $V$(as its linearly independent and has length $n$). Now, since 
$\dim \operatorname{range} T = 1$ we have $T(x)= \lambda T(v)$ and we choose $T(v),w_2,\ldots,w_m$ as our basis for $W$. Now,
we use a clever trick and set $w_1=T(v)-w_2-w_3- \cdots - w_m$ and notice that $w_1,\cdots,w_m$ is a basis of $W$.  Thus,
\[ T(v_i)=T(v)=\sum w_j \]
Thus, we're done.


\paragraph{Problem :}
Suppose that \(D\in\LL(\mathcal{P}_{3}({\bf R}),\mathcal{P}_{2}({\bf R}))\) is the differentiation map defined by \(Dp=p^{\prime}\). 
Find a basis of \(\mathcal{P}_{3}({\bf R})\) and a basis of \(\mathcal{P}_{2}({\bf R})\) such that the matrix of \(D\) with respect to these 
bases is
\[
\left(\begin{array}{cccc}1&0&0&0\\ 0&1&0&0\\ 0&0&1&0\end{array}\right)
\]

\vspace{4mm}
\textit{Solution :} Take the the basis of  $\mathcal{P}_3({\bf R})$ to be $z,\frac{z^2}{2},\frac{z^3}{3},1$ and $\mathcal{P}_2({{\bf R}})$ to be 
$1,z,z^2$.

\paragraph{Problem :}
Suppose \(V\) and \(W\) are finite-dimensional and \(T\in\LL(V,W)\). Prove that there exist a basis of \(V\) and a basis of \(W\) 
such that with respect to these bases, all entries of \(\mathcal{M}(T)\) are \(0\) except that the entries in row \(k\), column \(k\), equal 
\(1\) if \(1\leq k\leq\text{dim}\) range \(T\).

\vspace{4mm}
\textit{Solution :} Let $\dim V = n$ and $\dim \operatorname{range} T = m$. Now, let $v_{m+1}, \cdots, v_n$ be the basis of 
$\dim \operatorname{null} T$. Now, extend these basis such in the following way
\[ (v_1,\ldots,v_m,v_{m+1},\ldots,v_n) \]
Here, $T(v_i) \neq 0$ for $1 \le i \le m$. Now, since $(v_1,\cdots,v_m,v_{m+1},\cdots,v_n)$ spans $V$ the list $(Tv_1,,\ldots,Tv_m)$ 
must span $\operatorname{range} T$ and in fact it is the basis of $\operatorname{range} T$ (one can check that its linearly independent). 
We can now extend this basis to the basis of $W$. Suppose $(Tv_1,\ldots,Tv_m,w_1,\ldots,w_k)$ is the basis of $W$. Then,
\[ T(v_i) = 0 \cdot T(v_1) + \cdots + 1 \cdot T(v_i) + \cdots + 0 \cdot w_k \]
for $1 \le i \le m = \dim \operatorname{range} T$. But for $i > \dim \operatorname{range} T$ we have
\[ 0=T(v_i)= 0 \cdot T(v_1) + 0 \cdot T(v_2) + \cdots + 0 \cdot w_k \]


\paragraph{Problem :}
Suppose \(\sigma_{1},...,\sigma_{m}\) is a basis of \(V\) and \(W\) is finite-dimensional. Suppose \(T\in\LL(V,W)\). Prove that there 
exists a basis \(w_{1},...,w_{n}\) of \(W\) such that all entries in the first column of \(\mathcal{M}(T)\) [with respect to the bases 
\(\sigma_{1},...,\sigma_{m}\) and \(w_{1},...,w_{n}\)] are \(0\) except for possibly a \(1\) in the first row, first column.

\vspace{4mm}
\textit{Solution :} We know that $\operatorname{range}T =  \operatorname{span}\{T(\sigma_1),T(\sigma_2),\cdots,T(\sigma_m)\}$. 
Thus, we can make this span a basis. If $T(\sigma_1)=0$ then we're done but if not then the basis of $\operatorname{range} T$ would be

\[ (T(\sigma_1),z_2, \ldots, z_k) \]
Now, we can extend this basis to the basis of $W$, suppose its \[ (T(\sigma_1),z_2,\ldots,z_k,s_{k+1},\ldots,s_m) \] then
\[T(\sigma_1) = 1 \cdot T(\sigma_1) + 0 \cdot z_2 + \cdots + 0 \cdot s_m \]

\paragraph{Problem :}
Suppose \(w_{1},...,w_{n}\) is a basis of \(W\) and \(V\) is finite-dimensional. Suppose \(T\in\LL(V,W)\). Prove that there exists 
a basis \(\sigma_{1},...,\sigma_{m}\) of \(V\) such that all entries in the first row of \(\mathcal{M}(T)\) [with respect to the bases 
\(\sigma_{1},...,\sigma_{m}\) and \(w_{1},...,w_{n}\)] are \(0\) except for possibly a \(1\) in the first row, first column.

\vspace{4mm}
\textit{Solution :} Take any basis $v_1, \ldots, v_m$ of $V$. Then, suppose 
\begin{align*}
    T(v_i) &= \sum_{j=1}^{n} {}_{i} \lambda_{j} w_j \\
    &= {}_{i}\lambda_{1} w_1 + \sum_{j=2}^{n} {}_{i} \lambda_{j} w_j
\end{align*}

Now, if all the $T(v_i)$ has $0$ as the coefficient of $w_1$ then we're done. If not then take a $v_k$ for which ${}_{k}\lambda_1 \neq 0$
then swap it with $v_1$.
Then, define
\begin{align*}
    \sigma_1 &= \frac{v_1}{{}_{1}\lambda_1} \\
    \sigma_i &:= v_i-\frac{{}_{i}\lambda_1}{{}_{1}\lambda_1} v_1 \quad \text{for } i \ge 2  
\end{align*}

Now, one can check that $(\sigma_1, \ldots,\sigma_m)$ is a basis and 
\begin{align*}
    T(\sigma_i) &= T(v_i) - \frac{{}_{i}\lambda_1}{{}_{1}\lambda_1} T(v_k) \\
    &= 0 \cdot w_1 + \sum b_j w_j
\end{align*}



\paragraph{Problem :}
Give an example of \(2 \times 2\) matrices \(A\) and \(B\) such that \(AB \neq BA\).

\vspace{4mm}
\textit{Solution :} 
\[ A = \begin{pmatrix}
    1 && 2 \\
    3 && 4
\end{pmatrix} \quad \text{and} \quad B = \begin{pmatrix}
    5 && 6 \\
    7 && 8
\end{pmatrix} \]

\paragraph{Problem :}
Prove that the distributive property holds for matrix addition and matrix multiplication. In other words, suppose \(A, B, C, D, E, F\) are 
matrices whose sizes are such that \(A(B + C)\) and \((D + E)F\) make sense. Explain why \(AB + AC\) and \(DF + EF\) both make sense and 
prove that
\[
A(B + C) = AB + AC 
\quad \text{and} \quad
(D + E)F = DF + EF.
\]

\vspace{4mm}
\textit{Solution :} If $A(B+C)$ and $(D+E)F$ makes sense then $B$ and $C$ must be of the same size and $D$ and $E$ must be of the same size.
Also, the number of columns in $A$ must be same as the number of rows in $B$ and $C$. Let $A$ be a $n \times p$ matrix and $X=B+C$ then 
\begin{align*}
    (AX)_{i,j} &= \sum_{r=1}^{n} A_{i,r}X_{r,j} \\
    &= \sum_{r=1}^{n} A_{i,r} (B_{r,j}+C_{r,j}) \\
    &= \sum_{r=1}^{n} A_{i,r} B_{r,j} + \sum_{r=1}^{n} A_{i,r} C_{r,j} \\
    &= (AB)_{i,j} + (AC)_{i,j} 
\end{align*}
Thus, $A(B+C)=AB+AC$. Similar proof works for $(D+E)F$.


\paragraph{Problem :}
Prove that matrix multiplication is associative. In other words, suppose \(A, B, C\) are matrices whose sizes are such that \((AB)C\) makes 
sense. Explain why \((AB)C\) makes sense and prove that
\[
(AB)C = A(BC).
\]

\vspace{4mm}
\textit{Solution :} To make $(AB)C$ sense, we need $A$ to have same number of columns as the number of rows in $B$. Also, we need $B$ to 
have same number of columns as number of rows in $C$. To prove the associativity, you can just definition of matrix multiplication.

\paragraph{Problem :}
Suppose \(A\) is an \(n \times n\) matrix and \(1 \leq j, k \leq n\). Show that the entry in row \(j\), column \(k\), of \(A^3\) (which is 
defined to mean \(AAA\)) is
\[
\sum_{r=1}^n \sum_{i=1}^n A_{j,r} A_{r,i} A_{i,k}.
\]

\vspace{4mm}
\textit{Solution :} It follows directly from the definition of matrix multiplication.

\paragraph{Problem :}
Suppose \(m\) and \(n\) are positive integers. Prove that the function \(A \mapsto A^t\) is a linear map from 
\(\mathbf{F}^{m,n}\) to \(\mathbf{F}^{n,m}\).

\vspace{4mm}
\textit{Solution :}
Define \(T : \mathbf{F}^{m,n} \to \mathbf{F}^{n,m}\) by \(T(A) = A^t\).  
We must show that \(T\) is linear, i.e.
\[
T(A+B) = T(A) + T(B) \quad \text{and} \quad 
T(\lambda A) = \lambda T(A),
\]
for all \(A,B \in \mathbf{F}^{m,n}\) and all scalars \(\lambda \in \mathbf{F}\).

By definition of the transpose,
\[
(A^t)_{ij} = A_{ji}, \qquad (1 \leq i \leq n, \; 1 \leq j \leq m).
\]

Now let \(A,B \in \mathbf{F}^{m,n}\). Then for each \(i,j\),
\[
\big((A+B)^t\big)_{ij} = (A+B)_{ji} = A_{ji} + B_{ji} 
= (A^t)_{ij} + (B^t)_{ij} = \big(A^t + B^t\big)_{ij}.
\]
Hence \((A+B)^t = A^t + B^t\).

Similarly, for \(\lambda \in \mathbf{F}\),
\[
\big((\lambda A)^t\big)_{ij} = (\lambda A)_{ji} 
= \lambda A_{ji} = \lambda (A^t)_{ij} 
= \big(\lambda A^t\big)_{ij}.
\]
So \((\lambda A)^t = \lambda A^t\).

Therefore \(T\) preserves both addition and scalar multiplication.  
Thus \(T\) is a linear map from \(\mathbf{F}^{m,n}\) to \(\mathbf{F}^{n,m}\).

\paragraph{Problem :}
Prove that if \(A\) is an \(m \times n\) matrix and \(C\) is an \(n \times p\) matrix, then
\[
(AC)^t = C^t A^t.
\]

\vspace{4mm}
\textit{Solution :} 
\paragraph{Solution :}
Let \(A \in \mathbf{F}^{m,n}\) and \(C \in \mathbf{F}^{n,p}\).  
By definition of matrix multiplication, the \((i,j)\)-entry of \(AC\) is
\[
(AC)_{ij} = \sum_{k=1}^n A_{ik} C_{kj}, \qquad (1 \leq i \leq m, \; 1 \leq j \leq p).
\]
Taking the transpose, we get
\[
\big((AC)^t\big)_{ij} = (AC)_{ji} 
= \sum_{k=1}^n A_{jk} C_{ki}.
\]

On the other hand, consider the product \(C^t A^t\).  
Here \(C^t\) is \(p \times n\) and \(A^t\) is \(n \times m\), so \(C^t A^t\) is \(p \times m\).  
Its \((i,j)\)-entry is
\[
(C^t A^t)_{ij} = \sum_{k=1}^n (C^t)_{ik} (A^t)_{kj}.
\]
By definition of the transpose,
\[
(C^t)_{ik} = C_{ki}, \qquad (A^t)_{kj} = A_{jk}.
\]
Hence
\[
(C^t A^t)_{ij} = \sum_{k=1}^n C_{ki} A_{jk}.
\]

But scalar multiplication in \(\mathbf{F}\) is commutative, so
\[
\sum_{k=1}^n C_{ki} A_{jk} = \sum_{k=1}^n A_{jk} C_{ki}.
\]
Therefore,
\[
(C^t A^t)_{ij} = \big((AC)^t\big)_{ij}, \qquad (1 \leq i \leq p, \; 1 \leq j \leq m).
\]

Since all entries are equal, we conclude
\[
(AC)^t = C^t A^t. \quad 
\]

\paragraph{Problem :}
Suppose \(A\) is an \(m \times n\) matrix with \(A \neq 0\). Prove that the rank of \(A\) is \(1\) 
if and only if there exist \((c_{1}, \dots, c_{m}) \in \mathbf{F}^m\) and \((d_{1}, \dots, d_{n}) \in \mathbf{F}^n\) such that
\[
A_{j,k} = c_j d_k \quad \text{for every } j = 1, \dots, m \text{ and every } k = 1, \dots, n.
\]

\paragraph{Problem :}
Suppose \(T \in \mathcal{L}(V)\), and \(u_{1}, \dots, u_{n}\) and \(v_{1}, \dots, v_{n}\) are bases of \(V\). Prove that the following are
equivalent:
\begin{enumerate}[label=(\alph*)]
\item \(T\) is injective.
\item The columns of \(\mathcal{M}(T)\) are linearly independent in \(\mathbf{F}^{n,1}\).
\item The columns of \(\mathcal{M}(T)\) span \(\mathbf{F}^{n,1}\).
\item The rows of \(\mathcal{M}(T)\) span \(\mathbf{F}^{1,n}\).
\item The rows of \(\mathcal{M}(T)\) are linearly independent in \(\mathbf{F}^{1,n}\).
\end{enumerate}

Here \(\mathcal{M}(T)\) means \(\mathcal{M}(T, (u_{1}, \dots, u_{n}), (v_{1}, \dots, v_{n}))\).