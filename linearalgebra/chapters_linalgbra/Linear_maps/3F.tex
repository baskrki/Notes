\subsection{Duality}


\subsubsection{Dual Space and Dual Map}

\begin{definition}
    A \textit{linear functional} on $V$ is a linear map from $V$ to $\mathbf{F}$. In other words, a linear functional is an element of
    $\LL(V,\mathbf{F})$.
\end{definition}

\begin{definition}
    The \textit{dual space} of $V$, denoted $V'$, is the vector space of all linear functional on $V$. In other words, $V'=\LL(V,\mathbf{F})$. 
\end{definition}

\begin{proposition}
    Suppose $V$ is a finite-dimensional vector space. Then $V'$ is also finite-dimensional and 
    \[ \dim V' = \dim V \]
\end{proposition}

\begin{proof}
    From \textbf{Proposition 1.18.} we have
    \[ \dim V' = \dim \LL(V,\mathbf{F}) = (\dim V) (\dim \mathbf{F}) = \dim V \]
    as desired.
\end{proof}

\begin{definition}
    If $v_1, \ldots, v_n$ is a basis of $V$, then the \textit{dual space} of $v_1,\ldots,v_n$ is the list $\varphi_1 , \ldots, \varphi_n$ of 
    elements in $V'$ such that 
    \[ \varphi_j (k) = \begin{cases}
        1 & \text{if } k=j, \\
        0 & \text{if } k \neq j.
    \end{cases} \]
\end{definition}

\begin{proposition}
    Suppose $v_1, \ldots, v_n$ is a basis of $V$ and $\varphi_1, \ldots, \varphi_n$ is the dual basis. Then
    \[ v = \varphi_1(v) v_1 + \cdots + \varphi_n(v) v_n  \]
    for each $v \in V$.
\end{proposition}

\begin{proof}
    Let 
    \[ v= c_1 v_1 + \cdots + c_n v_n \]
    Then, $\varphi_n(v) = c_n$ thus 
    \[ v = \varphi_1(v) v_1 + \cdots + \varphi_n(v) v_n \]
    as desired.
\end{proof}

\begin{proposition}
    Suppose $V$ is a finite-dimensional. Then the dual basis of a basis of $V$ is a basis of $V'$. 
\end{proposition}

\begin{proof}
    Suppose $v_1, \ldots, v_n$ is a basis of $V$ and let $\varphi_1 , \ldots, \varphi_n$ be the dual basis. To show 
    $\varphi_1 , \ldots, \varphi_n$ is linearly independent, suppose there exists $a_1,\ldots,a_n \in \mathbf{F}$ such that
    \[ a_1 \varphi_1 + \cdots + a_n \varphi_n =0 \]
    Then, $(a_1 \varphi_1 + \cdots + a_n \varphi_n)(v_k) =a_k$ for $1 \le k \le n$. Thus, $a_1=a_2=\cdots=a_n=0$.
    And since the list is of the length $\dim V'$, we can conclude that the list is the basis of $V'$.
\end{proof}

\begin{definition}
    Suppose $T \in \LL(V,W)$. The \textit{dual map} of $T$ is the linear map $T' \in \LL(W',V')$ defined for each $\varphi \in W'$ by
    \[ T'(\varphi) = \varphi \circ T \]
\end{definition}

\begin{remark}
    Since $T'$ is a composition of linear maps $\varphi$ and $T$, it is a linear map as well. Also, $T'(\varphi) \in V'$ as $T'$ as it takes
    an element from $V$ to $\mathbf{F}$. Also, one can verify $T' \in \LL(W',V')$.
\end{remark}

\begin{proposition}
    Suppose $T \in \LL(V,W)$. Then
    \begin{enumerate}
        \item[(a)] $(S+T)' = S'+T'$ for all $S \in \LL(V,W)$, 
        \item[(b)] $(\lambda T)' = \lambda T'$ for all $\lambda \in \mathbf{F}$,
        \item[(c)] $(ST)'=T'S'$ for all $S \in \LL(W,U)$.
    \end{enumerate}    
\end{proposition}

\begin{proof}
    The proofs of (a) and (b) directly follow from the definitions.
    For (c),
    \[ (ST)'(\varphi) = \varphi \circ (ST) = (\varphi \circ S) \circ T = T'(\varphi \circ S) = T'(S'(\varphi)) = T'S' \]
    The fourth equation is due to $\varphi \circ S \in W'$.
\end{proof}

\subsubsection{Null Space and Range of Linear Map}

\begin{definition}
    For $U \subseteq V$, the annihilator of $U$, denoted by $U^0$, is defined by 
    \[ U^0 = \{ \varphi \in V' \mid \varphi(u)=0 \text{ for all } u \in U \} \]
\end{definition}

\begin{proposition}
    Suppose $U \subseteq V$. Then $U^0$ is a subspace of $V'$.
\end{proposition}

\begin{proof}
    One can just check the vector axioms.
\end{proof}

\begin{proposition}
    Suppose $V$ is finite-dimensional and $U$ is a subspace of $V$. Then
    \[ \dim U^0 = \dim V - \dim U \]
\end{proposition}

\begin{proof}
    Let $i \in \LL(U,V)$ be the linear map such that $i(u)=u$ for each $u \in U$. Thus, $i' \in \LL(V',U')$ and from fundamental theorem of
    linear maps we have,
    \[ \dim \operatorname{range} i' + \dim \operatorname{null} i' = \dim V' = \dim V \] 
    Also, $\operatorname{null} i' = \{ \varphi \in V' \mid i'(\varphi) = 0\} = \{ \varphi \in V' \mid \varphi \circ i = 0 \} =
    \{ \varphi \in V' \mid \varphi(x) = 0\}=U^0$. Thus, $\dim \operatorname{null} i' = \dim U^0$ and the equation above becomes
    \[ \dim \operatorname{range} i' + \dim U^0 = \dim V \]
    If $\varphi \in U'$, then $\varphi$ can be extended to a linear functional $\phi$ on $V$ (\textbf{Exercise 10} of the first section).
    Thus, $i'(\phi)=\varphi$ and $\operatorname{range} i' = U'$. Hence
    \[ \dim \operatorname{range} i' = \dim U' = \dim U \]
    And thus $\dim U + \dim U^0 = \dim V$ as desired.
\end{proof}

\begin{proposition}
    Suppose $V$ is finite-dimensional and $U$ is a subspace of $V$. Then
    \begin{enumerate}
        \item[(a)] $U^0=\{0\} \iff U=V$,
        \item[(b)] $U^0=V' \iff U=\{0\}$ 
    \end{enumerate}
\end{proposition}

\begin{proof}
    For (a) we have,
    \begin{align*}
        U^0 = \{0\} &\iff \dim U^0 = 0 \\
        &\iff \dim U = \dim V \\
        &\iff U=V
    \end{align*}
    Similarly, to prove (b) we have
    \begin{align*}
        U^0=V' &\iff \dim U^0 = \dim V' \\
        &\iff \dim U^0 = \dim V \\
        &\iff \dim U = 0 \\
        &\iff U=\{0\}
    \end{align*}
    And we're done.
\end{proof}

\begin{proposition}
    Let $V$ and $W$ be vector spaces and let $T \in \LL(V,W)$. Then
    \[ \operatorname{null} T' =(\operatorname{range} T)^0 \]
\end{proposition}

\begin{proof}
    First suppose $\varphi \in \operatorname{null} T'$, then $T'(\varphi)=(\varphi \circ T)(x) = 0$ for every $x \in V$. Since,
    $\varphi(Tx)=0$ we have $\varphi \in (\operatorname{range} T)^0$ and thus $\operatorname{null} T' \subseteq (\operatorname{range}T)^0$.

    Now, suppose $\varphi \in (\operatorname{range} T)^0$ then $\varphi(Tv)=0$ for all $v\in V$ which implies $T'(\varphi)=0$. Thus, 
    $\varphi \in \operatorname{null} T'$ and $(\operatorname{range} T)^0 \subseteq \operatorname{null} T'$ as desired.
\end{proof}

\begin{proposition}
    Suppose $V$ and $W$ are finite-dimensional and $T \in \LL(V,W)$ then 
    \[ \dim \operatorname{null} T' = \dim \operatorname{null} T + \dim W - \dim V \]
\end{proposition}

\begin{proof}
    We have 
    \begin{align*}
        \dim \operatorname{null} T' &= \dim (\operatorname{range}T)^0 \\
        &= \dim W - \dim \operatorname{range} T \\
        &= \dim W - (\dim V-\dim \operatorname{null} T) \\
        &= \dim \operatorname{null} T + \dim W - \dim V
    \end{align*}
\end{proof}

\begin{proposition}
    Suppose $V$ and $W$ are finite-dimensional and $T \in \LL(V,W)$. Then
    \[ T \text{ is surjective} \iff T' \text{ is injective} \]
\end{proposition}

\begin{proof}
    To prove this, we have
    \begin{align*}
        T \text{ is surjective} &\iff \operatorname{range} T = W \\
        &\iff (\operatorname{range}T)^0 = \{0\} \\
        &\iff \operatorname{null} T'=\{0\} \\
        &\iff T' \text{ is injective}
    \end{align*}
    as desired.
\end{proof}

\begin{proposition}
    Suppose $V$ and $W$ are finite-dimensional and $T \in \LL(V,W)$. Then
    \begin{enumerate}
        \item[(a)] $\dim \operatorname{range} T' = \dim \operatorname{range} T$,
        \item[(b)] $\operatorname{range} T' = (\operatorname{null} T)^0$  
    \end{enumerate}
\end{proposition}

\begin{proof}
    For (a) we have,
        \begin{align*}
            \dim \operatorname{range} T' &= \dim W' - \dim \operatorname{null} T' \\
            &= \dim W - (\dim \operatorname{null} T + \dim W - \dim V) \\
            &= \dim V- \dim \operatorname{null} T \\
            &= \dim \operatorname{range} T
        \end{align*}
        
    For (b), suppose $\varphi \in \operatorname{range} T'$ then there exists a $\phi \in W'$ such that $T'(\phi)=\varphi$. Thus, for 
    all $v \in \operatorname{null} T$ we have 
    \[ \varphi(v) = T'(\phi)v = (\phi \circ T)(v) = \phi(0)=0 \]
    Thus, $\varphi \in (\operatorname{null} T)^0$. Thus, $\operatorname{range} T' \subseteq (\operatorname{null}T)^0$.
    Now, we'll complete the proof by showing $\dim \operatorname{range} T' = \dim (\operatorname{null}T)^0$.
    Note, that 
    \begin{align*}
        \dim \operatorname{range} T' &= \dim \operatorname{range} T \\
        &= \dim V - \dim \operatorname{null} T \\
        &= \dim (\operatorname{null} T)^0
    \end{align*}
    where the last equation is from \textbf{Proposition 1.39}.
\end{proof}

\begin{proposition}
    Suppose $V$ and $W$ are finite-dimensional and $T \in \LL(V,W)$. Then,
    \[ T \text{ is injective} \iff T' \text{ is surjective} \]
\end{proposition}

\begin{proof}
    We have
    \begin{align*}
        T \text{ is injective} &\iff \operatorname{null} T= \{0\} \\
        &\iff (\operatorname{null} T)^0 = V' \\
        &\iff \operatorname{range} T'=V'
    \end{align*}
    as desired.
\end{proof}