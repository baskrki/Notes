\subsection{Invariant Subspaces}

\subsubsection{Eigenvalues}

\begin{definition}
    A linear map from vector space to itself is called an \textit{operator}.
\end{definition}

\begin{definition}
    Suppose $T \in \LL(V)$. A subspace $U$ of $V$ is called a \textit{invariant} under $T$ if $Tu \in U$ for every $u \in U$.
\end{definition}

From our definition, $U$ is invariant under $T$ if $T|_{U}$ is an operator on $U$. 

\begin{definition}
    Suppose $T \in \LL(V)$. A number $\lambda \in \FF$ is called a \textit{eigenvalue} of $T$  if there exists a $v \in V$ such that
    $v \neq 0$ and $Tv = \lambda v$.
\end{definition}

\begin{remark}
    Thus, $V$ has one-dimensional subspace invariant under $T$ if and only if $T$ has an eigenvalue. If $U$ is an one-dimensional subspace
    then $Tv = \lambda v$ for some $\lambda \in \FF$. Conversely, if $Tv = \lambda v$ for some $\lambda \in \FF$ then 
    $\operatorname{span}{v}$ is one-dimensional subspace $V$ invariant under $T$.
\end{remark}

\begin{proposition}
    Suppose $V$ is finite-dimensional, $T \in \LL(V)$, and $\lambda \in F$. Then the following are equivalent.
    \begin{enumerate}
        \item $\lambda$ is an eigenvalue of $T$.
        \item $T-\lambda I$ is not injective.
        \item $T-\lambda I$ is not surjective.
        \item $T-\lambda I$ is not invertible.
    \end{enumerate}
    where $I$ is the identity operator on $V$. 
\end{proposition}

\begin{proof}
    Condition $1.$ and $2.$ are equivalent because of $Tv = \lambda v \iff (T-\lambda I)v = 0$ and if it was injective then $v=0$.
    And $2.$, $3.$ and $4.$ are equivalent from $\textbf{Proposition 1.14.}$
\end{proof}

\begin{definition}
    Suppose $T \in \LL(V)$ and $\lambda \in \FF$ is an eigenvalue of $T$. A vector $v \in V$ is called and \textit{eigenvector}
    corresponding to $\lambda$ if $v \neq 0$ and $Tv = \lambda v$.
\end{definition}

\begin{proposition}
    Suppose $T \in \LL(V)$. Then every list of eigenvectors of $T$ corresponding to distinct eigenvalues of $T$ is linearly independent.
\end{proposition}

\begin{proof}
    For the sake of contradiction, suppose the result is false. Then there exists a smallest positive integer $m>1$ such that there exists a 
    list $v_1,\ldots,v_m$ of linearly dependent eigenvectors $\lambda_1,\ldots,\lambda_m$ of $T$, corresponding to distinct eigenvalues.
    Due to the minimality of $m$, there exists $a_1,\ldots,a_m \in \FF$ such that
    \[ a_1 v_1 + \ldots + a_m v_m =0 \] 
    Applying $T-\lambda I$ we get
    \[ a_1(\lambda_1 - \lambda_m)v_1 + \cdots + a_m(\lambda_{m-1}-\lambda_{m})v_{m-1}=0 \]
    Since, all the eigenvalues are different, none of the coefficient above is $0$. Thus, we get a new list of linearly dependent vector with
    length $m-1$, which contradicts the minimality of $m$. 
\end{proof}

\begin{proposition}
    Suppose $V$ is a finite-dimensional. Then each operator of $V$ has at most $\dim V$ distinct eigenvalue. 
\end{proposition}

\begin{proof}
    Since, every list of eigenvectors of $T$ corresponding to distinct eigenvalues is linearly independent by above proposition, we get the
    bound immediately.
\end{proof}

\subsubsection{Polynomials Applied to Operatos}

\begin{definition}
    Suppose $T \in \LL(V)$ and $m$ is a positive integer
    \begin{itemize}
        \item $T^m \in \LL(V)$ is defined by $T^m = \underbrace{T \cdots T}_{m}$ 
        \item $T^0$ is defined to be the identity operator $I$ on $V$.
        \item If $T$ is invertible with inverses $T^{-1}$, then $T^{-m} \in \LL(V)$ is defined by 
                \[ T^{-m} = (T^{-1})^m \]
    \end{itemize}
\end{definition}

\begin{definition}
    Suppose $T \in \LL(V)$ and $p \in \mathcal{P}(\FF)$ is a polynomial given by 
    \[ p(z) = a_0 + a_1 z + \cdots + a_m z^m \]
    for all $z \in \FF$. Then $p(T)$ is the operator on $V$ defined by 
    \[ p(T) = a_0 I + a_1 T + a_2 T^2 + \cdots + a_m T^m \]
\end{definition}

\begin{proposition}
    Let $T \in \LL(V)$, then the function $f : \mathcal{P}(\FF) \to \LL(V)$ given by $p \mapsto p(T)$ is linear.
\end{proposition}

\begin{proof}
    Left for future me as a exercise.
\end{proof}

\begin{proposition}
    Suppose $p,q \in \mathcal{P}(\FF)$ and $T \in \LL(V)$
    Then 
    \begin{enumerate}
        \item $(pq)(T) = p(T) q(T)$
        \item $p(T)q(T)=q(T)p(T)$
    \end{enumerate}
\end{proposition}

\begin{proof}
    Just define the polynomials and plug $T$.
\end{proof}

\begin{definition}
    Let $p,q \in \PP(\FF)$, then $pq \in \PP(\FF)$ is defined by
    \[ (pq)(z)=p(z) q(z) \]
\end{definition}

\begin{proposition}
    Suppose $T \in \LL(V)$ and $p \in \mathcal{P}(\FF)$. Then $\operatorname{null} p(T)$ and $\operatorname{range} p(T)$ are invariant 
    under $T$.
\end{proposition}

\begin{proof}
    Suppose $u \in \operatorname{null} p(T)$ then $p(T)u=0$
    \[ p(T)(Tu) = T(p(T)u) = T(0)=0 \]
    Hence, $Tu \in \operatorname{null} p(T)$. Thus, $\operatorname{null}p(T)$ is invariant under $T$ 

    Suppose, $u \in \operatorname{range} p(T)$. Then,
    \[ p(T)v=u  \] 
    \[ \implies T(p(T)v)=Tu \]
    \[ \implies p(T)(Tv)=Tu \]
    Hence, $Tu \in \operatorname{range} p(T)$. Thus, $\operatorname{range} p(T)$ is invariant under $T$, as desired.
\end{proof}\

\subsubsection{Exercise}

\paragraph{Problem :} Suppose $T \in \LL(V)$ and $U$ is a subspace of $V$.
\begin{enumerate}
    \item Prove that $U \subseteq \Null T$, then $U$ is invariant under $T$.
    \item Prove that $\range T \subseteq U$, then $U$ is invariant under $T$.
\end{enumerate}

\vspace{4mm}
\textit{Solution :} For $1.$ just notice that $Tu = 0 \in U$ for all $u \in U$. Thus, it is invariant under $T$.
For $2.$ notice that $Tu = k \in \range T \subseteq U$, thus $k \in U$. Therefore it is invariant under $T$.

\paragraph{Problem :} Suppose that $T \in \LL(V)$ and $V_1, \ldots, V_m$ are subspaces of $V$ invariant under $T$. Prove that 
$V_1 + \cdots + V_m$ is also invariant under $T$.

\vspace{4mm}
\textit{Solution :} Let $u \in V_1 + \cdots + V_m$ then $Tu = Tv_1 + \cdots + Tv_m$. Since $Tv_i \in V_i$ we have that 
\[ Tv_1 + \cdots + Tv_m \in V_1 + \cdots + V_m \]
Thus $V_1+\cdots+V_m$ is invariant under $T$.

\paragraph{Problem :} Suppose $T \in \LL(V)$. Prove that the intersection of every collection of subspaces of $V$ invariant under $T$ is 
invariant under $T$.

\vspace{4mm}
\textit{Solution :} Let $\dps \bigcap_{\alpha} V_{\alpha} = K$ where $V_{\alpha}$ is a subspace of $V$ invariant under $T$. 
If $v \in K$ then $Tv \in V_{\alpha}$ for every $\alpha$, hence $Tv \in K$.

\paragraph{Problem :} If $V$ is a finite-dimensional and $U$ is a subspace of $V$ that is invariant under every operator on $V$, then 
$U=\{0\}$ or $U=V$.

\vspace{4mm}
\textit{Solution :} Suppose $U \neq \{0\}$ and $U \neq V$. Let $u_1,\ldots,u_n$ be a basis of $U$ and extend it to basis of $V$, 
$u_1,\ldots,u_n,v_1,\ldots,v_{m-n}$. Define 
\[ T(a_1 u_1 + \cdots+a_{n}u_n+\cdots a_{m} v_{m-n})= a_1 v_1 \]
It is easy to see that $T \in \LL(V)$ but $U$ is not invariant under $T$ as $T(u_1)=v_1$.

\paragraph{Problem :} Suppose $T\in \LL(\RR^2)$ is defined by $T(x,y)=(-3y,x)$. Find the eigenvalues of $T$.

\vspace{4mm}
\textit{Solution :} If $\lambda$ is a eigenvalues then $(-3y,x)=T(x,y)=\lambda(x,y)=(\lambda x, \lambda y)$ then $-3y=\lambda^2 y$.
Notice that $y \neq 0$ because if $y=0$ then $x=0$ but $(x,y) \neq (0,0)$. Thus, $\lambda = \pm \sqrt{3}i$.

\paragraph{Problem :} Suppose $P \in \LL(V)$ is such that $P^2=P$. Prove that if $\lambda$ is an eigenvalue of $P$, then $\lambda = 0$ or 
$1$.

\vspace{4mm}
\textit{Solution :} If $\lambda$ is a eigenvalue then $\exists v \neq 0$ such that $P(v)=\lambda v$. Therefore,
\[ \lambda v = P(P(v))=P(\lambda v) = \lambda \cdot \lambda v \]
\[ \implies \lambda^2 v = \lambda v \implies v(\lambda^2 - \lambda) = 0 \]
\[ \implies \lambda^2 = \lambda \implies \lambda = 1,0 \]

\paragraph{Problem :} Suppose $T \in \LL(V)$. Suppose $S \in \LL(V)$ is invertible.
\begin{enumerate}
    \item Prove that $T$ and $S^{-1}TS$ have the same eigenvalues.
    \item What is the relationship between eigenvectors of $T$ and eigenvectors of $S^{-1}TS$?
\end{enumerate}

\vspace{4mm}
\textit{Solution :} If $\lambda$ is a eigenvalue of $T$ then $T(v)=\lambda v$. Since $S$ is invertible $\exists w \in V$ such that $S(w)=v$
thus $(S^{-1}(T(S(w)))=S^{-1}(\lambda v) = \lambda w$. If $\lambda$ is a eigenvalue of $S^{-1} T S$ then $S^{-1}(T(Sw))=\lambda w$ thus
$T(S(w))=S(\lambda w) = \lambda S(w)$. The relationship between the eigenvectors is that, eigenvectors of $S^{-1}TS$ maps to eigenvectors 
of $T$.

\paragraph{Problem :} Give an example of a operator in $\RR^4$ that has no eigenvalues.

\vspace{4mm}
\textit{Solution :} $T(w,x,y,z)=(-x,w,-z,y)$

\paragraph{Problem :} Suppose $V$ is finite-dimensional, $T \in \LL(V)$, and $\lambda \in \FF$. Show that $\lambda$ is an eigenvalue of $T$ 
if and only if $\lambda$ is an eigenvalue of the dual operator  $T' \in \LL(V')$.

\vspace{4mm}
\textit{Solution :} From exercise $17$ of $3F$ we know that $T$ is invertible if and only if $T'$ is invertible. Thus,
$T-\lambda I$ is not invertible if and only if $(T-\lambda I)=T'-\lambda I$ is not invertible. Thus we are done.

\paragraph{Problem :} Suppose $\FF = \RR$, $T \in \LL(V)$, and $\lambda \in \CC$. 
Prove that $\lambda$ is an eigenvalue of the complexification $T_{\CC}$ if and only if $\bar{\lambda}$ is an eigenvalue of $T_{\CC}$.

\vspace{4mm}
\textit{Solution :} If $\lambda$ is an eigenvalue then $T(v+iw)=\lambda(v+iw)$ thus we have 
\[ \ol{T(v+iw)}=T(\ol{v+iw})= \ol{\lambda}(\ol{v+iw}) \]
Since $v+iw \neq 0$ their conjugate isn't zero as well.