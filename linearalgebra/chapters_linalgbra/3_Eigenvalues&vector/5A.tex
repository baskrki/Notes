\subsection{Invariant Subspaces}

\subsubsection{Eigenvalues}

\begin{definition}
    A linear map from vector space to itself is called an \textit{operator}.
\end{definition}

\begin{definition}
    Suppose $T \in \LL(V)$. A subspace $U$ of $V$ is called a \textit{invariant} under $T$ if $Tu \in U$ for every $u \in U$.
\end{definition}

From our definition, $U$ is invariant under $T$ if $T|_{U}$ is an operator on $U$. 

\begin{definition}
    Suppose $T \in \LL(V)$. A number $\lambda \in \mathbf{F}$ is called a \textit{eigenvalue} of $T$  if there exists a $v \in V$ such that
    $v \neq 0$ and $Tv = \lambda v$.
\end{definition}

\begin{remark}
    Thus, $V$ has one-dimensional subspace invariant under $T$ if and only if $T$ has an eigenvalue. If $U$ is an one-dimensional subspace
    then $Tv = \lambda v$ for some $\lambda \in \mathbf{F}$. Conversely, if $Tv = \lambda v$ for some $\lambda \in \mathbf{F}$ then 
    $\operatorname{span}{v}$ is one-dimensional subspace $V$ invariant under $T$.
\end{remark}

\begin{proposition}
    Suppose $V$ is finite-dimensional, $T \in \LL(V)$, and $\lambda \in F$. Then the following are equivalent.
    \begin{enumerate}
        \item $\lambda$ is an eigenvalue of $T$.
        \item $T-\lambda I$ is not injective.
        \item $T-\lambda I$ is not surjective.
        \item $T-\lambda I$ is not invertible.
    \end{enumerate}
    where $I$ is the identity operator on $V$. 
\end{proposition}

\begin{proof}
    Condition $1.$ and $2.$ are equivalent because of $Tv = \lambda v \iff (T-\lambda I)v = 0$ and if it was injective then $v=0$.
    And $2.$, $3.$ and $4.$ are equivalent from $\textbf{Proposition 1.14.}$
\end{proof}

\begin{definition}
    Suppose $T \in \LL(V)$ and $\lambda \in \mathbf{F}$ is an eigenvalue of $T$. A vector $v \in V$ is called and \textit{eigenvector}
    corresponding to $\lambda$ if $v \neq 0$ and $Tv = \lambda v$.
\end{definition}

\begin{proposition}
    Suppose $T \in \LL(V)$. Then every list of eigenvectors of $T$ corresponding to distinct eigenvalues of $T$ is linearly independent.
\end{proposition}

\begin{proof}
    For the sake of contradiction, suppose the result is false. Then there exists a smallest positive integer $m>1$ such that there exists a 
    list $v_1,\ldots,v_m$ of linearly dependent eigenvectors $\lambda_1,\ldots,\lambda_m$ of $T$, corresponding to distinct eigenvalues.
    Due to the minimality of $m$, there exists $a_1,\ldots,a_m \in \mathbf{F}$ such that
    \[ a_1 v_1 + \ldots + a_m v_m =0 \] 
    Applying $T-\lambda I$ we get
    \[ a_1(\lambda_1 - \lambda_m)v_1 + \cdots + a_m(\lambda_{m-1}-\lambda_{m})v_{m-1}=0 \]
    Since, all the eigenvalues are different, none of the coefficient above is $0$. Thus, we get a new list of linearly dependent vector with
    length $m-1$, which contradicts the minimality of $m$. 
\end{proof}

\begin{proposition}
    Suppose $V$ is a finite-dimensional. Then each operator of $V$ has at most $\dim V$ distinct eigenvalue. 
\end{proposition}

\begin{proof}
    Since, every list of eigenvectors of $T$ corresponding to distinct eigenvalues is linearly independent by above proposition, we get the
    bound immediately.
\end{proof}

\subsubsection{Polynomials Applied to Operatos}

\begin{definition}
    Suppose $T \in \LL(V)$ and $m$ is a positive integer
    \begin{itemize}
        \item $T^m \in \LL(V)$ is defined by $T^m = \underbrace{T \cdots T}_{m}$ 
    \end{itemize}
\end{definition}

