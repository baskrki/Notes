\subsection{Invariant Subspaces}

\subsubsection{Eigenvalues}

\begin{definition}
    A linear map from vector space to itself is called an \textit{operator}.
\end{definition}

\begin{definition}
    Suppose $T \in \LL(V)$. A subspace $U$ of $V$ is called a \textit{invariant} under $T$ if $Tu \in U$ for every $u \in U$.
\end{definition}

From our definition, $U$ is invariant under $T$ if $T|_{U}$ is an operator on $U$. 

\begin{definition}
    Suppose $T \in \LL(V)$. A number $\lambda \in \mathbf{F}$ is called a \textit{eigenvalue} of $T$  if there exists a $v \in V$ such that
    $v \neq 0$ and $Tv = \lambda v$.
\end{definition}

