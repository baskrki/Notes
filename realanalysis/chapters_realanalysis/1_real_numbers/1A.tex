\subsection{Cuts}

\begin{theorem}
    No number $r$ in $\QQ$ has a square equal to $2$ i.e $\sqrt{2} \not \in \QQ$.     
\end{theorem}

\begin{proof}
    We can use the standard proof by letting $(p/q)^2 = 2$ with $p$ and $q$ not sharing a factor and deduce $p,q$ are even.
\end{proof}

\begin{definition}
    A \textbf{cut} in $\QQ$ is a pair of subsets of $A$ and $B$ of $\QQ$ such that
    \begin{enumerate}
        \item $A \cup B = \QQ$, $A \neq \emptyset$, $B \neq \emptyset$, $A \cap B = \emptyset$
        \item If $a \in A$ and $b \in B$ then $a < b$
        \item A contains no largest element 
    \end{enumerate}
    We denote the cut as $x = A|B$.
\end{definition}

\begin{example}
    Here are some examples of a cut.
    \begin{enumerate}
        \item[(i)] $A | B = \{r \in \QQ : r < 1\} | \{r \in \QQ : r \ge 1 \}$.
        \item[(ii)]  $A | B = \{r \in \QQ : r^2 < 2\} | \{r \in QQ : r^2 \ge 2\}$.
    \end{enumerate}
\end{example}

\begin{definition}
    A \textbf{real number} is a cut in $\QQ$.
\end{definition}


