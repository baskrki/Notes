\section{Problem Set 4}

\begin{problem}
    Give an example of a sequence $a_n$ that satisfies the following two conditions.
\begin{itemize}
    \item $a_n$ is divergent.
    \item For any $\varepsilon > 0$, there exists $N \in \mathbb{N}$ such that
    \[
        |a_{n+1} - a_n| < \varepsilon \quad \text{for all } n \ge N.
    \]
\end{itemize}
\end{problem}

\textit{Solution.} 

Take $\dps a_n=\sum_{i=1}^{n}\frac{1}{i}$.

\begin{problem}
     Let $p > 0$ be a positive number. Consider the $p$-series
\[
    \sum_{n=1}^{\infty} \frac{1}{n^p}.
\]
In the lecture we proved that the series diverges for $p = 1$ and converges for $p = 2$.
Show that the series converges for $p > 1$ and diverges for $0 < p \le 1$.
\end{problem}

\textit{Solution.}

For $0<p\le 1$ it is easy to see that $\dps \frac{1}{n^p} \ge \frac{1}{n}$. From the comparison test the series 
$\dps \sum_{n=1}^{\infty} \frac{1}{n^p}$ diverges. 

\vspace{6pt}
For $p>1$ we can use the same argument like we did for $p=2$. We can group the terms and use the comparison test.
I am skipping the details.

\begin{problem}
Let $r > 0$ be a positive number. Determine whether the series
\[
    \sum_{n=1}^{\infty} a_n
\]
converges or diverges for the following cases.
\begin{enumerate}
    \item $a_n = \sqrt{n+r} - \sqrt{n}$
    \item $a_n = n^3 r^n$
    \item $a_n = \dfrac{1}{n!} r^n$
\end{enumerate}
The answer may depend on the value of $r$.
\end{problem}

\textit{Solution.}

For $(1.)$, 
Let $r>0$. Consider the series
\[
\sum_{n=1}^{\infty} \bigl( \sqrt{n+r} - \sqrt{n} \bigr).
\]

Rationalizing the terms, we have
\[
\sqrt{n+r} - \sqrt{n} = \frac{r}{\sqrt{n+r} + \sqrt{n}}.
\]

Since $\sqrt{n+r} \le \sqrt{n} + \sqrt{r}$, it follows that
\[
\sqrt{n+r} + \sqrt{n} \le 2\sqrt{n} + \sqrt{r}.
\]

Hence,
\[
\sqrt{n+r} - \sqrt{n} = \frac{r}{\sqrt{n+r} + \sqrt{n}} \ge \frac{r}{2\sqrt{n} + \sqrt{r}}.
\]

Choose $N$ such that $2\sqrt{n} \ge \sqrt{r}$ for all $n \ge N$. Then for $n \ge N$,
\[
2\sqrt{n} + \sqrt{r} \le 3\sqrt{n},
\]
and therefore
\[
\sqrt{n+r} - \sqrt{n} \ge \frac{r}{3\sqrt{n}}.
\]

Since
\[
\sum_{n=1}^{\infty} \frac{1}{\sqrt{n}}
\]
diverges, the comparison test implies that
\[
\sum_{n=1}^{\infty} \bigl( \sqrt{n+r} - \sqrt{n} \bigr)
\]
also diverges.

\vspace{6pt}
For $(2.)$ we have and $(3.)$ just do the ratio test.

\begin{problem}
    Let $b_n$ be a sequence of non-negative numbers which decreases to zero. That is,
\[
    b_1 \ge b_2 \ge b_3 \ge \cdots \ge 0
    \quad \text{and} \quad
    \lim_{n \to \infty} b_n = 0.
\]

Let $a_n = (-1)^{n-1} b_n$. The purpose of this problem is to show that
\[
    \sum_{n=1}^{\infty} a_n
\]
converges. This is called the \emph{alternating series test}. Let
\[
    s_n = \sum_{k=1}^{n} a_k.
\]

\begin{enumerate}
    \item Show that $s_{2k+1}$ is decreasing and that $s_{2k}$ is increasing.
    \item Show that $s_{2k+1}$ is bounded from below and that $s_{2k}$ is bounded from above.
    \item Show that both $\lim_{k \to \infty} s_{2k+1}$ and $\lim_{k \to \infty} s_{2k}$ exist and are identical.
    \item Show that $\displaystyle \sum_{n=1}^{\infty} a_n$ converges.
\end{enumerate}
\end{problem}

\textit{Solution.}

For the first part notice that
\[ a_{2k}+a_{2k+1} = b_{2k+1}-b_{2k} \le 0 \]
Thus $s_{2k+1}=s_{2k-1}+a_{2k}+a_{2k+1} \le s_{2k-1}$.

\vspace{6pt}
Similarly, 
\[ a_{2k+1}+a_{2k+2} = b_{2k+1}-b_{2k+2} \ge 0 \]
Thus, $s_{2k+2} = a_{2k+1}+a_{2k+2} + s_{2k} \ge s_{2k}$.

\vspace{6pt}
To show that $\dps \{s_{2k+1}\}_{k=0}^{\infty}$ is bounded from below, we will show that $s_{2k+1} \ge 0$.
\begin{align*}
    s_{2k+1}&=\underbrace{b_1-b_2}_{\ge0}+\underbrace{b_3-b_4}_{\ge0}+\cdots+\underbrace{b_{2k-1}-b_{2k}}_{\ge0}+b_{2k+1} \\
    &\ge b_{2k+1} \\
    &\ge 0
\end{align*}

Also,
\begin{align*}
    s_{2k} &= b_1-b_2+b_3-b_4+\cdots+b_{2k-1}-b_{2k} \\
    &\le b_1-b_2+b_2-b_3+\cdots+b_{2k-2}-b_{2k-1} \\
    &\le b_1-b_{2k-1} \\
    &\le b_1
\end{align*}
Thus we're done with $(2.)$.

\vspace{6pt}

For $(3.)$, by \textbf{monotone convergence theorem},
\[ \lim_{n \to \infty} s_{2k+1} = \inf\{s_{2k+1}\} \quad \text{ and } \quad \lim_{n \to \infty} s_{2k} = \sup\{s_{2k}\} \]
To show they are identical, 
\begin{align*}
    &s_{2k+1} =s_{2k} + a_{2k+1} = s_{2k}+b_{2k-1} \\
    &\implies \lim_{n \to \infty} s_{2k+1} = \lim_{n \to \infty} s_{2k} = L
\end{align*}

For the $(4.)$ part,
\[ |s_n-L| < \eps \]
for large enough even and odd $n$.

\begin{problem}
    For any real number $x$, $\floor{x}$ be the largest integer less than or equal to $x$. Define a sequence
    \[ a_n = \sqrt{2}n - \lfloor \sqrt{2}n \rfloor \]
    \begin{enumerate}
        \item Show that $a_n$ has a convergent subsequence.
        \item Let $N \in N$ be an integer. Suppose $0 < a_m < 1/N$ for some integer $m$. Show that there 
              exists a integer $m$ such that 
              \[ a_n > 1-1/N \] 
    \end{enumerate}
\end{problem}