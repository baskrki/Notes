\section{Problem Set 7}

\begin{problem}
Consider the set
\[
X := \left\{ (a_1,a_2,a_3,\ldots) \;\middle|\; \sum_{j=1}^{\infty} a_j^2 \text{ converges} \right\}.
\]

Define the function $d : X \times X \to \RR$ as follows. For
\[
x = (a_1,a_2,a_3,\ldots), \quad y = (b_1,b_2,b_3,\ldots) \in X,
\]
define
\[
d(x,y) := \sqrt{\sum_{j=1}^{\infty} (a_j - b_j)^2 }.
\]

\begin{enumerate}
\item Show that the function $d$ is well-defined. Equivalently, show that for
\[
x = (a_1,a_2,a_3,\ldots) \quad \text{and} \quad y = (b_1,b_2,b_3,\ldots) \in X,
\]
the series
\[
\sum_{j=1}^{\infty} (a_j - b_j)^2
\]
converges.

\item Show that the function $d$ satisfies the triangle inequality. You may use the following triangle inequality in $\mathbb{R}^n$ without proof. For all $n \ge 1$,
\[
\sqrt{\sum_{j=1}^{n} (a_j - c_j)^2}
\le
\sqrt{\sum_{j=1}^{n} (a_j - b_j)^2}
+
\sqrt{\sum_{j=1}^{n} (b_j - c_j)^2}.
\]

\item Consider a sequence in $X$ defined as
\[
x_1 = (1,0,0,0,\ldots), \quad
x_2 = (0,1,0,0,\ldots), \quad \text{and so on}.
\]
In general,
\[
x_n = (0,\ldots,0,\underbrace{1}_{n\text{th position}},0,\ldots).
\]
Show that $(x_n)$ has no convergent subsequence.
\end{enumerate}
\end{problem}

\textit{Solution.}

For $(1.)$, Let $\dps S^a_N = \sum_{j=1}^{N} a_j^2$ and $\dps S^b_N = \sum_{j=1}^{N} b_j^2$. We know that $\dps \lim_{n \to \infty} S^a_n$
and $\dps \lim_{n \to \infty} S^b_n$ exists, therefore $\dps \lim_{n \to \infty} S^a_n + S^b_n$ exists. And since,

\[  0 \le (a_j - b_j)^2 = a_j^2+b_j^2- 2a_jb_j \le a_j^2+b_j^2+ 2|a_jb_j| \le 2(a_j^2+b_j^2)  \]
The last inequality is due to AM-GM inequality. Therefore,
\[ 0 \le (a_j-b_j)^2 \le 2(a_j^2+b_j^2)  \]
and we have that $\dps \sum_{j=1}^{\infty} (a_j - b_j)^2$ converges from comparison test.

\bigskip
For $(2.)$, just takes the limit as $n \to \infty$.

\bigskip

For $(3.)$, suppose there is a convergent subsequence and let $x_{n_k} \to x$
\[ d(x_{n_k},x_{n_{\ell}}) \le d(x_{n_k},x)+d(x,x_{n_{\ell}}) \]
Now, notice that for $n_k \neq n_\ell$ we have $d(x_{n_k},x_{n_\ell})=\sqrt{2}$.
Thus, for large $n_k$ and $m_{\ell}$ we have
\[ d(x_{n_k},x_{n_\ell}) \le d(x_{n_k},x)+d(x,x_{n_\ell}) < \frac{\eps}{2} + \frac{\eps}{2} \]
\[ \implies \sqrt{2} < \eps \]
But for $\eps < \sqrt{2}$ thats false.

\begin{problem}
Let $(X,d)$ be a metric space and $(x_n)$ a sequence in $X$. Denote
\[
E = \{x_1, x_2, x_3, \ldots\}.
\]
Suppose $(x_n)$ has no convergent subsequence. Show that for all $k \in \mathbb{N}$, there exists $r_k > 0$ such that
\[
B(x_k, r_k) \cap E = \{x_k\}.
\]

You may use the following fact without proof. Fix $x \in X$.
Suppose that for all $r > 0$, there are infinitely many elements in $E \cap B(x,r)$. Then $(x_n)$ has a subsequence which converges to $x$.
\end{problem}

\textit{Solution.} Using the fact, since no sequence of $(x_n)$ converges to $x$ there exists a $r>0$ such that $E \cap B(x,r)$ is finite.
Since no subsequence converges to $x_k$ for $k \in \NN$, we have $r>0$ s.t $E \cap B(x_k,r)$ is finite. If $\abs{E \cap B(x_k,r)}=1$ then 
we have the desired condition but suppose $\abs{E \cap B(x_k,r)}>1$ then
\[ E \cap B(x_k,r)= \{x_k,x_{m_1},\ldots,x_{m_\ell}\} \]
Then, since $d(x_k,x_{m_i})>0$ for all $1 \le i \le \ell$ we have
\[ E \cap B\left(x_k,\min_{1 \le i \le \ell}\{d(x_k,x_{m_i})\}\right)=\{x_k\} \]

\begin{problem}
Let $f:\RR \to \RR$ be a continuous function. Suppose that $f(x)$ is
differentiable on $(-\infty,0)\cup(0,\infty)$ and that the limit $\dps\lim_{x\to 0} f'(x)$
exists. Show that $f(x)$ is differentiable at $x=0$ and that $f'(0)=\lim_{x\to 0} f'(x).$
\end{problem}

\textit{Solution.} 

Let $h\neq 0$. Since $f$ is continuous on the closed interval with endpoints $0$ and $h$
and differentiable on the open interval with end points $0$ and $h$,we can use MVT.
Thus, there exists a point $c_h$ between $0$ and $h$ such that
\[
\frac{f(h)-f(0)}{h} = f'(c_h).
\]

As $h\to 0$, the point $c_h$ lies between $0$ and $h$, and therefore $c_h\to 0$.
Since $\lim_{x\to 0} f'(x) = L$, it follows that
\[
f'(c_h) \to L \quad \text{as } h\to 0.
\]

Hence,
\[
\lim_{h\to 0} \frac{f(h)-f(0)}{h}
= \lim_{h\to 0} f'(c_h)
= L.
\]

Therefore, the derivative of $f$ at $0$ exists and satisfies
\[
f'(0)=L=\lim_{x\to 0} f'(x).
\]

\begin{problem}
    Use $\frac{d}{dx}e^x=e^x$ to show that $e^x \ge 1+x$ for all $x \in \RR$.
\end{problem}

\textit{Solution.}

Notice that $f(x)=e^x$ is continuous everywhere and differentiable everywhere. For $x=0$ it is obvious $e^0 =1$.
Assume that $x>0$ then $f$ is continuous on $[0,x]$ and differentiable on $(0,x)$ thus applying MVT 
we get,
\[ \frac{f(x)-f(0)}{x} = f(c) \]
\[ \implies \frac{e^x-1}{x}=e^c \]
some for $0 < c < x$, we have $e^c \ge 1$ we have
\begin{align*}
    e^x &=e^c \cdot x + 1 \\
    & \ge x + 1 
\end{align*}
Now assume $x < 0$, we have $x=-a$ thus using MVT again we get,
\[ \frac{e^{-a}-1}{-a} = e^{c} \]
Since $0 > c > -a$ we get $\dps e^{c} \le 1 \implies e^{-a} = e^c (-a)+1 \ge -a+1$.