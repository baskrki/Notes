\section{Problem Set 8}

\begin{problem}
    Let $f: \RR \to \RR$ be a function. Suppose both $f'(x)$ and $f''(x)$ are continuous on $\RR$ and that $f(0)=0$. Define the function
    \[ g(x) = \begin{cases}
        f(x)/x & x \neq 0 \\
        f'(0) & x=0
    \end{cases} \]
    Show that $g'(x)$ exists for all $x \in \RR$ and express $g'(x)$ in terms of $f(x)$ and its derivative.
\end{problem}

\textit{Solution.} 

For $x \neq 0$, we have that 
\[ g'(x) = \frac{xf'(x)-f(x)}{x^2} \]

For $x=0$, we can use the definition of the derivative,

\[ g'(0) = \lim_{x \to 0} \frac{g(x)-g(0)}{x-0} = \lim_{x \to 0} \frac{f(x)-xf'(0)}{x^2} \]

Now using taylor's formula near $0$, we have that 
\[ f(x)=f(0)+xf'(0)+x^2f''(c_x) \cdot \frac12 \]
\[ \implies \frac{f(x)-xf'(0)}{x^2} = \frac{1}{2} \cdot f''(c_x) \]
\[ \implies \lim_{x \to 0} \frac{f(x)-xf'(0)}{x^2} = \lim_{x \to 0}\frac{f''(c_x)}{2} = \frac{f''(0)}{2} \]

Thus, $g'(0)$ also exists and 
\[ g'(x)=\begin{cases}
\dps \frac{xf'(x)-f(x)}{x^2} & x \neq 0 \\
\dps \frac{f''(0)}{2} & x = 0
\end{cases} \]

\begin{problem}
    Suppose there exist two functions $S : \RR \to \RR$ and $C : \RR \to \RR$ which satisfy the following properties:
\begin{itemize}
    \item $\frac{d}{dx}S(x) = C(x), \quad \frac{d}{dx}C(x) = S(x)$.
    \item $S(0) = 0, \quad C(0) = 1$.
\end{itemize}

\noindent (1) Let $S^{(n)}(x)$ be the $n$th derivative of $S(x)$. Show that for $k \in \mathbb{N} \cup \{0\}$,
\[
S^{(2k)}(x) = S(x), \quad S^{(2k+1)}(x) = C(x).
\]

\noindent (2) Show that for all $x \in \mathbb{R}$,
\[
S(x) = \sum_{n=0}^{\infty} \frac{x^{2n+1}}{(2n+1)!}.
\]
\end{problem}

\textit{Solution.}

For $(1.)$ just use induction.

\bigskip

For $(2.)$, restrict $S$ to $[0,x]$ for some fixed $x>0$. Then, 
\[ S(x)=\sum_{n=0}^{k} \frac{S^{n}(0)x^n}{(n)!} + R_k(x)= \sum_{n=0}^{\floor{k/2}} \frac{x^{2n+1}}{(2n+1)!}+R_k(x) \]

But since $S$ and $C$ are bounded and continuous on $[0,x]$ and we have 
$S^n(x) = S(x)$ or $C(x)$, thus $|S^n(x)| \le M$ for $n \ge 0$.
\[ |R_k| = \abs{\frac{S^{k+1}(c)x^{k+1}}{(k+1)!}} \le \abs{\frac{M x^{k+1}}{(k+1)!}} < \eps   \] 

Thus,
\[ S(x)=\sum_{n=0}^{\infty} \frac{x^{2n+1}}{(2n+1)!} \]

\begin{problem}
Let $a < b$ be two real numbers and let $f : [a,b] \to \mathbb{R}$ be a function.
Suppose $f(x)$ is continuous on $[a,b]$ and differentiable on $(a,b)$.
Suppose $f'(x) > 0$ for all $x \in (a,b)$.

\begin{enumerate}
\item Show that $f(x)$ is strictly increasing on $[a,b]$. That is,
$f(x_1) < f(x_2)$ for all $x_1 < x_2$ in $[a,b]$.

\item Show that for all $y \in (f(a), f(b))$, there exists a unique
$x \in (a,b)$ such that $f(x) = y$.

\item Let the function $g : (f(a), f(b)) \to (a,b)$ be the inverse function of $f(x)$.
In other words, $g(y) = x$ if $f(x) = y$. Show that $g$ is continuous on
$(f(a), f(b))$.

\item Show that $g$ is differentiable on $(f(a), f(b))$ and that
\[
g'(y) = \frac{1}{f'(g(y))}
\quad \text{for all } y \in (f(a), f(b)).
\]
\end{enumerate}
\end{problem}

\textit{Solution.}

For $(1.)$, we restrict the function to $[x_1,x_2]$. Then using MVT we have
\[ \frac{f(x_2)-f(x_1)}{x_2-x_1}=f'(c) \]
for some $c \in (x_1,x_2)$. Thus,
\[ f(x_2)-f(x_1) = f'(c)(x_2-x_1) > 0 \]

\bigskip
For $(2.)$, suppose $f(a)=f(b)$. If $a>b$ then $f(a)>f(b)$ from $(1.)$ also if $a<b$ then $f(a)<f(b)$. Thus, $a=b$.
\bigskip

For $(3.)$,

\begin{problem}
Define the function $f : [-1,1] \to \mathbb{R}$ by
\[
f(x) =
\begin{cases}
1, & x \in [-1,0) \cup (0,1], \\
0, & x = 0.
\end{cases}
\]

Show that $f(x)$ is Riemann integrable on $[-1,1]$ and that
\[
\int_{-1}^{1} f(x)\,dx = 2.
\]
\end{problem}