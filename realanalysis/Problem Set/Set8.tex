\section{Problem Set 8}

\begin{problem}
    Let $f: \RR \to \RR$ be a function. Suppose both $f'(x)$ and $f''(x)$ are continuous on $\RR$ and that $f(0)=0$. Define the function
    \[ g(x) = \begin{cases}
        f(x)/x & x \neq 0 \\
        f'(0) & x=0
    \end{cases} \]
    Show that $g'(x)$ exists for all $x \in \RR$ and express $g'(x)$ in terms of $f(x)$ and its derivative.
\end{problem}

\textit{Solution.} 

For $x \neq 0$, we have that 
\[ g'(x) = \frac{xf'(x)-f(x)}{x^2} \]

For $x=0$, we can use the definition of the derivative,

\[ g'(0) = \lim_{x \to 0} \frac{g(x)-g(0)}{x-0} = \lim_{x \to 0} \frac{f(x)-xf'(0)}{x^2} \]

Now using taylor's formula near $0$, we have that 
\[ f(x)=f(0)+xf'(0)+x^2f''(c_x) \cdot \frac12 \]
\[ \implies \frac{f(x)-xf'(0)}{x^2} = \frac{1}{2} \cdot f''(c_x) \]
\[ \implies \lim_{x \to 0} \frac{f(x)-xf'(0)}{x^2} = \lim_{x \to 0}\frac{f''(c_x)}{2} = \frac{f''(0)}{2} \]

Thus, $g'(0)$ also exists and 
\[ g'(x)=\begin{cases}
\dps \frac{xf'(x)-f(x)}{x^2} & x \neq 0 \\
\dps \frac{f''(0)}{2} & x = 0
\end{cases} \]

\begin{problem}
    Suppose there exist two functions $S : \RR \to \RR$ and $C : \RR \to \RR$ which satisfy the following properties:
\begin{itemize}
    \item $\frac{d}{dx}S(x) = C(x), \quad \frac{d}{dx}C(x) = S(x)$.
    \item $S(0) = 0, \quad C(0) = 1$.
\end{itemize}

\noindent (1) Let $S^{(n)}(x)$ be the $n$th derivative of $S(x)$. Show that for $k \in \mathbb{N} \cup \{0\}$,
\[
S^{(2k)}(x) = S(x), \quad S^{(2k+1)}(x) = C(x).
\]

\noindent (2) Show that for all $x \in \RR$,
\[
S(x) = \sum_{n=0}^{\infty} \frac{x^{2n+1}}{(2n+1)!}.
\]
\end{problem}

\textit{Solution.}

For $(1.)$ just use induction.

\bigskip

For $(2.)$, restrict $S$ to $[0,x]$ for some fixed $x>0$. Then, 
\[ S(x)=\sum_{n=0}^{k} \frac{S^{n}(0)x^n}{(n)!} + R_k(x)= \sum_{n=0}^{\floor{k/2}} \frac{x^{2n+1}}{(2n+1)!}+R_k(x) \]

But since $S$ and $C$ are bounded and continuous on $[0,x]$ and we have 
$S^n(x) = S(x)$ or $C(x)$, thus $|S^n(x)| \le M$ for $n \ge 0$.
\[ |R_k| = \abs{\frac{S^{k+1}(c)x^{k+1}}{(k+1)!}} \le \abs{\frac{M x^{k+1}}{(k+1)!}} < \eps   \] 

Thus,
\[ S(x)=\sum_{n=0}^{\infty} \frac{x^{2n+1}}{(2n+1)!} \]

\begin{problem}
Let $a < b$ be two real numbers and let $f : [a,b] \to \RR$ be a function.
Suppose $f(x)$ is continuous on $[a,b]$ and differentiable on $(a,b)$.
Suppose $f'(x) > 0$ for all $x \in (a,b)$.

\begin{enumerate}
\item Show that $f(x)$ is strictly increasing on $[a,b]$. That is,
$f(x_1) < f(x_2)$ for all $x_1 < x_2$ in $[a,b]$.

\item Show that for all $y \in (f(a), f(b))$, there exists a unique
$x \in (a,b)$ such that $f(x) = y$.

\item Let the function $g : (f(a), f(b)) \to (a,b)$ be the inverse function of $f(x)$.
In other words, $g(y) = x$ if $f(x) = y$. Show that $g$ is continuous on
$(f(a), f(b))$.

\item Show that $g$ is differentiable on $(f(a), f(b))$ and that
\[
g'(y) = \frac{1}{f'(g(y))}
\quad \text{for all } y \in (f(a), f(b)).
\]
\end{enumerate}
\end{problem}

\textit{Solution.}

For $(1.)$, Let $x_1<x_2$ be points in $[a,b]$. By the Mean Value Theorem, there exists
$c\in(x_1,x_2)$ such that
\[
\frac{f(x_2)-f(x_1)}{x_2-x_1}=f'(c).
\]
Since $f'(c)>0$ and $x_2-x_1>0$, it follows that
\[
f(x_2)-f(x_1)>0,
\]
and hence $f(x_1)<f(x_2)$. Therefore, $f$ is strictly increasing on $[a,b]$.

\bigskip

For $(2.)$, since $f$ is continuous on $[a,b]$, the IVT implies that
for every $y\in(f(a),f(b))$ there exists $x\in(a,b)$ such that $f(x)=y$. To prove uniqueness, suppose $x_1<x_2$ and $f(x_1)=f(x_2)$. 
This is a contradiction as $f(x_1)<f(x_2)$.


\bigskip
For $(3.)$, fix $y_0\in(f(a),f(b))$ and let $x_0=g(y_0)$. Let $\eps >0$ be given.
Since $f$ is strictly increasing, we have
\[
f(x_0-\varepsilon) < f(x_0) < f(x_0+\varepsilon).
\]
Define
\[
\delta
:=
\min\!\left\{
f(x_0+\varepsilon)-f(x_0),\;
f(x_0)-f(x_0-\varepsilon)
\right\}.
\]
Then $\delta>0$.

Now suppose $y\in(f(a),f(b))$ satisfies $|y-y_0|<\delta$.
Then
\[
f(x_0-\varepsilon) < y < f(x_0+\varepsilon).
\]
By the definition of the inverse function and since $f$ is increasing , we have
\[
x_0-\varepsilon < g(y) < x_0+\varepsilon.
\]
Hence,
\[
|g(y)-g(y_0)|<\varepsilon.
\]
Thus $g$ is continuous on $(f(a),f(b))$.

\bigskip

For $(4.)$, fix $y_0\in(f(a),f(b))$ and let $x_0=g(y_0)$, so that $f(x_0)=y_0$. For $y\neq y_0$,
write $y=f(x)$, so that $x=g(y)$. Then
\[
\frac{g(y)-g(y_0)}{y-y_0}
=
\frac{x-x_0}{f(x)-f(x_0)}
=
\frac{1}{\dfrac{f(x)-f(x_0)}{x-x_0}}.
\]
Taking the limit as $y\to y_0$ is equivalent to taking the limit as $x\to x_0$,
since $g$ is continuous. Therefore,
\[
\lim_{y\to y_0}\frac{g(y)-g(y_0)}{y-y_0}
=
\frac{1}{\lim_{x\to x_0}\dfrac{f(x)-f(x_0)}{x-x_0}}
=
\frac{1}{f'(x_0)}.
\]
Thus $g$ is differentiable at $y_0$ and
\[
g'(y_0)=\frac{1}{f'(x_0)}=\frac{1}{f'(g(y_0))}.
\]
Since $y_0$ was arbitrary, this holds for all $y\in(f(a),f(b))$.

\begin{problem}
Define the function $f : [-1,1] \to \RR$ by
\[
f(x) =
\begin{cases}
1, & x \in [-1,0) \cup (0,1], \\
0, & x = 0.
\end{cases}
\]

Show that $f(x)$ is Riemann integrable on $[-1,1]$ and that
\[
\int_{-1}^{1} f(x)\,dx = 2.
\]
\end{problem}

\textit{Solution.}

For any partition $P = \{x_0, x_1, \dots, x_n\}$ of $[-1, 1]$, every sub interval $[x_{i-1}, x_i]$ contains points $x \neq 0$.
 Thus, the supremum on each subinterval is $M_i = 1$. The upper sum is:
\[
U(f, P) = \sum_{i=1}^{n} M_i \Delta x_i = \sum_{i=1}^{n} 1 \cdot \Delta x_i = 1 \cdot (1 - (-1)) = 2
\]
Hence, the upper integral is $\dps \ol{\int_{-1}^1} f(x) \, dx = \inf_P U(f, P) = 2$.

\bigskip
Let $\eps > 0$. Choose a partition $P_\eps = \{-1, -\frac{\eps}{4}, \frac{\eps}{4}, 1\}$. The infimums $m_i$ for the three sub 
intervals are $m_1 = 1$, $m_2 = 0$ (since $0 \in [-\frac{\eps}{4}, \frac{\eps}{4}]$), and $m_3 = 1$. The lower sum is:
\[
L(f, P_\eps) = 1\left(1 - \frac{\eps}{4}\right) + 0\left(\frac{\eps}{2}\right) + 1\left(1 - \frac{\eps}{4}\right) = 2 - \frac{\eps}{2}
\]
Since $\dps L(f, P_\eps) \leq \ul{\int_{-1}^1} f(x) \, dx \leq 2$, and we can make $L(f, P_\eps)$ arbitrarily close to 2 by choosing a 
small $\eps$, the lower integral is $\dps \ul{\int_{-1}^1} f(x) \, dx = 2$.

Since the upper and lower integrals are equal:
\[
\underline{\int_{-1}^1} f(x) \, dx = \overline{\int_{-1}^1} f(x) \, dx = 2
\]
The function $f$ is Riemann integrable on $[-1, 1]$ and $\dps \int_{-1}^1 f(x) \, dx = 2$.