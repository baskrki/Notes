\section{Problem Set 5}
\begin{problem}
    Let $a_n,b_n$ be two sequence of real numbers and $x$ be a real number. The fourier series is 
    \[ \frac{a_0}{2}+\sum_{n=1}^{\infty} a_n \cos(nx)+b_n \sin(nx) \]
    Show that if $\displaystyle \sum_{n=1}^{\infty} \left(|a_n|+|b_n|\right)$ converges, then so does the fourier series.
\end{problem}

\textit{Solution.}

We can ignore the $\dps \frac{a_0}{2}$ term and just focus on the sum. Let 
\[ S_N(x) = \sum_{n=1}^{N} a_n \cos(nx) + b_n \sin(nx) \]

We will show that the sequence $\{S_N(x)\}$ is cauchy. Let $N,M \in \NN$ such that $M>N$.
\[ |S_M(x)-S_N(x)| = \sum_{n=N+1}^{M} a_n \cos(nx) + b_n \sin(nx) \le \sum_{n=N+1}^{M} |a_n| + |b_n| \]
As $N \to \infty$ the $\dps \sum_{n=N+1}^{M} |a_n| + |b_n| \to 0$. Thus the sequence is cauchy and from the 
$\textbf{cauchy convergence theorem}$ we know the series converges.

\begin{problem}
    Give an example of a function $f(x)$ defined on $[-1,1]$ with the following property:
    $(f(x))^2$ is continuous on $[-1,1]$ but $f(x)$ is not continuous on $[-1,1]$.
\end{problem}

\textit{Solution.} 

An example of such a function would be 
\[ f(x) = \begin{cases}
    1 & \text{if } x > 0 \\
    -1 & \text{if } x \le 0 
\end{cases} \]

\begin{problem}
    Determine at which points the function $f(x)=\floor{x}$ is continuous or discontinuous.
\end{problem}

\textit{Solution.}

Let $n \in \mathbb{Z}$. Suppose, for contradiction, that $\lfloor x \rfloor$ is continuous at $x = n$. Then, by definition, for every $\varepsilon > 0$ there exists $\delta > 0$ such that
\[
|x - n| < \delta \implies |\lfloor x \rfloor - \lfloor n \rfloor| < \varepsilon.
\]

Take $\varepsilon = \frac{1}{2}$. Consider $x = n - \frac{\delta}{2}$. Clearly, $|x - n| = \frac{\delta}{2} < \delta$, so it should satisfy the continuity condition.  

However,
\[
\lfloor x \rfloor = \lfloor n - \tfrac{\delta}{2} \rfloor = n-1
\]
and
\[
\lfloor n \rfloor = n.
\]

Thus,
\[
|\lfloor x \rfloor - \lfloor n \rfloor| = |(n-1) - n| = 1 > \varepsilon = \frac{1}{2},
\]
which is a contradiction. Hence, $\lfloor x \rfloor$ is \textbf{discontinuous at every integer}.

\bigskip

Let $x_0 \notin \mathbb{Z}$ and set $n = \lfloor x_0 \rfloor$. Then $n < x_0 < n+1$. Define
\[
\delta = \min\{x_0 - n,\, n+1 - x_0\} > 0.
\]

For any $x$ with $|x - x_0| < \delta$, we have
\[
n < x < n+1 \implies \lfloor x \rfloor = n = \lfloor x_0 \rfloor.
\]

Hence,
\[
|\lfloor x \rfloor - \lfloor x_0 \rfloor| = 0,
\]
proving that $\lfloor x \rfloor$ is \textbf{continuous at every non-integer}.

\begin{problem}
    Let $f(x)$ and $g(x)$ be two continuous functions defined on $\RR$ with
    $f(x) = g(x)$ for all $x \in \QQ$. Show that $f(x) = g(x)$ for all $x \in \RR$.
\end{problem}

\textit{Solution.} 

Let $x_n$ be a rational number such that 
\[ r - \frac{1}{n} < x_n < r+\frac{1}{n} \]
where $r$ is any irrational number. From our definition of $x_n$, we have that $x_n \to r$.
Thus,
\begin{align*}
    g(r)=g\left( \lim_{n \to \infty} x_n \right) &= \lim_{n \to \infty} g(x_n) \\
    &= \lim_{n \to \infty} f(x_n) \\
    &= f\left( \lim_{n \to \infty} x_n \right) \\
    &= f(r)
\end{align*}
Thus, $f(r)=g(r)$ for any $r \in \RR \setminus \QQ$ and therefore $f(x)=g(x)$ for any $x \in \RR$.

\begin{problem}
    Recall that
    \[ E(x) := 1+x+\frac{x^2}{2!} + \cdots + = \sum_{n=1}^{infty} \frac{x^n}{n!} \]
    \begin{enumerate}
        \item For $k \in \NN$, define
            \[ E_k(x) := 1+x+\frac{x^2}{2!} + \cdots + \frac{x^k}{k!} \]
            Show that $E_k(x)$ is continuous on $\RR$ for any $k \in \NN$.

        \item Let $M > 0$ be a fixed number. Show that for all $\eps > 0$, there exists $N \in \NN$ such that
            \[ |E_k (x) - E(x)| < \eps \]
            for all $k \ge N$ and for all $x \in [-M, M ]$.

            Remark: We require that a single number $N$ that works for all $x \in [-M, M ]$.

        \item Show that $E(x)$ is continuous on $\RR$.
    \end{enumerate}
\end{problem}

\textit{Solution.}

For $(1.)$ just use the fact that if $f$ is continuous then $c \cdot f$ is continuous for some fixed constant $c$.

\bigskip
For $(2.)$, let $M>0$ and fix $x\in[-M,M]$. Then $|x|\le M$. We have
\[
E(x)-E_k(x)=\sum_{n=k+1}^{\infty}\frac{x^n}{n!}.
\]
By the triangle inequality,
\[
|E(x)-E_k(x)|
\le \sum_{n=k+1}^{\infty}\frac{|x|^n}{n!}.
\]
Since $|x|\le M$, we have
\[
|E(x)-E_k(x)|
\le \sum_{n=k+1}^{\infty}\frac{M^n}{n!}
\]
The series $\dps \sum_{n=0}^{\infty}\frac{M^n}{n!}$ converges, hence its tail $\dps \sum_{n=k+1}^{\infty}\frac{M^n}{n!} \to 0$ as $k \to \infty$.
Therefore,
\[
\sup_{x\in[-M,M]} |E(x)-E_k(x)|
\le \sum_{n=k+1}^{\infty}\frac{M^n}{n!} < \eps
\]
for some $k \ge N$. Therefore,
\[ |E(x)-E_k(x)| < \eps \]
for $k \ge N$ and for all $x \in [-M,M]$. 

\bigskip

For $(3.)$, notice that

\[ |E(x) - E(x_0 )| \le |E(x) - E_k (x)| + |E_k (x) - E_k (x_0 )| + |E_k (x_0 ) - E(x_0 )| \]
Now, choose a $M > |x_0|$ such that $x \in [-M,M]$. Then, from $(2.)$,
\[ |E(x) - E_k (x)| < \frac{\eps}{3} \quad \text{and} \quad |E(x_0)-E_k(x_0)| < \frac{\eps}{3} \]
for large enough $k$. Also,
\[ |E_k (x) - E_k (x_0 )| < \frac{\eps}{3}\]
by $(1.)$ and thus,
\[ |E(x) - E_k (x)| + |E_k (x) - E_k (x_0 )| + |E_k (x_0 ) - E(x_0 )| < 3 \times \frac{\eps}{3} \]
\[ \implies |E(x) - E(x_0 )| < \eps \]