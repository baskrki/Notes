\section{Problem Set 3}

\begin{problem}
    Prove that for reals $x<y$, there exists a $r \in \QQ$ such that $x<r<y$.
\end{problem}

\textit{Solution.}
Using the archimedean property of $\RR$
\[ \exists n \in \NN, \text{ s.t } \frac{1}{n} < y-x \]
\[ \implies n(y-x)>1 \]
Since the difference is greater than $1$, there exists a integer $m$ such that
\[ nx < m < ny \Longrightarrow x < \frac{m}{n} < y  \]
Hence we are done.

\begin{problem}
    Let $E$ be an non-empty subset of $\RR$ which is bounded. Define
    \[ F := \set{x^2 \mid x \in E} \]
    Show that $\sup F$ exists and that $\sup F=\max\{(\sup E)^2,(\inf E)^2\}$.
\end{problem}

\textit{Solution.}

Since $E$ is bounded $F$ is also bounded. Then since $\inf E \le x \le \sup E$ we have
\[ x^2 \le \max\{(\sup E)^2,(\inf E)^2\} \]
If $\max\{(\sup E)^2,(\inf E)^2\}=(\sup E)^2$ then $\sup E \ge 0$ and $(\sup E)^2$ is an upper bound for $F$. Let $C$ be any 
upper bound for $F$. Then $\dps C \ge x^2 \Rightarrow \sqrt{C} \ge |x| \ge x$ for all $x \in E$. Hence $\sqrt{C}$ is an upper bound for $E$.
Thus, $\sqrt{C} \ge \sup E \Rightarrow C \ge (\sup E)^2$. It follows that $(\sup E)^2 = \sup F$.

\vspace{6pt}
Similarly if $\max\{(\sup E)^2,(\inf E)^2\}=(\inf E)^2$ then $\sup F= (\inf E)^2$.

\begin{problem}
    Let $E$ be an non-empty subset of $\RR$ which is bounded from above. Show that there exists a sequence $\{a_n\}$ such that
    $a_n \in E$ and $\dps \lim_{n \to \infty} a_n = \sup E$.
\end{problem}

\textit{Solution.}

We know that for each $\eps > 0$ there exists at least one $a \in E$ such that $a > \sup E-\eps$. Let $a_k \in E$ such that 
$\dps a_k > \sup E - \frac{1}{k}$. Thus, $\dps \frac{1}{k} > \sup E - a_k \ge 0$.
Thus the sequence $\{a_n\}$ converges to $\sup E$ and $\dps \lim_{n \to \infty} a_n = \sup E$.

\begin{problem}
    Let $a_1=4$ and define $a_n$ inductively by
    \[ a_n = 4 - \frac{4}{a_{n-1}} \text{ for } n \ge 2 \]
    Show that $\dps \lim_{n \to \infty} a_n = 2$.
\end{problem}

\textit{Solution.}
Using induction one can prove that
\[ a_n = 2+\frac{2}{n}, \text{ for } n \ge 1 \]

Thus, $\dps \lim_{n \to \infty} a_n = 2$.

\begin{problem}
Let $T : \mathbb{R} \to \mathbb{R}$ be a contraction map and $x \in \mathbb{R}$ be a number.
Define a sequence $a_n$ by requiring $a_1 = x$ and $a_{n+1} = T(a_n)$.

\begin{enumerate}
    \item Show that for any $m \in \mathbb{N}$, $|a_1 - a_m| \le \frac{1}{1-\lambda}|a_1 - a_2|$
    \item Show that $a_n$ is a Cauchy sequence.
\end{enumerate}
\end{problem}

\textit{Solution.}

Notice that,
\begin{align*}
    |a_n-a_{n+1}| &\le \lambda|a_{n-1}-a_{n}| \\
    &\le \lambda^2 |a_{n-2}-a_{n-1}| \\
    &\le \hspace{10mm}\vdots \\
    &\le \lambda^{n-1} |a_1-a_2|
\end{align*}

Therefore,
\begin{align*}
    |a_1-a_m| &\le |a_1-a_{m-1}| + |a_{m-1}-a_{m}| \\
    &\le |a_1 - a_{m-2}| + |a_{m-2}-a_{m-1}| + |a_{m-1}-a_{m}| \\  
    &\le \hspace{10mm} \vdots \\
    &\le |a_1-a_2| + \sum_{i=2}^{m-1} |a_i - a_{i+1}| \\
    &\le |a_1-a_2| + \sum_{i=2}^{m-1} \lambda^{i} |a_1-a_2| \\
    &\le |a_1-a_2| \left( \sum_{i=1}^{m-1} \lambda^i \right) \\
    &\le |a_1-a_2| \left(\frac{1-\lambda^{m}}{1-\lambda}\right) \le \left(\frac{1}{1-\lambda}\right)|a_1-a_2|
\end{align*}

To show that $\{a_n\}$ is a cauchy sequence,
\begin{align*}
    |a_n-a_m| \le 
\end{align*}