\section{Problem Set 3}

\begin{problem}
    Prove that for reals $x<y$, there exists a $r \in \QQ$ such that $x<r<y$.
\end{problem}

\textit{Solution.}
Using the archimedean property of $\RR$
\[ \exists n \in \NN, \text{ s.t } \frac{1}{n} < y-x \]
\[ \implies n(y-x)>1 \]
Since the difference is greater than $1$, there exists a integer $m$ such that
\[ nx < m < ny \Longrightarrow x < \frac{m}{n} < y  \]
Hence we are done.

\begin{problem}
    Let $E$ be an non-empty subset of $\RR$ which is bounded. Define
    \[ F := \set{x^2 \mid x \in E} \]
    Show that $\sup F$ exists and that $\sup F=\max\{(\sup E)^2,(\inf E)^2\}$.
\end{problem}

\textit{Solution.}

Since $E$ is bounded $F$ is also bounded. Then since $\inf E \le x \le \sup E$ we have
\[ x^2 \le \max\{(\sup E)^2,(\inf E)^2\} \]
If $\max\{(\sup E)^2,(\inf E)^2\}=(\sup E)^2$ then $\sup E \ge 0$ and $(\sup E)^2$ is an upper bound for $F$. Let $C$ be any 
upper bound for $F$. Then $\dps C \ge x^2 \Rightarrow \sqrt{C} \ge |x| \ge x$ for all $x \in E$. Hence $\sqrt{C}$ is an upper bound for $E$.
Thus, $\sqrt{C} \ge \sup E \Rightarrow C \ge (\sup E)^2$. It follows that $(\sup E)^2 = \sup F$.

\vspace{6pt}
Similarly if $\max\{(\sup E)^2,(\inf E)^2\}=(\inf E)^2$ then $\sup F= (\inf E)^2$.

\begin{problem}
    Let $E$ be an non-empty subset of $\RR$ which is bounded from above. Show that there exists a sequence $\{a_n\}$ such that
    $a_n \in E$ and $\dps \lim_{n \to infty} a_n = \sup E$.
\end{problem}