\begin{problem}
    Let $\FF$ be a ordered field with $1 \neq 0$. Show that $1 > 0$.
\end{problem}

\vspace{4mm}
\textit{Solution.}
First let us prove that $(-1)\cdot (-1)=1$. We know that for all $x \in \FF$, there is an inverse element $-x$ such that,
\[ x+(-x)=0 \]
Thus, $1+(-1)=0$ which means that 
\[ 0 = (-1) \cdot 0 = (-1) \cdot (1+(-1)) = (-1) \cdot 1 + (-1) \cdot (-1) = (-1) + (-1) \cdot (-1)   \]
\[ \implies 1 = (-1) \cdot (-1) \]

Since $\FF$ is a ordered field, one of the statement below must be true because of the \textbf{first axiom of order}.
\begin{align}
    1 < 0, \quad 1 = 0, \quad 0 < 1 
\end{align}
We assumed that $1 \neq 0$ so the middle statement can't be true and if $1 < 0$ then $0 < (-1)$. But from 
\textbf{axiom of order and multiplication} $0 < (-1)\cdot(-1) = 1$.
Thus a contradiction.

\begin{problem}
    Define the addition of two rational numbers  by 
    \[ \frac{n}{m} + \frac{p}{q} := \frac{nq+mp}{mq} .\]
    Show that it is well-defined.
\end{problem}

\vspace{4mm}
\textit{Solution.}
Suppose $\frac{n}{m}=\frac{n_1}{m_1}$ and $\frac{p}{q}=\frac{p_1}{q_1}$ then using the definition of when two rational numbers are equal,
we get $n m_1 = n_1 m$ and $pq_1=p_1q$. Thus,
\begin{align*}
    m_1 q_1 (nq + pm) &= m_1 q_1 nq + m_1 q_1 pm \\
    &= n_1 m q_1 q + p_1 q m_1 m \\
    &= mq (n_1 q_1 + m_1 p_1)
\end{align*}
Thus, $\frac{n}{m}+\frac{p}{q} = \frac{n_1}{m_1} + \frac{p_1}{q_1}$.

\begin{problem}
    Find the $\sup E$ and $\inf E$ for the following set $E$.
    \begin{enumerate}
        \item $E = \set{n \in \ZZ \mid n < \sqrt{12}}$
        \item $E = \set{r \in \QQ \mid r < \sqrt{12}}$
        \item $E = \set{x \in \RR \mid x^2-x-1 < 0}$
        \item $E = \set{\frac{n^2+n}{n+1} \mid n \in \NN}$
    \end{enumerate}
\end{problem}

\vspace{4mm}
\textit{Solution.}

\begin{enumerate}
    \item $\sup E = 3$ but $\inf E$ doesn't exist.
    \item $\sup E = \sqrt{12}$ but $\inf E$ doesn't exist.
    \item $\sup E = \frac{1+\sqrt{5}}{2}$ and $\inf E = \frac{1-\sqrt{5}}{2}$.
    \item $\inf E = 1$ but $\sup E$ doesn't exist.
\end{enumerate}

\begin{problem}
    Let $\mathbb{M}$ be the set of polynomials with integer coefficients i.e,
    \[ \mathbb{M} := \set{ f(x)=a_0+a_1x+ \cdots + a_n x^n \mid a_i \in \ZZ} \]
    Define the relation $0 \prec f$ if $0 < f(x)$ for $x$ large enough. More precisely, we say
\[
0 \prec f \quad \text{if there exists } M > 0 \text{ such that } f(x) > 0 \text{ for all } x > M.
\]
Then define
\[
f \prec g \quad \text{if } 0 \prec (g - f).
\]

Show that $(M, \prec)$ is an ordered set.
\end{problem}

\vspace{4mm}
\textit{Solution.}

We'll use the fact that for large enough $x$, $f(x)>0$ for $a_n >0$. Let $f,g \in \mathbb{M}$ such that $f \neq g$ and 
\[ f(x)=a_0+a_1x+ \cdots + a_n x^n \quad \text{and} \quad g(x)=b_0+b_1x+ \cdots + b_n x^n \] 
Define $h(x)=g(x)-f(x)=(b_0-a_0) + (b_1-a_1)x + \cdots + (b_n-a_n)x^n$ and let $c_i=b_i-a_i$.
Suppose $c_k$ is the highest degree non zero coefficient, then if
\begin{enumerate}
    \item $c_k > 0$ then $f \prec g$
    \item $c_k < 0$ then $g \prec f$
\end{enumerate}

Suppose $f \prec g$ and $g \prec h$. Then, $f(x) < g(x)$ and $g(x) < h(x)$ for large $x$ and hence $f(x) < h(x)$ for large $x$.
Thus, $0 \prec (h-f)$ which means $f \prec h$.

\begin{problem}
    Prove that $(\mathbb{M},\prec)$ \textbf{doesn't} satisfy the archimedean property.
\end{problem}

\vspace{4mm}
\textit{Solution.}

To show that it doesn't satisfy the archimedean property, we need to show that $\exists f,g \in \mathbb{M}$ such that
\[ \forall n \in \NN, g \not \prec nf \]
If we choose $g(x)=x^2$ and $f(x)=x$ and assume $g \prec nf$ then , $0 \prec x(n-x)$ which means
that $0 < x(n-x)$ which we know is false for large $x$. Thus, $g \not \prec nf$.

\begin{problem}
    Show that for any non-empty set $E \subset R$ which is bounded from below, $E$ has the greatest lower bound.
\end{problem}

\vspace{4mm}
\textit{Solution.}

To show that $E$ has greatest lower bound define 
\[ -E = \set{-x \mid x \in E} \]
If $\alpha$ is any lower bound of $E$ then $x \ge \alpha \Rightarrow -x \le -\alpha$. That means that $-E$ is bounded above by $-\alpha$.
Since $-E$ is bounded above by $-\alpha$, it must have the least upper bound property. Let 
\[ \sup -E = \beta \]
Thus, $\beta \ge -x \Rightarrow x \ge -\beta$ and $\beta \le - \alpha$ for any lower bound $\alpha$ of $E$. Thus, $\alpha \le -\beta$.
Hence, $-\beta = \inf E$.

\begin{problem}
    Show that for any real number $x \in \RR$ there exists a real number $y \in \RR$ such that $y^3 = x$.
\end{problem}

\vspace{4mm}
\textit{Solution.}

Let us define 
\[ A = \set{a \mid a > 0 \text{ and } a^3 \le x} \]
for $x>0$. Let $y = \sup A$. We will show that $y^3=x$.

\vspace{6pt}
Suppose $x > y^3$. Let $h=\min\{\frac{1}{2}, \frac{x-y^3}{3y^2+3y+1}\}$ then,
\begin{align*}
    (y+h)^3 &= y^3 + h^3 + 3yh^2 + 3y^2h \\
    &< y^3 + h(1+3y+3y^2) \\
    &\le y^3 +   \frac{x-y^3}{3y^2+3y+1} \cdot (1+3y+3y^2) \\
    &\le x
\end{align*}
Thus, $y+h \in A$ but since $y$ is the least upper bound of $A$ we have, $y+h \le y \Rightarrow h \le 0$. This is a contradiction.

\vspace{6pt}
Suppose $x < y^3$. Let $h =\frac{y^3-x}{3y^2}$ then
\begin{align*}
    (y-h)^3 &= y^3-h^3 - 3y^2h + 3yh^2 \\
    &= y^3 - h^3 -3y^2 \cdot \frac{(y^3-x)}{3y^2} + 3yh^2 \\
    &= x-h^3 + 3yh^2
\end{align*}
Since $3yh^2-h^3 > 0$ we have $(y-h)^3 > x$. Thus $(y-h)$ is an upper bound for $A$. Thus $y \le y-h \Rightarrow h \le 0$. Contradiction!!